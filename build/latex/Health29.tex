%% Generated by Sphinx.
\def\sphinxdocclass{report}
\documentclass[letterpaper,10pt,english]{sphinxmanual}
\ifdefined\pdfpxdimen
   \let\sphinxpxdimen\pdfpxdimen\else\newdimen\sphinxpxdimen
\fi \sphinxpxdimen=.75bp\relax

\PassOptionsToPackage{warn}{textcomp}
\usepackage[utf8]{inputenc}
\ifdefined\DeclareUnicodeCharacter
% support both utf8 and utf8x syntaxes
  \ifdefined\DeclareUnicodeCharacterAsOptional
    \def\sphinxDUC#1{\DeclareUnicodeCharacter{"#1}}
  \else
    \let\sphinxDUC\DeclareUnicodeCharacter
  \fi
  \sphinxDUC{00A0}{\nobreakspace}
  \sphinxDUC{2500}{\sphinxunichar{2500}}
  \sphinxDUC{2502}{\sphinxunichar{2502}}
  \sphinxDUC{2514}{\sphinxunichar{2514}}
  \sphinxDUC{251C}{\sphinxunichar{251C}}
  \sphinxDUC{2572}{\textbackslash}
\fi
\usepackage{cmap}
\usepackage[T1]{fontenc}
\usepackage{amsmath,amssymb,amstext}
\usepackage{babel}



\usepackage{times}
\expandafter\ifx\csname T@LGR\endcsname\relax
\else
% LGR was declared as font encoding
  \substitutefont{LGR}{\rmdefault}{cmr}
  \substitutefont{LGR}{\sfdefault}{cmss}
  \substitutefont{LGR}{\ttdefault}{cmtt}
\fi
\expandafter\ifx\csname T@X2\endcsname\relax
  \expandafter\ifx\csname T@T2A\endcsname\relax
  \else
  % T2A was declared as font encoding
    \substitutefont{T2A}{\rmdefault}{cmr}
    \substitutefont{T2A}{\sfdefault}{cmss}
    \substitutefont{T2A}{\ttdefault}{cmtt}
  \fi
\else
% X2 was declared as font encoding
  \substitutefont{X2}{\rmdefault}{cmr}
  \substitutefont{X2}{\sfdefault}{cmss}
  \substitutefont{X2}{\ttdefault}{cmtt}
\fi


\usepackage[Bjarne]{fncychap}
\usepackage{sphinx}

\fvset{fontsize=\small}
\usepackage{geometry}


% Include hyperref last.
\usepackage{hyperref}
% Fix anchor placement for figures with captions.
\usepackage{hypcap}% it must be loaded after hyperref.
% Set up styles of URL: it should be placed after hyperref.
\urlstyle{same}
\addto\captionsenglish{\renewcommand{\contentsname}{H29}}

\usepackage{sphinxmessages}
\setcounter{tocdepth}{1}


\setcounter{tocdepth}{2}
\definecolor{VerbatimColor}{rgb}{0.15,0.16,0.13}


\title{Health29 Documentation}
\date{Apr 16, 2021}
\release{1.0}
\author{Foundation29}
\newcommand{\sphinxlogo}{\vbox{}}
\renewcommand{\releasename}{Release}
\makeindex
\begin{document}

\pagestyle{empty}
\sphinxmaketitle
\pagestyle{plain}
\sphinxtableofcontents
\pagestyle{normal}
\phantomsection\label{\detokenize{index::doc}}
\noindent\sphinxincludegraphics[width=0\sphinxpxdimen]{{Foundation29}.png}

\noindent\sphinxincludegraphics[width=0\sphinxpxdimen]{{Environments}.jpg}

\noindent\sphinxincludegraphics[width=0\sphinxpxdimen]{{AzureSlots}.jpg}

\noindent\sphinxincludegraphics[width=0\sphinxpxdimen]{{applicationPhases}.jpg}

\noindent\sphinxincludegraphics[width=0\sphinxpxdimen]{{gitflow}.jpg}

\noindent\sphinxincludegraphics[width=0\sphinxpxdimen]{{Repositories}.jpg}

\noindent\sphinxincludegraphics[width=0\sphinxpxdimen]{{Build_deploy}.jpg}

\noindent\sphinxincludegraphics[width=0\sphinxpxdimen]{{pipelines_server_1}.jpg}

\noindent\sphinxincludegraphics[width=0\sphinxpxdimen]{{pipelines_server_2}.jpg}

\noindent\sphinxincludegraphics[width=0\sphinxpxdimen]{{pipelines_server_3}.jpg}

\noindent\sphinxincludegraphics[width=0\sphinxpxdimen]{{pipelines_server_4}.jpg}

\noindent\sphinxincludegraphics[width=0\sphinxpxdimen]{{pipelines_server_5}.jpg}

\noindent\sphinxincludegraphics[width=0\sphinxpxdimen]{{pipelines_client_1}.jpg}

\noindent\sphinxincludegraphics[width=0\sphinxpxdimen]{{pipelines_client_2}.jpg}

\noindent\sphinxincludegraphics[width=0\sphinxpxdimen]{{pipelines_client_3}.jpg}

\noindent\sphinxincludegraphics[width=0\sphinxpxdimen]{{pipelines_client_4}.jpg}

\noindent\sphinxincludegraphics[width=0\sphinxpxdimen]{{H29C4model}.png}

\noindent\sphinxincludegraphics[width=0\sphinxpxdimen]{{SystemContext_intro}.jpg}

\noindent\sphinxincludegraphics[width=0\sphinxpxdimen]{{funcionalities_H29}.jpg}

\noindent\sphinxincludegraphics[width=0\sphinxpxdimen]{{containersH29}.jpg}

\noindent\sphinxincludegraphics[width=0\sphinxpxdimen]{{Webapp}.jpg}

\noindent\sphinxincludegraphics[width=0\sphinxpxdimen]{{Webapp_frameworks_architecture}.jpg}

\noindent\sphinxincludegraphics[width=0\sphinxpxdimen]{{F29API_component}.jpg}

\noindent\sphinxincludegraphics[width=0\sphinxpxdimen]{{monarchAPI_component}.jpg}

\noindent\sphinxincludegraphics[width=0\sphinxpxdimen]{{cognitiveServices_component}.jpg}

\noindent\sphinxincludegraphics[width=0\sphinxpxdimen]{{healthbot_component}.jpg}

\noindent\sphinxincludegraphics[width=0\sphinxpxdimen]{{blobs_component}.jpg}

\noindent\sphinxincludegraphics[width=0\sphinxpxdimen]{{ddbb_component}.jpg}

\noindent\sphinxincludegraphics[width=0\sphinxpxdimen]{{Scenarios}.jpg}

\noindent\sphinxincludegraphics[width=0\sphinxpxdimen]{{botflow}.png}

\noindent\sphinxincludegraphics[width=0\sphinxpxdimen]{{botscenarios}.jpg}

\noindent\sphinxincludegraphics[width=0\sphinxpxdimen]{{H29_client_code}.jpg}

\noindent\sphinxincludegraphics[width=0\sphinxpxdimen]{{Health29_client_code_app}.jpg}

\noindent\sphinxincludegraphics[width=0\sphinxpxdimen]{{H29_server_code}.jpg}

\noindent\sphinxincludegraphics[width=0\sphinxpxdimen]{{F29API_code}.jpg}

\noindent\sphinxincludegraphics[width=0\sphinxpxdimen]{{DDBB_Division}.jpg}

\noindent\sphinxincludegraphics[width=0\sphinxpxdimen]{{ddbb_relationship}.jpg}

\noindent\sphinxincludegraphics[width=0\sphinxpxdimen]{{Multilanguage_assets}.png}

\noindent\sphinxincludegraphics[width=0\sphinxpxdimen]{{groups_section}.jpg}

\noindent\sphinxincludegraphics[width=0\sphinxpxdimen]{{mySymptoms_section1}.jpg}

\noindent\sphinxincludegraphics[width=0\sphinxpxdimen]{{mySymptoms_section2}.jpg}

\noindent\sphinxincludegraphics[width=0\sphinxpxdimen]{{languages_section}.jpg}

\noindent\sphinxincludegraphics[width=0\sphinxpxdimen]{{manageFAQ_section}.jpg}

\noindent\sphinxincludegraphics[width=0\sphinxpxdimen]{{manageFAQ_section2}.jpg}

\noindent\sphinxincludegraphics[width=0\sphinxpxdimen]{{manageFAQ_section3}.jpg}

\noindent\sphinxincludegraphics[width=0\sphinxpxdimen]{{manageFAQ_section4}.jpg}

\noindent\sphinxincludegraphics[width=0\sphinxpxdimen]{{proms_section1}.jpg}

\noindent\sphinxincludegraphics[width=0\sphinxpxdimen]{{proms_section2}.jpg}

\noindent\sphinxincludegraphics[width=0\sphinxpxdimen]{{proms_section3}.jpg}

\noindent\sphinxincludegraphics[width=0\sphinxpxdimen]{{proms_section4}.jpg}

\noindent\sphinxincludegraphics[width=0\sphinxpxdimen]{{translations_section}.jpg}

\noindent\sphinxincludegraphics[width=0\sphinxpxdimen]{{myclinicalhistory_section}.jpg}

\noindent\sphinxincludegraphics[width=0\sphinxpxdimen]{{myclinicalhistory_section2}.jpg}

\noindent\sphinxincludegraphics[width=0\sphinxpxdimen]{{myclinicalhistory_section3}.jpg}

\noindent\sphinxincludegraphics[width=0\sphinxpxdimen]{{myclinicalhistory_section4}.jpg}

\noindent\sphinxincludegraphics[width=0\sphinxpxdimen]{{notifications_section}.jpg}

\noindent\sphinxincludegraphics[width=0\sphinxpxdimen]{{notifications_section2}.jpg}

\noindent\sphinxincludegraphics[width=0\sphinxpxdimen]{{notifications_section3}.jpg}

\noindent\sphinxincludegraphics[width=0\sphinxpxdimen]{{support_section}.jpg}

\noindent{\hspace*{\fill}\sphinxincludegraphics[width=100\sphinxpxdimen]{{Foundation29}.png}}



\sphinxstylestrong{We are a technology Foundation with focus on rare diseases​}

We create health decision support systems by obtaining medical information produced by patients and physicians. We analyze that information to generate intelligence that can help make better decisions.​

\noindent{\hspace*{\fill}\sphinxincludegraphics[width=800\sphinxpxdimen]{{Health29_app_schema}.png}\hspace*{\fill}}

This document presents a technical description of the Health29 software architecture.




\chapter{1. Environments}
\label{\detokenize{pages/Environments:environments}}\label{\detokenize{pages/Environments::doc}}
The different phases defined for software development are followed:



In the development phase, it will be achieved:
\begin{itemize}
\item {} 
Integrate the work of different developers in a central repository, resulting in an updated and consolidated version of the code.

\item {} 
Automate the integration tests and their validation before being moved to the next environment.

\item {} 
Send the code to the next environment if the tests have been passed successfully.

\end{itemize}

Once the integration tests in the development environment have been passed, the code will be moved to the pre\sphinxhyphen{}production (test) environment. Here the validation tests will be performed to the whole software, with the objective of locating any error before reaching the production environment and thus avoid problems arising from them. This environment can also be used as a demo environment, where the final customer can test the new application or the modifications or corrections made to the existing application. From here, the customer’s impressions will be extracted and possible deficiencies in the initial requirements, in the design or in its implementation will be located early on.
Finally, the version will be moved into the production environment.

So, the Health29 platform is divided into three environments:
\begin{itemize}
\item {} 
The \sphinxstylestrong{development} environment, for programmers.

\item {} 
The \sphinxstylestrong{test} environment for the pre\sphinxhyphen{}production phase.

\item {} 
The \sphinxstylestrong{production} environment for users.

\end{itemize}



On the one hand, These environments are created as \sphinxhref{https://docs.microsoft.com/en-us/azure/azure-functions/functions-deployment-slots}{slots of an Azure App Service}. Deployment slots are live apps with their own hostnames.
In particular, you can access the \sphinxstylestrong{Health29’s Azure App Service} through the following \sphinxhref{https://portal.azure.com/\#@foundation29outlook.onmicrosoft.com/resource/subscriptions/53348303-e009-4241-9ac7-a8e4465ece27/resourceGroups/health29/providers/Microsoft.Web/sites/health29/appServices}{link}.

On the other hand, each environment will use specific or common resources or containers for all. Thus:
\begin{itemize}
\item {} 
The client\sphinxhyphen{}server webapp will be located in a different repository for each environment.

\item {} 
The healthbot will also be specific to each environment

\item {} 
There will be two database containers: the test and development environments will share the resource while production will have its own.

\item {} 
The rest of the resources will be shared by all the environments.

\end{itemize}

We will not go into more detail on this since the functionality, architecture and usability of each of these containers will be explained in section 2 of this document.

In this section we will only explain the architecture of the platform environments, the distribution of the webapp repositories and the steps to follow for the build and the deploy of the platform.

Therefore, the section will be divided into different sections:
\begin{itemize}
\item {} 
Design and implementation of the Azure App Service architecture.

\item {} 
Division of the code repositories for each environment

\item {} 
Build and deploy of each environment

\end{itemize}




\chapter{1.1. Azure App service: Health29}
\label{\detokenize{pages/Environments/Azure_App_Services:azure-app-service-health29}}\label{\detokenize{pages/Environments/Azure_App_Services::doc}}
As already mentioned, Health29 is implemented as an Azure App Service divided into slots.

First we have to create an Azure App service to locate our webapp, and then we will make the division into slots to create and configure the different environments.

In particular, for the Health29 platform we have created a Web application type App service: \sphinxhref{https://portal.azure.com/\#@foundation29outlook.onmicrosoft.com/resource/subscriptions/53348303-e009-4241-9ac7-a8e4465ece27/resourceGroups/health29/providers/Microsoft.Web/sites/health29/appServices}{health29}

The steps to follow would be:
\begin{enumerate}
\sphinxsetlistlabels{\arabic}{enumi}{enumii}{}{.}%
\item {} 
Enter the Azure portal with the account you have active, go to the option “All services”, look for the section “Web and mobile”, select the service “App Service”.
We will see a list of services in case you have done any previously, otherwise we have an empty panel, click on the option “Add” and Azure shows us a catalog of all services available for web and mobile.

\item {} 
Select the web application without SQL as a database.

\item {} 
We have a small summary of the functions of the service, the options for implementation with other technologies from Microsoft and third parties, we click on “Create”.

\item {} 
It is necessary to add a name to the service that is unique, because we are using a shared service (.azurewebsites.net) We can select the operating system of the server, filling in the fields we select “Create”.

\end{enumerate}

Once we have the App service created, we proceed to divide it to configure the different work environments. For this, as we have already said, we are going to use the slots:
\begin{quote}
\begin{quote}

Deployment slots let you deploy different versions of your web app to different URLs. You can test a certain version and then swap content and configuration between slots.
\end{quote}
\end{quote}

In general,
\begin{itemize}
\item {} 
Each deployment slot is like a full\sphinxhyphen{}fledged App Service instance.

\item {} 
The original App Service deployment slot is also called the production slot.

\item {} 
Deployment slots can copy the configuration (AppSettings and Connectionstrings) of the original App Service or other deployment slots.

\item {} 
When you scale a deployment slot (up or out), you also scale all the other slots of the App Service. This is because all slots share the same App Service Plan.

\item {} 
Because all deployments lots run within the same App Service and its App Service Plan, deployment slots are free to use if you are using the Standard pricing tier or higher.

\item {} 
If you have installed any site extensions, you need to do that again in a deployment slot, as it is a new App Service instance.

\item {} 
Deployment slots have a different URL than the original App Service. This URL is based on the name you give the deployment slot.

\end{itemize}

The slots that have been created and configured in Health29 are:



Each of these slots will correspond to a platform environment, with the aim of achieving an orderly workflow between teams:
\begin{itemize}
\item {} 
Developers or programmers will be able to work on environment to incorporate the new functionalities or correct errors without this directly affecting the users.

\item {} 
There will be a pre\sphinxhyphen{}production environment to be able to validate the tasks performed or even make demos to clients to collect feedback and establish future lines on which to work.

\item {} 
Finally, this will ensure that the production environment or the one used by customers/users is stable.

\end{itemize}

The steps to follow to create and configure slots are summarized in the official \sphinxhref{https://docs.microsoft.com/en-us/azure/azure-functions/functions-deployment-slots}{Microsoft document}.It is a simple task, just select in the App Service “Deployment slots” and click on add a new one. It opens the Add Slot dialog, to the right of the screen where you need to enter a name, and choose if you want to copy the configuration from another slot, or not at all. Choosing to copy the configuration is important because your the application does things like connect to the databases and has specific connection strings in these settings.

In addition, the following domains have been configured as indicated in the following \sphinxhref{https://docs.microsoft.com/en-us/azure/app-service/manage-custom-dns-buy-domain}{link}:
\begin{itemize}
\item {} 
\sphinxhref{https://www.health29.org/login}{www.health29.org}

\item {} 
\sphinxhref{https://health29-dev.azurewebsites.net/login}{health29\sphinxhyphen{}dev.azurewebsites.net}

\item {} 
\sphinxhref{https://health29-test.azurewebsites.net/login}{health29\sphinxhyphen{}test.azurewebsites.net}

\end{itemize}




\chapter{1.2. Code repositories}
\label{\detokenize{pages/Environments/Code_repository:code-repositories}}\label{\detokenize{pages/Environments/Code_repository::doc}}
The code is open source, and is public on GitHub.
\begin{itemize}
\item {} 
\sphinxhref{https://github.com/foundation29org/H29\_Client}{Client} and \sphinxhref{https://github.com/foundation29org/H29\_Server}{Server}

\end{itemize}

Each environment will be composed of two projects: client and server, using a REST architecture.
Therefore, it will have the characteristics of this type of implementation. Among them, we can mainly highlight:

\sphinxstylestrong{Uniform interface}
\begin{itemize}
\item {} 
The interface is resource\sphinxhyphen{}based.

\item {} 
The server will send the data, so that communication with the databases will be transparent to the client.

\item {} 
The representation of the resource that arrives to the client, will be enough to be able to interact with the external resource, assuming that it has permissions.

\end{itemize}

\sphinxstylestrong{Cacheable}
\begin{itemize}
\item {} 
On the web, clients can cache server responses

\item {} 
Answers should be implicitly or explicitly marked as searchable or uncheckable.

\item {} 
In future requests, the client will know whether or not it can reuse the data it has already obtained.

\item {} 
Saving requests will improve application scalability and client performance (mainly avoiding latency).

\end{itemize}

\sphinxstylestrong{Client and server separation}
\begin{itemize}
\item {} 
The client and server are separated, their union is through the uniform interface

\item {} 
The developments in frontend and backend are done separately, taking into account the API.

\item {} 
As long as the interface doesn’t change, we can change the client or the server without problems.

\end{itemize}

For all this, it is advisable to follow \sphinxhref{https://stackoverflow.blog/2020/03/02/best-practices-for-rest-api-design/}{good practices in programming REST applications} for the maintenance and scaling of the platform.

The architecture of the client\sphinxhyphen{}server code will be explained in more detail in section 2 of this document, but here it is important to note that the client is implemented in \sphinxhref{https://angular.io/}{Angular 5} and the server in \sphinxhref{https://nodejs.org/en/}{node.js}.

For the development and versioning of the code we will follow the GitFlow defined in this \sphinxhref{https://blog.axosoft.com/gitflow/}{guide}.



It consists of a division by branches according to the task to be performed. Like this:
\begin{itemize}
\item {} 
The master branch will contain the latest version of the code uploaded to production.

\item {} 
The develop branch will be used to prepare the next versions. It will be the main branch on which the developers will work.

\item {} 
Feature branches will be created to implement the functionality of the tasks defined in each sprint.

\item {} 
Hotfixes branches will be created to implement the bugs defined in each Sprint.

\item {} 
The releases branch will serve as a link between the development and master branches. It will be the one on which the following versions will be built and added to the production.

\end{itemize}

Thus, there will be a relationship between each of the branches:
\begin{itemize}
\item {} 
The Develop branch is generated from the Master branch. This is part of the latest production version when starting a Sprint.

\item {} 
Features branches are created from the Develop branch. A branch will be created for each Sprint Task.

\item {} 
From the Master branch the Hotfixes branches are created where the Sprint bugs will be resolved.

\item {} 
The releases branch will be the union between Develop and Master, so that to move to Master (new version) you will first have to move from Develop to the releases branch, and then perform a validation phase. When this validation phase is finished, you can move to the Master branch.

\end{itemize}




\chapter{1.3. Build and deploy environments}
\label{\detokenize{pages/Environments/Build_deploy:build-and-deploy-environments}}\label{\detokenize{pages/Environments/Build_deploy::doc}}
The architecture of our Webapp follows the client\sphinxhyphen{}server model. So, we use:
\begin{itemize}
\item {} 
Angular 5 framework for the client.

\item {} 
Nodejs for the server.

\end{itemize}

According to this, different commands will be used to perform the build and deploy tasks of each of the environments.

\begin{sphinxVerbatim}[commandchars=\\\{\}]
\PYG{l+s+s2}{\PYGZdq{}}\PYG{l+s+s2}{scripts}\PYG{l+s+s2}{\PYGZdq{}}\PYG{p}{:} \PYG{p}{\PYGZob{}}
    \PYG{l+s+s2}{\PYGZdq{}}\PYG{l+s+s2}{ng}\PYG{l+s+s2}{\PYGZdq{}}\PYG{p}{:} \PYG{l+s+s2}{\PYGZdq{}}\PYG{l+s+s2}{ng}\PYG{l+s+s2}{\PYGZdq{}}\PYG{p}{,}
    \PYG{l+s+s2}{\PYGZdq{}}\PYG{l+s+s2}{start}\PYG{l+s+s2}{\PYGZdq{}}\PYG{p}{:} \PYG{l+s+s2}{\PYGZdq{}}\PYG{l+s+s2}{ng serve}\PYG{l+s+s2}{\PYGZdq{}}\PYG{p}{,}
    \PYG{l+s+s2}{\PYGZdq{}}\PYG{l+s+s2}{startLocal}\PYG{l+s+s2}{\PYGZdq{}}\PYG{p}{:} \PYG{l+s+s2}{\PYGZdq{}}\PYG{l+s+s2}{ng serve \PYGZhy{}\PYGZhy{}env=local}\PYG{l+s+s2}{\PYGZdq{}}\PYG{p}{,}
    \PYG{l+s+s2}{\PYGZdq{}}\PYG{l+s+s2}{build}\PYG{l+s+s2}{\PYGZdq{}}\PYG{p}{:} \PYG{l+s+s2}{\PYGZdq{}}\PYG{l+s+s2}{ng build \PYGZhy{}\PYGZhy{}env=local}\PYG{l+s+s2}{\PYGZdq{}}\PYG{p}{,}
    \PYG{l+s+s2}{\PYGZdq{}}\PYG{l+s+s2}{buildDev}\PYG{l+s+s2}{\PYGZdq{}}\PYG{p}{:} \PYG{l+s+s2}{\PYGZdq{}}\PYG{l+s+s2}{ng build \PYGZhy{}\PYGZhy{}env=dev}\PYG{l+s+s2}{\PYGZdq{}}\PYG{p}{,}
    \PYG{l+s+s2}{\PYGZdq{}}\PYG{l+s+s2}{buildStaging}\PYG{l+s+s2}{\PYGZdq{}}\PYG{p}{:} \PYG{l+s+s2}{\PYGZdq{}}\PYG{l+s+s2}{ng build \PYGZhy{}\PYGZhy{}env=test}\PYG{l+s+s2}{\PYGZdq{}}\PYG{p}{,}
    \PYG{l+s+s2}{\PYGZdq{}}\PYG{l+s+s2}{buildProd}\PYG{l+s+s2}{\PYGZdq{}}\PYG{p}{:} \PYG{l+s+s2}{\PYGZdq{}}\PYG{l+s+s2}{ng build \PYGZhy{}\PYGZhy{}env=prod}\PYG{l+s+s2}{\PYGZdq{}}\PYG{p}{,}
    \PYG{l+s+s2}{\PYGZdq{}}\PYG{l+s+s2}{test}\PYG{l+s+s2}{\PYGZdq{}}\PYG{p}{:} \PYG{l+s+s2}{\PYGZdq{}}\PYG{l+s+s2}{ng test}\PYG{l+s+s2}{\PYGZdq{}}\PYG{p}{,}
    \PYG{l+s+s2}{\PYGZdq{}}\PYG{l+s+s2}{lint}\PYG{l+s+s2}{\PYGZdq{}}\PYG{p}{:} \PYG{l+s+s2}{\PYGZdq{}}\PYG{l+s+s2}{ng lint}\PYG{l+s+s2}{\PYGZdq{}}\PYG{p}{,}
    \PYG{l+s+s2}{\PYGZdq{}}\PYG{l+s+s2}{e2e}\PYG{l+s+s2}{\PYGZdq{}}\PYG{p}{:} \PYG{l+s+s2}{\PYGZdq{}}\PYG{l+s+s2}{ng e2e}\PYG{l+s+s2}{\PYGZdq{}}
  \PYG{p}{\PYGZcb{}}
\end{sphinxVerbatim}

The following image shows a scheme with the steps to perform the different tasks ( execution, build and deploy) in a particular environment, “x”.



As we are working with gitflow each branch of the repository corresponds to a specific environment and therefore a different script will be added to perform these tasks on each of the environments.

In this section we will explain the procedure to follow to perform each of these tasks. Therefore, the following section has been divided into sections to detail the specific steps for each of them.


\section{1.3.1. Execution.}
\label{\detokenize{pages/Environments/Build_deploy:execution}}
To run one of the application environments on our machine and be able to perform the implementation and testing tasks, based on the technologies chosen in our client\sphinxhyphen{}server model, we have to perform the following steps:
\begin{itemize}
\item {} 
The server is designed in nodejs v10.16.3, so it will be executed from the node console with the corresponding command.

\item {} 
As the client is designed in Angular 5, we will use \sphinxhref{https://angular.io/cli}{Angular Cli} executing the corresponding script.

\end{itemize}

So, the first thing to do is to launch the server with “npm run serve”.

Then, we will see how to launch the client. To do this, first it is necessary to understand the \sphinxhref{https://angular.io/guide/workspace-config}{configuration and architecture of the workspace of Angular’s projects}. In the file package.json we will find the different scripts that have been designed to perform different tasks on the project, such as: the execution, the testing or the build of it.

In this case, to execute the client of our application we use the command: \sphinxhref{https://angular.io/cli/serve}{ng serve}.

With this, we will have the Health29 application running at http://localhost:4200/.

To execute the application of each of the available environments, it will be enough to checkout the corresponding branch. This will work on the code of that branch and therefore what will be deployed in the corresponding environment.


\section{1.3.2. Build: scripts.}
\label{\detokenize{pages/Environments/Build_deploy:build-scripts}}
The build is done on the Angular client. As seen in the previous sections, using the scripts defined in the package.json file, we could build and deploy the application from local memory. But when we are ready to \sphinxhref{https://angular.io/guide/deployment}{deploy} in a environment, we must use the \sphinxhref{https://angular.io/cli/build}{ng build} corresponding command to build the app and deploy the build artifacts elsewhere.

So that, for each environment we must use:
\begin{itemize}
\item {} 
Local environment: \sphinxstylestrong{“npm run build”}. For that use the code that you want.

\item {} 
Development environment: \sphinxstylestrong{“npm run buildDev”}. For that use the code of \sphinxstylestrong{develop} branch.

\item {} 
Test enviroment: \sphinxstylestrong{“npm run buildStaging”}. For that use the code of \sphinxstylestrong{release} branch.

\item {} 
Production environment: \sphinxstylestrong{“npm run buildProd”}. For that use the code of \sphinxstylestrong{main} branch.

\end{itemize}

Both ng build and ng serve clear the output folder before they build the project, but only the ng build command writes the generated build artifacts to the output folder.

The output folder is dist/project\sphinxhyphen{}name/ by default. To output to a different folder, change the outputPath in angular.json.

As we saw in the diagram shown at the beginning of this section, by runningnpm run buildX we minimize and compress the code. This command will create a new folder called dist, and it will have the project optimized. This dist folder will be the one uploaded to the root of the node server that manages the API. It will also be used to create the mobile apps, adding it to the www folder of the cordoba project.


\section{1.3.3. Deploy.}
\label{\detokenize{pages/Environments/Build_deploy:deploy}}
The deploy tasks can be executed according to gitflow structure of the project. That is, you can configure the listening of the push or commits in each branch of the project according to a specific environment to automate the deploy of this environment.According to the build that has been executed, a different dist folder will be generated and used by the server. In addition to this, the “config.js” file has to be modified to match the secrets or keys of the corresponding environment.

As our server is created in node.js we can use \sphinxhref{https://docs.microsoft.com/en-us/azure/devops/pipelines/?view=azure-devops}{Azure Pipelines} for deploy our App Service. This configuration is explained in the next section.


\section{1.3.4. Semiutomation of processes: Azure Pipelines.}
\label{\detokenize{pages/Environments/Build_deploy:semiutomation-of-processes-azure-pipelines}}
To deploy one of the environments, we could use \sphinxhref{https://f29.visualstudio.com/Health29/\_build}{Azure Pipelines}.

The architecture proposed consists of the construction of CI/CD pipelines which helps you automate steps in your software delivery process, such as initiating code builds, running automated tests, and deploying to a staging or production environment. Automated pipelines remove manual errors, provide standardized development feedback loops and enable fast product iterations.

Thus, the corresponding pipelines for build and deploy the environment with the changes made should be executed according to gitflow. That is, we have 6 pipelines available in the project: one for the client and one for the server, of develop, test and production environments.

The deployment of the development and test environments will be automatic, so that the pipelines will have the Trigger parameters configured to be launched automatically when a push is made on the corresponding branches. While the production environment will be launched manually to prevent possible errors as it is a critical task.

For convenience, both build and deploy tasks have been added to these pipelines. In this way, when you want to update any of the environments, you only need to execute the client and server pipelines, in that order. It is important to respect the order since when the build tasks are included, as indicated in the previous sections, it will be necessary to update the node server root with the content of the last client compilation.

It is important to emphasize here that the title of the section indicated that we make a semiautomation of the processes, this is due to the fact that the execution of the production pipelines is manual.


\subsection{1.3.4.1. Setup Azure Pipelines.}
\label{\detokenize{pages/Environments/Build_deploy:setup-azure-pipelines}}
Starting from an azure app service and an azure devops project created, the pipelines are implemented as explained in this \sphinxhref{https://www.c-sharpcorner.com/article/hosting-angular-application-on-azure-with-cicd/}{document} with some modifications since, as we have said, the execution of the production pipelines will be manual.


\subsubsection{1.3.4.1.1. Pipeline for Server code}
\label{\detokenize{pages/Environments/Build_deploy:pipeline-for-server-code}}
Access Azure/ App Service/app service name/Deployment slots:



Select the slot you are going to work on. Go to deployment center (if the slot was already deployed, click on disconnect):



Click on azure repos



Select azure pipelines and configure and finish.
Now go to azure devops/pipelines. A new pipeline should be created and we must replace the content of the /dist folder by the last compiled version of the client






\subsubsection{1.3.4.1.2. Pipeline for Client code}
\label{\detokenize{pages/Environments/Build_deploy:pipeline-for-client-code}}
Go to Azure DevOps. In the left navigation, select ‘Pipelines’. Then click ‘Builds’. Now you can see, ‘+ New’ drop down. Click it and in the list, select ‘New build pipeline’.



Then it will take you to select the repository, where your project is hosted (Github option). Here select ‘Use the Classic Editor’.



Then after, you will take to select the Project, and the Branch, which the build pipeline works on.

Next it asks to select template. Select, ‘Empty Job’.

Now you can see the screen, where you have to say the steps to take, to build the project. And with this we would have the Pipeline created. Now we have to add Task to perform the different functions that we want to execute in the call to this Pipeline.

In general, the first thing to do is to indicate that we are going to use Node and its version and download all the libraries we need:
\begin{quote}
\begin{itemize}
\item {} 
Install Node: Type: ‘Node.js’ in the Search box. Select Node.js tool installer and configure it with selecting a version (We will use 8.9.3 version of Node.js)

\item {} 
When we have installed Node.js we can use the commands. So, type ‘npm’, in the Search box. Add ‘npm’. Then you can see the panel, where you can define the Task.

\end{itemize}
\begin{itemize}
\item {} 
Display Name — Name of the Task. ( Display Purpose)

\item {} 
Command — What the task has to do. ( install = npm install, custom = customized command )

\item {} 
Working folder that contains package.json — Where is package.json is located. ( you can locate to package.json by clicking …. in the right side )

\end{itemize}
\begin{itemize}
\item {} 
The commands we’ll use are Angular CLI 5 installation and the Angular packs (node modules)

\end{itemize}
\end{quote}

The next steps would be to include the Tasks with the functions we want to perform during the Pipeline execution. In our case: build and deploy tasks.
The build command is executed by calling the corresponding script that has been programmed in the client code for each environment. In this case, since the client is programmed in Angular, this is defined in the package.json file.

And this script will execute the “ng build” command configured for the project environment (in environment folder).
The task of deploy will be to save the contents of the dist folder that has been generated after the client build. This is the one that will be included later in the server to make the environment deploy.


\section{1.3.5. Keys and secrets for each environment.}
\label{\detokenize{pages/Environments/Build_deploy:keys-and-secrets-for-each-environment}}

\subsection{1.3.5.1. Health29 client.}
\label{\detokenize{pages/Environments/Build_deploy:health29-client}}

\subsubsection{1.3.5.1.1. Secret keys.}
\label{\detokenize{pages/Environments/Build_deploy:secret-keys}}
This project uses external services. For each of the environments it will be necessary to configure the value of the secret keys to connect with the different APIs.

As it is an Angular application, for the project to be compiled and executed it is necessary for the environments files to exist (at least environment.ts).

To be able to compile and execute this project you have to modify the extension of the src/environments files, removing “.sample” (for example, you have to modify environments.ts.sample by environments.ts) and here complete the information of the secret keys of the services.

As a minimum, it is mandatory to perform these actions on the environment.ts file (to work locally) in order to compile and run the platform. If you want to use any of the other environments, it is also essential that this file has been modified in addition to the one corresponding to the environment on which you want to work.


\subsubsection{1.3.5.1.2. Environment deployment \textendash{} Azure pipelines.}
\label{\detokenize{pages/Environments/Build_deploy:environment-deployment-azure-pipelines}}
For the continuous deployment we use the Azure Devops Pipelines, which must be configured to do this task automatically and to be able to execute the complication and the deploy of the application correctly.
For this purpose, we have uploaded the environment files with the values for Foundation29 to Azure Devops: Pipelines/Library, as secure files.

In the pipelines, prior to the server’s build, the following tasks have been added: \sphinxhref{https://docs.microsoft.com/en-us/azure/devops/pipelines/tasks/utility/download-secure-file?view=azure-devops}{Download secure file} and \sphinxhref{https://docs.microsoft.com/en-us/azure/devops/pipelines/tasks/utility/copy-files?view=azure-devops\&tabs=yaml}{Copy files}, in each case downloading and copying to the destination folder (src/environments) the files: environment.ts and the one corresponding to the environment that manages the pipeline (dev, staging or prod).

In the same way, the tasks to modify the README.md file to include the information about the pipeline execution state (badges) have been implemented in the pipelines (according to the environment they are stored as secure files). For this purpose, the “Download secure file” task has been used again. However, in this case the \sphinxhref{https://docs.microsoft.com/en-us/azure/devops/pipelines/tasks/utility/powershell?view=azure-devops}{Powershell} \sphinxhref{https://docs.microsoft.com/en-us/powershell/module/microsoft.powershell.management/move-item?view=powershell-7}{“Move\sphinxhyphen{}Item”} command has been used because besides copying the file you also want to rename it (in each branch or environment there has to be a file with the name README.md). Finally, the git actions are executed on the \sphinxhref{https://docs.microsoft.com/en-us/azure/devops/pipelines/tasks/utility/command-line?view=azure-devops\&tabs=yaml}{command line} to update the repository with these changes.

NOTE: This is done in the pipeline, otherwise every time you change the environment (from dev to test or from test to prod) you would have to manually update the README.md file because it would be crushed with the previous environment.


\subsection{1.3.5.2. Health29 server.}
\label{\detokenize{pages/Environments/Build_deploy:health29-server}}

\subsubsection{1.3.5.2.1. Secret keys.}
\label{\detokenize{pages/Environments/Build_deploy:id1}}
This project uses external services. In this case it is a project in node.js, where the file containing the information of the secret keys to connect to the different APIs is config.js. As with the client, we have chosen to create a config.js.sample file with the indications to complete the necessary information to compile and execute the project.

As a minimun, for local develop it is mandatory to configure the file config.js.sample. So, to be able to compile and execute this project you have to modify the extension of the config.js.sample file, removing “.sample” (you have to modify config.js.sample by config.js) and here complete the information of the secret keys of the services. If you want to deploy on each environment you must configure in the Azure App Service slot in Configuration/Aplication settings the variables defined in config.js file.


\subsubsection{1.3.5.2.2. Environment deployment \textendash{} Azure pipelines.}
\label{\detokenize{pages/Environments/Build_deploy:id2}}
For the continuous deployment we use the Azure Devops Pipelines, which must be configured to do this task automatically and to be able to execute the complication and the deploy of the application correctly.
For this purpose, we have uploaded the config.js file with the values for Foundation29 to Azure Devops: Pipelines/Library, as secure file.

In the pipelines, prior to the server’s build, the following tasks have been added: \sphinxhref{https://docs.microsoft.com/en-us/azure/devops/pipelines/tasks/utility/download-secure-file?view=azure-devops}{Download secure file} and \sphinxhref{https://docs.microsoft.com/en-us/azure/devops/pipelines/tasks/utility/copy-files?view=azure-devops\&tabs=yaml}{Copy files}.

In the same way, the tasks to modify the README.md file to include the information about the pipeline execution state (badges) have been implemented in the pipelines (according to the environment they are stored as secure files). For this purpose, the “Download secure file” task has been used again. However, in this case the \sphinxhref{https://docs.microsoft.com/en-us/azure/devops/pipelines/tasks/utility/powershell?view=azure-devops}{Powershell} \sphinxhref{https://docs.microsoft.com/en-us/powershell/module/microsoft.powershell.management/move-item?view=powershell-7}{“Move\sphinxhyphen{}Item”} command has been used because besides copying the file you also want to rename it (in each branch or environment there has to be a file with the name README.md). Finally, the git actions are executed on the \sphinxhref{https://docs.microsoft.com/en-us/azure/devops/pipelines/tasks/utility/command-line?view=azure-devops\&tabs=yaml}{command line} to update the repository with these changes.




\chapter{2. Software architecture}
\label{\detokenize{pages/Software architecture:software-architecture}}\label{\detokenize{pages/Software architecture::doc}}
We will use the C4 model to expose the software architecture. Thus, in the description of this section we will differentiate between:
\begin{itemize}
\item {} 
Level 1: A \sphinxstylestrong{System Context} diagram provides a starting point, showing how the software system in scope fits into the world around it.

\item {} 
Level 2: A \sphinxstylestrong{Container} diagram zooms into the software system in scope, showing the high\sphinxhyphen{}level technical building blocks.

\item {} 
Level 3: A \sphinxstylestrong{Component} diagram zooms into an individual container, showing the components inside it.

\item {} 
Level 4: A \sphinxstylestrong{code} diagram with the technical explanation about the programming of each one of the modules or components.

\end{itemize}






\chapter{2.1. Level 1: System Context}
\label{\detokenize{pages/SW/SystemContext:level-1-system-context}}\label{\detokenize{pages/SW/SystemContext::doc}}
There are four user profiles on this platform, so each one will have different associated functionalities: user, administrator, superadministrator and clinical.



For each group of patients we will have these three roles:
\begin{itemize}
\item {} 
\sphinxstylestrong{User}.  A user will be created for each patient in the group. For each user a series of functionalities will be provided:

\end{itemize}
\begin{itemize}
\item {} 
Using the navbar, the user will navigate to different pages where he or she can manage the information of the user profile (name, language, weight and length units), consult the notifications or obtain help and contact the platform’s support team.

\item {} 
From the menu, the user will be able to access different pages to complete his or her personal and medical information, consult the FAQs or the privacy policies, or contact support.

\end{itemize}
\begin{itemize}
\item {} 
\sphinxstylestrong{Administrator} (of each patient group). The administrator will have the same functionalities in the navbar, except for notifications.Using the navigation through the menu of this role, the administrator will be able to request new language or translations, manage the FAQs, and obtain information from the patients (statistics) or send them notifications/alerts.

\item {} 
\sphinxstylestrong{Super Administrator}. This is a more technical profile. He or she can add languages to the platform, manage translations, and manage the different groups of patients (Add symptoms, FAQs, datapoints, medicines).

\end{itemize}

Furthermore, we also have the profile of the \sphinxstylestrong{researcher}. They will be able to consuming data through the API.

In general, this platform will be able to store patient data and have it managed by the administrators of the group to which it belongs. The latter will be able, on the one hand, to obtain statistical information about the interaction of the users with the platform, and on the other hand, to interact with the patients through notifications or alerts.



The functionalities offered by the Health29 platform can be encompassed, according to the image above, as follows:
\begin{itemize}
\item {} 
Capacities of the platform: It is a multilingual platform designed to be easily adaptable to the needs or characteristics of any group of patients.

\item {} 
Data: it allows to store and to extract the personal information and of sanitary character of the patients. It also provides statistics on patient information to group administrators.

\item {} 
Communication. The platform includes several methods to establish communication between the different roles. It allows the sending of emails so that the administrators or patients can contact the technical support team, and the administrators can send notifications to the users to send them information they consider of interest.

\item {} 
It includes several services that allow to provide other functionalities to the application, such as a FAQ page about the specific disease of the patient group or a wizard to guide users through the platform.

\end{itemize}




\chapter{2.2. Level 2: Containers}
\label{\detokenize{pages/SW/Containers:level-2-containers}}\label{\detokenize{pages/SW/Containers::doc}}
Health29 consists of several connected containers.
We have divided them according to their functionality and use within the platform.Thus, we will have:
\begin{itemize}
\item {} 
Webapp container.

\item {} 
External APIs

\item {} 
Azure cognitive services container.

\item {} 
Azure healthbot container.

\item {} 
Azure blobs container.

\item {} 
Other services container.

\item {} 
Databases container.

\end{itemize}



These modules are intercommunicated using a \sphinxhref{https://restfulapi.net/}{REST} interface, that is, the communication is established according to the \sphinxhref{https://restfulapi.net/http-methods/}{HTTP protocol}.

In this way, it will be necessary to configure the different services you want to use so that communications can be established. To do this, during the implementation of the webapp the keys and the corresponding endpoint will be used to establish communication, and the sending of the REST commands will use specific authorization headers.

The most common headers we can find are:
\begin{itemize}
\item {} 
Ocp\sphinxhyphen{}Apim\sphinxhyphen{}Subscription\sphinxhyphen{}Key. Use with Cognitive Services subscription if you pass your secret key.

\item {} 
Authorization. Use with your Cognitive Services subscription if you pass an authentication token. The value is the bearer token: Bearer token\_value

\end{itemize}

The webapp will be the core of the Health29 application, from where the frontend will be developed and the communications with different services to provide functionalities.

Health29 is a clinical history application that will allow users to enter their medical data. Different Azure services will be used to carry out the relevant actions:
\begin{itemize}
\item {} 
External APIs that will be used as intermediaries between the webapp and Azure’s cognitive services. To add different functionalities to the application and thus simplify the development and implementation of the webapp.

\item {} 
Healthbot. The application will have a chatbot to help the user.

\item {} 
Azure blobs and databases. To store information.

\item {} 
Other services that add additional functionalities. (\sphinxstyleemphasis{NOTE: In this section we will add services from Azure})

\end{itemize}




\chapter{2.3. Level 3: Component}
\label{\detokenize{pages/SW/Components:level-3-component}}\label{\detokenize{pages/SW/Components::doc}}
Taking into account the division by containers exposed in the previous point, we are now going to study in detail the internal architecture and the components that make up each one of them.


\section{2.3.1. Webapp}
\label{\detokenize{pages/SW/Components:webapp}}
Health29’s architecture uses a client\sphinxhyphen{}server software design model, so that the architecture of the webapp is like:



For the client we use the Angular 5 framework, and for the nodejs \sphinxhyphen{} express server, in which we have implemented an API to connect to the CosmoDb databases using mongoose for its management.



These modules are intercommunicated using a \sphinxhref{https://restfulapi.net/}{REST} interface, that is, the communication is established according to the \sphinxhref{https://restfulapi.net/http-methods/}{HTTP protocol}.

This communication between client and server is mainly used for data management, i.e. to obtain, add or modify information from databases.
For this purpose, an HTTP service will be used to allow, through requests from the client to the server, to carry out these operations. The server will listen to these requests and will perform the appropriate operations to give an answer to the client.

The Health29 platform has been developed as a Platform as a Service (PaaS) in Azure. That is, an App Service has been created that contains the client\sphinxhyphen{}server webapp. In particular, the App Service created is \sphinxhref{https://portal.azure.com/\#@foundation29outlook.onmicrosoft.com/resource/subscriptions/53348303-e009-4241-9ac7-a8e4465ece27/resourceGroups/health29/providers/Microsoft.Web/sites/health29/appServices}{health29}


\section{2.3.2. External APIs}
\label{\detokenize{pages/SW/Components:external-apis}}
\sphinxstylestrong{\sphinxhref{https://f29api.northeurope.cloudapp.azure.com/index.html}{Foundation29 API}}, implemented by Foundation29 to use it as an intermediary between the webapp and the azure qnamaker service. It is used for QNA functions for the different roles of Health29 platform.



\sphinxstylestrong{The \sphinxhref{https://api.monarchinitiative.org/}{Monarch} API}, is an external API that is used to obtain symptom information.




\section{2.3.3. Azure cognitive services}
\label{\detokenize{pages/SW/Components:azure-cognitive-services}}
As explained in the previous section of this document (2.2.Containers) several “Azure cognitive services” will be used.

In particular, services will be used for the conversion of formats, for the translation of texts and for the creation of databases for the FAQs of each group of patients.

All of them will communicate with webapp.

In the following subsections each of them is introduced.


\subsection{2.3.3.1. Qna maker}
\label{\detokenize{pages/SW/Components:qna-maker}}
QnA Maker is a cloud\sphinxhyphen{}based Natural Language Processing (NLP) service that easily creates a natural conversation layer with the data.

In Health29 it is used to manage the FAQs in different ways and from different points of the platform:
\begin{itemize}
\item {} 
To show the list of FAQs to the \sphinxstylestrong{users}. From the user profile you can access the FAQ page where you will be shown the results of the service consultation in the form of a list.

\item {} 
So that the \sphinxstylestrong{administrators} of the platform can manage the list of FAQs. From the administrator profile you can add, delete and edit the list of FAQs of the service.

\item {} 
It is integrated into the \sphinxstylestrong{healthbot} to allow users to ask questions or doubts in a more guided and personal way.

\end{itemize}

In this case the communication of the webapp with this service is requested from the client through an external API that acts as an intermediary. However, there will also be information that will be stored in the databases, therefore, the client will also establish a communication with the server who will manage the storage of this information.

You can configure and use this azure service by following the steps in the \sphinxhref{https://docs.microsoft.com/en-us/azure/cognitive-services/qnamaker/}{Microsoft guide}.

In particular for Health29 the following databases have been created in qnamaker: \sphinxhref{https://www.qnamaker.ai/Home/MyServices}{My knowledge bases}. One would be created for each group of patients to contain their specific information and this in turn would be replicated in as many languages as the question\sphinxhyphen{}answer pairs are translated on the Health29 platform.


\subsection{2.3.3.2. Translator}
\label{\detokenize{pages/SW/Components:translator}}
This service is used to be able to make a translation of the different datapoints, to add a new language to the platform (it translates all the tags), and to translate into English the text obtained from the vision service to call the NCR service that extracts the symptoms.

You can configure and use this azure service by following the steps in the \sphinxhref{https://docs.microsoft.com/bs-cyrl-ba/azure/cognitive-services/translator/}{Microsoft guide}.


\section{2.3.4. Azure healthbot}
\label{\detokenize{pages/SW/Components:azure-healthbot}}
Health29 has an assistant to guide the user in the use of the platform.



For this purpose, the Azure \sphinxhref{https://docs.microsoft.com/en-us/healthbot/}{Healthbot} service is used.

It guides the user through different configured scenarios where he or she can perform different actions. Initially we can divide the scenarios into three blocks:
\begin{itemize}
\item {} 
\sphinxstylestrong{New user} scenario or first time in the platform. The chatbot will provide the user with help and guidance in this process and provide he or she with more information if required.

\item {} 
Scenario of \sphinxstylestrong{pending notifications}. In case the user has any pending notifications, he or she will be informed during the first run of the assistant.

\item {} 
\sphinxstylestrong{Main} scenario. A series of guided scenarios will appear where the user will be able to make different types of queries.

\end{itemize}

You can configure and use this azure service by following the steps in the \sphinxhref{https://docs.microsoft.com/en-us/healthbot/quickstart-createyourhealthcarebot}{Microsoft guide}.


\section{2.3.5. Azure blobs}
\label{\detokenize{pages/SW/Components:azure-blobs}}
Azure Blob storage is Microsoft’s object storage solution for the cloud. Blob storage is optimized for storing massive amounts of unstructured data. Unstructured data is data that doesn’t adhere to a particular data model or definition, such as text or binary data.

One “Storage accounts (classic)” container called “blobgenomics” has been created to store information from various sections of Health29:
\begin{itemize}
\item {} 
Medical care section. Here, one container per patient is created to store their medical information.

\item {} 
Diagnosis section. As in the previous point, one container per patient is created to store the diagnostic information.

\item {} 
Genotype section. One container per patient is created to store the genotype data of the patient.

\item {} 
Phenotype section. Same as above but for saving the phenotype data.

\item {} 
Support section. To manage the data of this section.

\end{itemize}

You can configure and use this azure service by following the steps in the \sphinxhref{https://docs.microsoft.com/es-es/azure/storage/blobs/storage-blobs-introduction}{Microsoft guide}.


\section{2.3.7. Databases}
\label{\detokenize{pages/SW/Components:databases}}
We have several separate collections in two databases, one for accounts and general things, and one for patient data.

The development and test environments share the same databases, while the production ones do not. So, in total we have 4 databases:
\begin{itemize}
\item {} 
Two for accounts

\end{itemize}
\begin{itemize}
\item {} 
Development and testing

\item {} 
Production

\end{itemize}
\begin{itemize}
\item {} 
Two for patient data

\end{itemize}
\begin{itemize}
\item {} 
Development and testing

\item {} 
Production

\end{itemize}



Access from the Health29 platform is done from the application server,using \sphinxhref{https://mongoosejs.com/docs/}{mongoose}.




\chapter{2.4. Level 4: Code}
\label{\detokenize{pages/SW/Code:level-4-code}}\label{\detokenize{pages/SW/Code::doc}}

\section{2.4.1. Webapp}
\label{\detokenize{pages/SW/Code:webapp}}
All documentation for the Health29 application code is contained in:
\begin{itemize}
\item {} 
For the \sphinxhref{https://health29-dev.azurewebsites.net/APIDOC/}{development environment}

\item {} 
For the \sphinxhref{https://health29-test.azurewebsites.net/APIDOC}{test environment}

\item {} 
For the \sphinxhref{https://health29.org/APIDOC/}{production environment}

\end{itemize}

For Android and iOS apps, we use the cordova framework. When compiling the Angular project, it generates the dist folder, which is the one to be added to the www folder of the Corova project.  We were evaluating ionic, nativescript and reactscript, they would be good options if we were more people, but at the moment we fit more cordova.

Our idea is to migrate to FHIR, so we would have to modify our API to make calls to \sphinxhref{https://www.hl7.org/fhir/}{FHIR’s API}

Health29 provides a web API to access the data. Anyone can develop an application to access and modify the data of a Health29 user. OAuth 2.0 is used as an authorization protocol to give an API client limited access to the user’s data.


\subsection{2.4.1.1. Azure App service}
\label{\detokenize{pages/SW/Code:azure-app-service}}
A \sphinxstylestrong{\sphinxhref{https://docs.microsoft.com/en-US/azure/app-service/}{WebApp App Service}} has been created in Azure. The steps to follow would be:
\begin{enumerate}
\sphinxsetlistlabels{\arabic}{enumi}{enumii}{}{.}%
\item {} 
Login to Azure’s portal

\item {} 
Select App services and click on add.

\item {} 
Select Webapp and configure: Application Name, Subscription Type, Resource Group, OS Type, Publish Type, App Service Plan/Location and Application Insights

\end{enumerate}

Then, it has been taken into account that several environments have been created to work on the platform, that is, there will be different slots and the deployment of the code will be done in the corresponding one. All this has been explained in section 1 of this document.


\subsection{2.4.1.2. Modules communication}
\label{\detokenize{pages/SW/Code:modules-communication}}
For establish the communication:
\begin{itemize}
\item {} 
The client will make requests of the type:

\end{itemize}

\begin{sphinxVerbatim}[commandchars=\\\{\}]
\PYG{n}{this}\PYG{o}{.}\PYG{n}{http}\PYG{o}{.}\PYG{n}{get}\PYG{p}{(}\PYG{n}{environment}\PYG{o}{.}\PYG{n}{api}\PYG{o}{+}\PYG{l+s+s1}{\PYGZsq{}}\PYG{l+s+s1}{/api/}\PYG{l+s+s1}{\PYGZsq{}}\PYG{o}{+}\PYG{o}{\PYGZlt{}}\PYG{n}{url}\PYG{o}{\PYGZgt{}}\PYG{p}{)}
        \PYG{o}{.}\PYG{n}{map}\PYG{p}{(} \PYG{p}{(}\PYG{n}{res} \PYG{p}{:} \PYG{n+nb}{any}\PYG{p}{)} \PYG{o}{=}\PYG{o}{\PYGZgt{}} \PYG{p}{\PYGZob{}}
            \PYG{o}{/}\PYG{o}{/} \PYG{n}{do} \PYG{n}{something} \PYG{n}{when} \PYG{n}{res} \PYG{n}{OK}
         \PYG{p}{\PYGZcb{}}\PYG{p}{,} \PYG{p}{(}\PYG{n}{err}\PYG{p}{)} \PYG{o}{=}\PYG{o}{\PYGZgt{}} \PYG{p}{\PYGZob{}}
           \PYG{o}{/}\PYG{o}{/} \PYG{n}{do} \PYG{n}{something} \PYG{n}{when} \PYG{n}{error}
         \PYG{p}{\PYGZcb{}}\PYG{p}{)}
\end{sphinxVerbatim}
\begin{itemize}
\item {} 
The server will listen to these requests and will perform the appropriate operations to give an answer to the client.

\end{itemize}

\begin{sphinxVerbatim}[commandchars=\\\{\}]
\PYG{n}{api}\PYG{o}{.}\PYG{n}{get}\PYG{p}{(}\PYG{o}{\PYGZlt{}}\PYG{n}{url}\PYG{o}{\PYGZgt{}}\PYG{p}{,} \PYG{n}{auth}\PYG{p}{,} \PYG{n}{function}\PYG{p}{)}
\end{sphinxVerbatim}

This communication between components is done in a secure way, that is, several tools or techniques have been implemented so that the information is encrypted and the communication is reliable.
For this purpose, crypto libraries (aes\sphinxhyphen{}256\sphinxhyphen{}ecb) are used for data encryption and decryption, and jwt\sphinxhyphen{}simple for encoding the authentication token for API calls.
\begin{itemize}
\item {} 
Encryption: When a user tries to create an account, the pw is sha512 before it is sent to the server. On the server a jump is created using the bcrypt library and a hash is applied to it before it is stored. This password will never be returned as described in the database model (password: \{ type: String, select: false,…). On the other hand, in the login process, it is the data model itself that checks that the pw is valid (bcrypt.compare). If it exceeds a maximum of 5 attempts, the account is blocked for 2 hours.

\item {} 
Authentication token. All Methods APIs that have the authorization field in the header use Bearer authentication to restrict access to protected resources, , and always be sent next to a token. The bearer token is a cryptic string, generated by the server in response to a login request. Example of the header: Authorization: Bearer (token).These requests can return some errors, such as the token is invalid, or has expired: \{ status: 401, message: “Token expired”\} or \{ status: 401, message: “Invalid Token”\}

\item {} 
Authorization: All Methods APIs that have the authorization, it is verified that it has the expected role.

\end{itemize}

Health29 provides a Web API for accessing data. Anyone can develop an application to access and modify a Health29 user’s data. OAuth 2.0 (Implicit Grant) is used as an authorization protocol to give an API client limited access to user data.


\subsection{2.4.1.3. Code Structure: Client Structure}
\label{\detokenize{pages/SW/Code:code-structure-client-structure}}
\sphinxstyleemphasis{NOTE: A template was purchased to have a base: \sphinxhref{https://themeforest.net/item/apex-angular-4-bootstrap-admin-template/20774875}{Template Link}.}

The structure is as follows:



The src folder has the following:
\begin{itemize}
\item {} 
The \sphinxstylestrong{app folder} is the one with all the code.
\begin{itemize}
\item {} 
app/app.component.\{ts,html,css,spec.ts\}: Defines the AppComponent along with an HTML template, CSS stylesheet, and a unit test. It is the root component of what will become a tree of nested components as the application evolves. In this file it is controlling the events of inactivity of a session, loading the language of the app depending on the language of the browser, the title that appears in the browser tab with the change of pages. If it is a mobile app, it also controls the backbutton, pause and resume events.

\item {} 
app/app.module.ts: Defines AppModule, the root module that tells Angular how to assemble the application.

\end{itemize}

\item {} 
The \sphinxstylestrong{assets folder}: It contains all the information that will be accessible from any url. Css files, js, images, language files, jsons listing countries, types of subscriptions, frequently asked questions for each group of patients, etc. It will be the only visible folder when a build is made for production.

\item {} 
The \sphinxstylestrong{environments folder}. This folder contains one file for each of your destination environments, each exporting simple configuration variables to use in your application. The files are replaced on\sphinxhyphen{}the\sphinxhyphen{}fly when you build your app. You might use a different API endpoint for development than you do for production or maybe different analytics tokens. You might even use some mock services. Either way, the CLI has you covered.

\end{itemize}

\begin{sphinxVerbatim}[commandchars=\\\{\}]
\PYG{n}{export} \PYG{n}{const} \PYG{n}{environment} \PYG{o}{=} \PYG{p}{\PYGZob{}}
  \PYG{n}{production}\PYG{p}{:} \PYG{n}{true}\PYG{p}{,}
  \PYG{n}{api}\PYG{p}{:} \PYG{n}{undefined}\PYG{p}{,} \PYG{o}{/}\PYG{o}{/}\PYG{n}{Server} \PYG{n}{URL}\PYG{p}{:} \PYG{l+s+s1}{\PYGZsq{}}\PYG{l+s+s1}{http://\PYGZlt{}direction\PYGZgt{}:\PYGZlt{}port\PYGZgt{}}\PYG{l+s+s1}{\PYGZsq{}}
  \PYG{n}{blobAccessToken}\PYG{p}{:}\PYG{n}{undefined}\PYG{p}{,} \PYG{o}{/}\PYG{o}{/} \PYG{n}{Blob} \PYG{n}{access} \PYG{n}{information}\PYG{p}{:} \PYG{p}{\PYGZob{}}\PYG{n}{sasToken}\PYG{p}{:}\PYG{n}{null}\PYG{p}{,}\PYG{n}{blobAccountUrl}\PYG{p}{:} \PYG{l+s+s1}{\PYGZsq{}}\PYG{l+s+s1}{https://\PYGZlt{}blob\PYGZus{}name\PYGZgt{}.blob.core.windows.net}\PYG{l+s+s1}{\PYGZsq{}}\PYG{p}{\PYGZcb{}}
  \PYG{n}{keyCognitiveMicrosoft}\PYG{p}{:}\PYG{n}{undefined}\PYG{p}{,} \PYG{o}{/}\PYG{o}{/} \PYG{n}{key} \PYG{n}{Microsoft} \PYG{n}{Cognitive} \PYG{n}{Services}
  \PYG{n}{keyF29api}\PYG{p}{:}\PYG{n}{undefined}\PYG{p}{,} \PYG{o}{/}\PYG{o}{/} \PYG{n}{f29Bio} \PYG{n}{sevice} \PYG{n}{URL}
  \PYG{n}{f29api}\PYG{p}{:} \PYG{l+s+s1}{\PYGZsq{}}\PYG{l+s+s1}{undefined}\PYG{l+s+s1}{\PYGZsq{}}\PYG{p}{,} \PYG{o}{/}\PYG{o}{/} \PYG{n}{f29api} \PYG{n}{service} \PYG{n}{URL}
  \PYG{n}{healthbot}\PYG{p}{:} \PYG{l+s+s1}{\PYGZsq{}}\PYG{l+s+s1}{undefined}\PYG{l+s+s1}{\PYGZsq{}} \PYG{o}{/}\PYG{o}{/} \PYG{n}{Azure} \PYG{n}{healthbot} \PYG{n}{service}
\PYG{p}{\PYGZcb{}}\PYG{p}{;}
\end{sphinxVerbatim}
\begin{itemize}
\item {} 
\sphinxstylestrong{Some files}:

\end{itemize}
\begin{itemize}
\item {} 
favicon.ico: Every site wants to look good on the bookmark bar. Get started with your very own Angular icon

\item {} 
index.html: The main HTML page that is served when someone visits your site. Most of the time you’ll never need to edit it. The CLI automatically adds all js and css files when building your app so you never need to add any script or link tags here manually. For the mobile version, two small changes must be made in this file:
Change base: base href=”./”
Add cordova:  “script src=”cordova.js”

\item {} 
main.ts: The main entry point for your app. Compiles the application with the JIT compiler and bootstraps the application’s root module (AppModule) to run in the browser. You can also use the AOT compiler without changing any code by passing in \textendash{}aot to ng build or ng serve

\item {} 
polyfills.ts: Different browsers have different levels of support of the web standards. Polyfills help normalize those differences. You should be pretty safe with core\sphinxhyphen{}js and zone.js, but be sure to check out the Browser Support guide for more information.

\item {} 
styles.css: Your global styles go here. Most of the time you’ll want to have local styles in your components for easier maintenance, but styles that affect all of your app need to be in a central place.

\item {} 
test.ts: This is the main entry point for your unit tests. It has some custom configuration that might be unfamiliar, but it’s not something you’ll need to edit.

\item {} 
tsconfig.\{app|spec\}.json: TypeScript compiler configuration for the Angular app (tsconfig.app.json) and for the unit tests (tsconfig.spec.json).

\end{itemize}

The \sphinxstylestrong{app folder} is the one with all the code:


\begin{itemize}
\item {} 
\sphinxstylestrong{Layouts}: the different layouts that there are, at the moment two subfolders, for the logged ones (full) and for the ones that are not (content).

\item {} 
\sphinxstylestrong{Pages}: It has two subfolders, content\sphinxhyphen{}pages (corresponds to the logged out pages like login page, registration, etc) and full\sphinxhyphen{}pages (common pages for all logged out roles, like the user\sphinxhyphen{}profile page).

\item {} 
\sphinxstylestrong{Role pages}

\end{itemize}
\begin{itemize}
\item {} 
\sphinxstyleemphasis{admin}: contains all pages for the admin role

\item {} 
\sphinxstyleemphasis{superadmin}: contains all pages for the superadmin role

\item {} 
\sphinxstyleemphasis{user}: contains all the pages for the user role.
Each folder of these roles has a module file (it loads the needed modules), and a route file to manage the routes and control the authentication and authorization (auth\sphinxhyphen{}guard and role\sphinxhyphen{}guard)

\end{itemize}
\begin{itemize}
\item {} 
The \sphinxstylestrong{shared folder}, which is the shared code:

\end{itemize}
\begin{itemize}
\item {} 
auth: The Auth folder includes the services that will handle the platform’s authentication and authorization tasks. Thus, this folder will contain: authorization management (role\sphinxhyphen{}guard), authentication (auth\sphinxhyphen{}service and auth\sphinxhyphen{}guard), control of headers in the application’s http communications
(http interceptor),page routing control(canDeactivate) and oauth.services for external services like fitbit.

\item {} 
Configs: configuration files, for example configuration for toasts, or parameters for graphs.

\item {} 
Customizer: This is the customizer or chatbot component of the platform. At the moment it’s only implemented for the user profile, but it’s exposed as a shared module in case you want to include this functionality in some other profile in the future.

\item {} 
Directives: correspond to the Angular directives

\item {} 
Footer: It’s the footer component of the platform.

\item {} 
Models: Here you can add the data models you want to use on the platform that are common to all profiles. At the moment only the FAQ template is added.

\item {} 
Navbar and navbar\sphinxhyphen{}nolog: It corresponds to the top bar. Navbar\sphinxhyphen{}nolog is for when you are not logged in.

\item {} 
Routes: route management. It has two files, one for the paths of the unlogged pages, and another for the rest.

\item {} 
Services: corresponds to the Angular services. Several have been developed as faq.service or lang.service

\item {} 
Sidebar: side menu, depending on the role, loads one menu or another.

\end{itemize}

\sphinxstylestrong{Routes}: In the root of the app folder, there is a file called app\sphinxhyphen{}routing.module.ts, which is the root of the routes management. Depending on the path, it loads some subroutes and others. For example, if it is unlogged, it loads the routes ./shared/routes/content\sphinxhyphen{}layout.routes, on the other hand, if it is logged in it loads ./shared/routes/full\sphinxhyphen{}layout.routes. In this second case, check that it is authenticated (canActivate: {[}AuthGuard{]})
The rest of the subroutes that come from full\sphinxhyphen{}layout, are controlled if they are logged and authenticated in the routing\sphinxhyphen{}module file of each module (each profile has a module) with canActivate: {[}AuthGuard, RoleGuard{]}.


\subsection{2.4.1.4. Code Structure: Server Structure}
\label{\detokenize{pages/SW/Code:code-structure-server-structure}}

\begin{itemize}
\item {} 
The \sphinxstylestrong{dist folder} contains the compilation of the client code.

\item {} 
In the \sphinxstylestrong{models folder}, the different templates for working with the platform’s database collections are defined.

\item {} 
The \sphinxstylestrong{controllers folder} contains the functionality to work with the previous collections. It is organized according to the actions that each role of the health29 platform can perform.

\item {} 
In the \sphinxstylestrong{routes folder}, all the routes of the api appear. It contains the file index.js that links the models with the controllers, defining the requests that will be available to the Health29 client. Those that have auth means that they need authentication to be used, and authorization.

\item {} 
\sphinxstylestrong{Middlewares} has the file auth.js which is in charge of checking if the token is valid, so that you know if you have permission to make the request to the API call. It also has the roles.js file, where the different role groups are defined.

\item {} 
In \sphinxstylestrong{views} we have the views generated by the server and the templates for the emails.

\item {} 
Analogous to the client structure we have the \sphinxstylestrong{services folder}. Here, we have the following services: Auth.js creates and decodes token, crypt.js encrypts and decrypts data and email.js is the emailing service. There are also other services for exomizer and phenolizer functions.

\item {} 
\sphinxstylestrong{Some files}:

\end{itemize}
\begin{itemize}
\item {} 
index.js: file where the app.js and config.js file is loaded It listens to requests.

\item {} 
Config.js: configuration file. The server url, port, connection with the data bases are passed to it through environment variables. The credentials of the mail that send email, the secret token of jwt, and the secret of crypto are not passed through environment variables.

\item {} 
Db\_connect.js: will be in charge of creating the connection with the databases.

\item {} 
App.js: the crossdomain is established, and other node configurations. It manages the requests, and these can be of 3 types:
\textgreater{}1. API routes managed by routes folder.
\textgreater{}2. Server\sphinxhyphen{}generated view paths (handlebars)
\textgreater{}3. the paths of the Angular client application (dist folder)

\end{itemize}


\subsection{2.4.1.4. External libraries and dependencies}
\label{\detokenize{pages/SW/Code:external-libraries-and-dependencies}}
All the necessary dependencies or libraries that will be used in the implementation of the application can be checked in the “package.json” files of both the client and the server. The installation of all these libraries and dependencies has been done with \sphinxhref{https://docs.npmjs.com/using-npm-packages-in-your-projects}{npm}, which updates the “Node modules” folder with the necessary packages to work with. Therefore, when you want to work with the projects, it is necessary to execute previously the command “npm install”.

On the one hand, among all those found in this file, we can mainly highlight that the versions are being used:
\begin{itemize}
\item {} 
Both client and server:

\end{itemize}
\begin{itemize}
\item {} 
v10.16.3 of \sphinxhref{https://www.npmjs.com/package/node}{node}. License: \sphinxhref{https://opensource.org/licenses/ISC}{ISC}

\item {} 
0.11.4 of \sphinxhref{https://www.npmjs.com/package/botframework-webchat}{botframework\sphinxhyphen{}webchat}. License: \sphinxhref{https://opensource.org/licenses/MIT}{MIT}

\end{itemize}
\begin{itemize}
\item {} 
For the client:

\end{itemize}
\begin{itemize}
\item {} 
1.7.0 of Angular CLI and 5.2.5 of the node modules Angular libraries (Angular 5). \sphinxhref{https://www.npmjs.com/package/@angular/cli}{Angular}. License: \sphinxhref{https://opensource.org/licenses/MIT}{MIT}

\item {} 
2.6.2 of \sphinxhref{https://www.npmjs.com/package/typescript}{TypeScript}. License: \sphinxhref{https://opensource.org/licenses/Apache-2.0}{Apache\sphinxhyphen{}2.0}.

\item {} 
1.0.0 of \sphinxhref{https://www.npmjs.com/package/@ng-bootstrap/ng-bootstrap}{ng\sphinxhyphen{}bootstrap}. License: \sphinxhref{https://opensource.org/licenses/MIT}{MIT}

\item {} 
4.0.0 of \sphinxhref{https://www.npmjs.com/package/bootstrap}{bootstrap}. License: \sphinxhref{https://opensource.org/licenses/MIT}{MIT}

\item {} 
3.2.1 of \sphinxhref{https://www.npmjs.com/package/jquery}{Jquery}. License: \sphinxhref{https://opensource.org/licenses/MIT}{MIT}

\item {} 
9.0.2 of \sphinxhref{https://www.npmjs.com/package/@ngx-translate/core}{ngx\sphinxhyphen{}translate/core}. License: \sphinxhref{https://opensource.org/licenses/MIT}{MIT}

\item {} 
2.0.1 of \sphinxhref{https://www.npmjs.com/package/@ngx-translate/http-loader}{ngx\sphinxhyphen{}translate/http\sphinxhyphen{}loader}. License: \sphinxhref{https://opensource.org/licenses/MIT}{MIT}

\item {} 
11.1.7 of \sphinxhref{https://www.npmjs.com/package/@swimlane/ngx-datatable}{ngx\sphinxhyphen{}datatable}. License: \sphinxhref{https://opensource.org/licenses/MIT}{MIT}

\item {} 
5.1.0 of \sphinxhref{https://www.npmjs.com/package/@ngrx/store}{ngrx/store}. License: \sphinxhref{https://opensource.org/licenses/MIT}{MIT}

\item {} 
4.3.4 of \sphinxhref{https://www.npmjs.com/package/bourbon}{bourbon}. License: \sphinxhref{https://opensource.org/licenses/MIT}{MIT}

\item {} 
4.6.3 of \sphinxhref{https://www.npmjs.com/package/angulartics2}{angulartics2}. License: \sphinxhref{https://opensource.org/licenses/MIT}{MIT}

\item {} 
2.5.3 of \sphinxhref{https://www.npmjs.com/package/core-js}{core\sphinxhyphen{}js}. License: \sphinxhref{https://opensource.org/licenses/MIT}{MIT}

\item {} 
1.5.3 of \sphinxhref{https://www.npmjs.com/package/jspdf}{jspdf}. License: \sphinxhref{https://opensource.org/licenses/MIT}{MIT}

\item {} 
0.8.0 of \sphinxhref{https://www.npmjs.com/package/marked}{marked}. License: \sphinxhref{https://opensource.org/licenses/MIT}{MIT}

\item {} 
5.5.6 of \sphinxhref{https://www.npmjs.com/package/rxjs}{rxjs}. License: \sphinxhref{https://opensource.org/licenses/Apache-2.0}{Apache\sphinxhyphen{}2.0}

\item {} 
3.3.1 of \sphinxhref{https://www.npmjs.com/package/screenfull}{screenfull}. License: \sphinxhref{https://opensource.org/licenses/MIT}{MIT}

\item {} 
0.4.19 of \sphinxhref{https://www.npmjs.com/package/xml2js}{xml2js}. License: \sphinxhref{https://opensource.org/licenses/MIT}{MIT}

\item {} 
1.2.2 of \sphinxhref{https://www.npmjs.com/package/ng2-smart-table}{ng2\sphinxhyphen{}smart\sphinxhyphen{}table}. License: \sphinxhref{https://opensource.org/licenses/MIT}{MIT}

\item {} 
4.1.2 of \sphinxhref{https://www.npmjs.com/package/ng2-toastr}{ng2\sphinxhyphen{}toastr}. License: \sphinxhref{https://opensource.org/licenses/ISC}{ISC}

\item {} 
1.4.9 of \sphinxhref{https://www.npmjs.com/package/ngx-ui-switch}{ngx\sphinxhyphen{}ui\sphinxhyphen{}switch}. License: \sphinxhref{https://opensource.org/licenses/MIT}{MIT}

\item {} 
4.2.0 of \sphinxhref{https://www.npmjs.com/package/ng2-validation}{ng2\sphinxhyphen{}validation}. License: \sphinxhref{https://opensource.org/licenses/MIT}{MIT}

\item {} 
0.8.18 of \sphinxhref{https://www.npmjs.com/package/zone.js?activeTab=readme}{zone.js}. License: \sphinxhref{https://opensource.org/licenses/MIT}{MIT}

\item {} 
2.2.5 of \sphinxhref{https://www.npmjs.com/package/web-animations-js}{web\sphinxhyphen{}animations\sphinxhyphen{}js}. License: \sphinxhref{https://opensource.org/licenses/Apache-2.0}{Apache\sphinxhyphen{}2.0}

\end{itemize}
\begin{itemize}
\item {} 
For the server:

\end{itemize}
\begin{itemize}
\item {} 
1.12.1 of \sphinxhref{https://www.npmjs.com/package/nodemon}{nodemon}. License: \sphinxhref{https://opensource.org/licenses/MIT}{MIT}

\item {} 
4.16.2 of \sphinxhref{https://www.npmjs.com/package/express}{express}. License: \sphinxhref{https://opensource.org/licenses/MIT}{MIT} and 3.0.0 of \sphinxhref{https://www.npmjs.com/package/express-handlebars}{express\sphinxhyphen{}handlebars}. License: \sphinxhref{https://opensource.org/licenses/BSD-3-Clause}{BSD\sphinxhyphen{}3\sphinxhyphen{}Clause}

\item {} 
4.13.1 \sphinxhref{https://www.npmjs.com/package/mongoose}{moongose}. License: \sphinxhref{https://opensource.org/licenses/MIT}{MIT}

\item {} 
2.6.0 of \sphinxhref{https://www.npmjs.com/package/async}{async}. License: \sphinxhref{https://opensource.org/licenses/MIT}{MIT}

\item {} 
0.0.3 of \sphinxhref{https://www.npmjs.com/package/bcrypt-nodejs}{bcrypt\sphinxhyphen{}nodejs}. License: \sphinxhref{https://opensource.org/licenses/MIT}{MIT}

\item {} 
1.18.2 of \sphinxhref{https://www.npmjs.com/package/body-parser}{body\sphinxhyphen{}parser}. License: \sphinxhref{https://opensource.org/licenses/MIT}{MIT}

\item {} 
5.0.0 of \sphinxhref{https://www.npmjs.com/package/fs-extra}{fs\sphinxhyphen{}extra}. License: \sphinxhref{https://opensource.org/licenses/MIT}{MIT}

\item {} 
0.5.1 of \sphinxhref{https://www.npmjs.com/package/jwt-simple}{jwt\sphinxhyphen{}simple}. License: \sphinxhref{https://opensource.org/licenses/MIT}{MIT}

\item {} 
2.19.1 of \sphinxhref{https://www.npmjs.com/package/moment}{moment}. License: \sphinxhref{https://opensource.org/licenses/MIT}{MIT}

\item {} 
2.87.0 of \sphinxhref{https://www.npmjs.com/package/request}{request}. License: \sphinxhref{https://opensource.org/licenses/Apache-2.0}{Apache\sphinxhyphen{}2.0}

\end{itemize}

On the other hand, specific NPM libraries have been installed for some functionalities:
\begin{itemize}
\item {} 
For the client:

\end{itemize}
\begin{itemize}
\item {} 
For the HTTP requests, the version 1.0.0\sphinxhyphen{}beta.2 of the library \sphinxhref{https://www.npmjs.com/package/@auth0/angular-jwt}{@auth0/angular\sphinxhyphen{}jwt} is used. License: \sphinxhref{https://opensource.org/licenses/MIT}{MIT}

\item {} 
For graphic representations \sphinxhref{https://www.npmjs.com/package/d3}{d3} has been installed. License: \sphinxhref{https://opensource.org/licenses/BSD-3-Clause}{BSD\sphinxhyphen{}3\sphinxhyphen{}Clause} Version 5.7.1 of @types/d3 and 1.2.2 of @types/d3\sphinxhyphen{}shape

\item {} 
version 2.0.0 of \sphinxhref{https://www.npmjs.com/package/fingerprintjs2}{Fingerprint2} is used to obtain the information of the device being used to access the platform. License: \sphinxhref{https://opensource.org/licenses/MIT}{MIT}

\item {} 
Version 0.24.1 of \sphinxhref{https://www.npmjs.com/package/angular-calendar}{angular\sphinxhyphen{}calendar} is used to work with the date entries of the platform. License: \sphinxhref{https://opensource.org/licenses/MIT}{MIT}

\item {} 
For data encryption, version 0.7.1 of \sphinxhref{https://www.npmjs.com/package/js-sha512}{js\sphinxhyphen{}sha512} is used and for the reverse process, version 2.2.0 of \sphinxhref{https://www.npmjs.com/package/jwt-decode}{jwt\sphinxhyphen{}decode} is used. License: \sphinxhref{https://opensource.org/licenses/MIT}{MIT}

\item {} 
Version 7.12.9 of \sphinxhref{https://www.npmjs.com/package/sweetalert2}{sweetalert2} is used for platform popups. License: \sphinxhref{https://opensource.org/licenses/MIT}{MIT}

\end{itemize}
\begin{itemize}
\item {} 
For the server

\end{itemize}
\begin{itemize}
\item {} 
For emailing tasks: 4.4.0 \sphinxhref{https://www.npmjs.com/package/nodemailer}{nodemailer}. License: \sphinxhref{https://opensource.org/licenses/MIT}{MIT} and 2.0.0 \sphinxhref{https://www.npmjs.com/package/nodemailer-express-handlebars}{nodemailer\sphinxhyphen{}express\sphinxhyphen{}handlebars}. License: \sphinxhref{https://opensource.org/licenses/MIT}{MIT}.

\item {} 
For communication with Azure: 0.10.6 of \sphinxhref{https://www.npmjs.com/package/azure-sb}{azure\sphinxhyphen{}sb}. License: \sphinxhref{https://opensource.org/licenses/Apache-2.0}{Apache\sphinxhyphen{}2.0}

\item {} 
To use the Authy application as 2FA: 1.4.0 of \sphinxhref{https://www.npmjs.com/package/authy}{authy}. License: \sphinxhref{https://opensource.org/licenses/MIT}{MIT} and 1.1.4 of \sphinxhref{https://www.npmjs.com/package/authy-client}{authy\sphinxhyphen{}client}. License: \sphinxhref{https://opensource.org/licenses/MIT}{MIT}

\item {} 
For data encryption, version 0.7.0 of \sphinxhref{https://www.npmjs.com/package/js-sha512}{js\sphinxhyphen{}sha512} is used. License: \sphinxhref{https://opensource.org/licenses/MIT}{MIT}

\item {} 
For storage, version 12.3.0 of \sphinxhref{https://www.npmjs.com/package/@azure/storage-blob}{azure/storage\sphinxhyphen{}blob} is used. License: \sphinxhref{https://opensource.org/licenses/MIT}{MIT}

\item {} 
For storage, version 2.10.3 of \sphinxhref{https://www.npmjs.com/package/azure-storage}{azure\sphinxhyphen{}storage} is used. License: \sphinxhref{https://opensource.org/licenses/Apache-2.0}{Apache\sphinxhyphen{}2.0}

\end{itemize}

In addition to this, Javascript scripts and JSON files have been added as libraries to optimize the programming of different functionalities for the client:
\begin{itemize}
\item {} 
Javascript scripts:

\end{itemize}
\begin{itemize}
\item {} 
\sphinxhref{http://hilios.github.io/jQuery.countdown/}{Jquery countdown}

\item {} 
\sphinxhref{https://www.npmjs.com/package/pdfjs}{PDF JS}

\item {} 
\sphinxhref{https://www.npmjs.com/package/@azure/storage-blob}{Azure storage}

\end{itemize}
\begin{itemize}
\item {} 
JSON Files located in src/assets/jsons:

\end{itemize}
\begin{itemize}
\item {} 
Some JSON files for managing the users’ location information (country, province, cities and phone codes): Countries folder, cities.json, countries.json, countries\_nl.json and phone\_codes.json

\item {} 
Some JSON files for obtain and manage the symptons: genes\_to\_phenotype.json, orpha\sphinxhyphen{}omim\sphinxhyphen{}orpha.json, orphaids.json and phenotypes.json.

\item {} 
A JSON with the languages available in Azure Cognitive Service for translation tasks: cognitive\sphinxhyphen{}services\sphinxhyphen{}languages.json

\item {} 
A JSON for the management of languages available on the platform: all\sphinxhyphen{}languages.json

\item {} 
A JSON for the types of subscription on the platform: subscription\sphinxhyphen{}types.json

\end{itemize}


\section{2.4.2. External APIs}
\label{\detokenize{pages/SW/Code:external-apis}}
\sphinxstylestrong{API Foundation29} has been designed to act as an intermediary between the webapp and Azure Qnamaker’s service.
The functions that have been implemented to perform actions on the Azure service are analogous to those in the \sphinxhref{https://docs.microsoft.com/en-us/azure/cognitive-services/qnamaker/quickstarts/quickstart-rest-curl}{Azure documentation}.

It can be consulted at the following link: \sphinxhref{https://f29api.northeurope.cloudapp.azure.com/index.html}{API Foundation29}.
And the functionality and methodology of use is described in the qnamaker section of this document (2.4.3.2. Qna maker)


\section{2.4.3. Azure cognitive services}
\label{\detokenize{pages/SW/Code:azure-cognitive-services}}

\subsection{2.4.3.1. Qna maker}
\label{\detokenize{pages/SW/Code:qna-maker}}
To create it from azure you just have to select the cognitive service in the marketplace: “Qna maker”.
The configuration has no complexity, just:
\begin{itemize}
\item {} 
Enter a name

\item {} 
Select subscription of Azure, the Price tier, the resource group adn the localization.

\item {} 
Enter an application name for establish the url: “app\_name.azurewebsites.net”

\end{itemize}

As previously stated, in Health29 it is used to manage the FAQs in different ways and from different points of the platform. To work with the data of this azure service, an external API will be used as an intermediary: f29API, described in the previous section, so that all calls will be made to it.



First, the Qna Maker Knowledge base ID, which the webapp wants to access (according to patient group and language), is queried in the qna collection.
With this ID it is possible to make requests to F29 API to obtain, modify or delete elements from the database.

However, the same format has been used as if the calls were made directly to Azure’s service, as an example:

\begin{sphinxVerbatim}[commandchars=\\\{\}]
\PYG{n}{this}\PYG{o}{.}\PYG{n}{http}\PYG{o}{.}\PYG{n}{get}\PYG{p}{(}\PYG{l+s+s1}{\PYGZsq{}}\PYG{l+s+s1}{https://f29api.northeurope.cloudapp.azure.com/api/knowledgebases/}\PYG{l+s+s1}{\PYGZsq{}}\PYG{o}{+}\PYG{n}{knowledgeBaseID}\PYG{o}{+}\PYG{l+s+s1}{\PYGZsq{}}\PYG{l+s+s1}{/Prod/qna}\PYG{l+s+s1}{\PYGZsq{}}\PYG{p}{)}
    \PYG{o}{.}\PYG{n}{map}\PYG{p}{(} \PYG{p}{(}\PYG{n}{res} \PYG{p}{:} \PYG{n+nb}{any}\PYG{p}{)} \PYG{o}{=}\PYG{o}{\PYGZgt{}} \PYG{p}{\PYGZob{}}
      \PYG{o}{/}\PYG{o}{/} \PYG{n}{Do} \PYG{n}{something} \PYG{n}{when} \PYG{n}{get} \PYG{n}{the} \PYG{n}{Knoledbase} \PYG{n}{information}
	\PYG{p}{\PYGZcb{}}\PYG{p}{,} \PYG{p}{(}\PYG{n}{err}\PYG{p}{)} \PYG{o}{=}\PYG{o}{\PYGZgt{}} \PYG{p}{\PYGZob{}}
     \PYG{o}{/}\PYG{o}{/} \PYG{n}{Do} \PYG{n}{something} \PYG{n}{when} \PYG{n}{error}
\PYG{p}{\PYGZcb{}}\PYG{p}{)}\PYG{p}{)}\PYG{p}{;}
\end{sphinxVerbatim}

And with the configuration of the headers (auth.interceptor.ts) to use this service the same thing happens, the ones that would be necessary to establish the connection with Azure’s service are sent:

\begin{sphinxVerbatim}[commandchars=\\\{\}]
\PYG{k}{if}\PYG{p}{(}\PYG{n}{req}\PYG{o}{.}\PYG{n}{url}\PYG{o}{.}\PYG{n}{indexOf}\PYG{p}{(}\PYG{l+s+s1}{\PYGZsq{}}\PYG{l+s+s1}{https://f29api.northeurope.cloudapp.azure.com}\PYG{l+s+s1}{\PYGZsq{}}\PYG{p}{)}\PYG{o}{!=}\PYG{o}{=}\PYG{o}{\PYGZhy{}}\PYG{l+m+mi}{1}\PYG{p}{)}\PYG{p}{\PYGZob{}}
  \PYG{n}{isExternalReq} \PYG{o}{=} \PYG{n}{true}\PYG{p}{;}
  \PYG{n}{const} \PYG{n}{headers} \PYG{o}{=} \PYG{n}{new} \PYG{n}{HttpHeaders}\PYG{p}{(}\PYG{p}{\PYGZob{}}
    \PYG{l+s+s1}{\PYGZsq{}}\PYG{l+s+s1}{Ocp\PYGZhy{}Apim\PYGZhy{}Subscription\PYGZhy{}Key}\PYG{l+s+s1}{\PYGZsq{}}\PYG{p}{:} \PYG{n}{environment}\PYG{o}{.}\PYG{n}{keyF29api}
  \PYG{p}{\PYGZcb{}}\PYG{p}{)}\PYG{p}{;}
  \PYG{n}{authReq} \PYG{o}{=} \PYG{n}{req}\PYG{o}{.}\PYG{n}{clone}\PYG{p}{(}\PYG{p}{\PYGZob{}} \PYG{n}{headers}\PYG{p}{\PYGZcb{}}\PYG{p}{)}\PYG{p}{;}
\PYG{p}{\PYGZcb{}}
\end{sphinxVerbatim}

All the commands and settings for establishing this communication are described in the \sphinxhref{https://docs.microsoft.com/en-us/azure/cognitive-services/qnamaker/quickstarts/quickstart-rest-curl}{Microsoft documentation}.


\subsection{2.4.3.2. Translator}
\label{\detokenize{pages/SW/Code:translator}}
To create it from azure you just have to select the cognitive service in the marketplace: “Translator text”.
The configuration has no complexity, just select the Price tier and the resource group.

It is used in the webapp client, as it has been indicated until now, establishing a REST communication:
\begin{itemize}
\item {} 
First case: communication without authorization required.

\end{itemize}

\begin{sphinxVerbatim}[commandchars=\\\{\}]
\PYG{n}{this}\PYG{o}{.}\PYG{n}{subscription}\PYG{o}{.}\PYG{n}{add}\PYG{p}{(} \PYG{n}{this}\PYG{o}{.}\PYG{n}{http}\PYG{o}{.}\PYG{n}{post}\PYG{p}{(}\PYG{l+s+s1}{\PYGZsq{}}\PYG{l+s+s1}{https://api.cognitive.microsofttranslator.com/translate?api\PYGZhy{}version=3.0\PYGZam{}to=}\PYG{l+s+s1}{\PYGZsq{}}\PYG{o}{+}\PYG{n}{this}\PYG{o}{.}\PYG{n}{authService}\PYG{o}{.}\PYG{n}{getLang}\PYG{p}{(}\PYG{p}{)}\PYG{p}{,} \PYG{n}{jsonText}\PYG{p}{)}
    \PYG{o}{.}\PYG{n}{subscribe}\PYG{p}{(} \PYG{p}{(}\PYG{n}{res} \PYG{p}{:} \PYG{n+nb}{any}\PYG{p}{)} \PYG{o}{=}\PYG{o}{\PYGZgt{}} \PYG{p}{\PYGZob{}}
    	\PYG{o}{/}\PYG{o}{/} \PYG{n}{Do} \PYG{n}{something} \PYG{n}{when} \PYG{n}{it} \PYG{n}{returns} \PYG{n}{the} \PYG{n}{text} \PYG{k+kn}{from} \PYG{n+nn}{jsonText} \PYG{n}{translated}
    \PYG{p}{\PYGZcb{}}\PYG{p}{,} \PYG{p}{(}\PYG{n}{err}\PYG{p}{)} \PYG{o}{=}\PYG{o}{\PYGZgt{}} \PYG{p}{\PYGZob{}}
       \PYG{o}{/}\PYG{o}{/} \PYG{n}{Do} \PYG{n}{something} \PYG{n}{when} \PYG{n}{error}
    \PYG{p}{\PYGZcb{}}\PYG{p}{)}\PYG{p}{)}\PYG{p}{;}
\end{sphinxVerbatim}
\begin{itemize}
\item {} 
Second case: communication with authorization required.

\end{itemize}

\begin{sphinxVerbatim}[commandchars=\\\{\}]
\PYG{n}{this}\PYG{o}{.}\PYG{n}{subscription}\PYG{o}{.}\PYG{n}{add}\PYG{p}{(} \PYG{n}{this}\PYG{o}{.}\PYG{n}{http}\PYG{o}{.}\PYG{n}{post}\PYG{p}{(}\PYG{l+s+s1}{\PYGZsq{}}\PYG{l+s+s1}{https://api.cognitive.microsoft.com/sts/v1.0/issueToken}\PYG{l+s+s1}{\PYGZsq{}}\PYG{p}{,}\PYG{l+s+s1}{\PYGZsq{}}\PYG{l+s+s1}{\PYGZsq{}}\PYG{p}{)}
    \PYG{o}{.}\PYG{n}{subscribe}\PYG{p}{(} \PYG{p}{(}\PYG{n}{res} \PYG{p}{:} \PYG{n+nb}{any}\PYG{p}{)} \PYG{o}{=}\PYG{o}{\PYGZgt{}} \PYG{p}{\PYGZob{}}
      	\PYG{o}{/}\PYG{o}{/} \PYG{n}{get} \PYG{n}{the} \PYG{n}{token} \PYG{k}{for} \PYG{n}{establish} \PYG{n}{communication} \PYG{n}{between} \PYG{n}{webapp} \PYG{o+ow}{and} \PYG{n}{text} \PYG{n}{translator}
        \PYG{n}{sessionStorage}\PYG{o}{.}\PYG{n}{setItem}\PYG{p}{(}\PYG{l+s+s1}{\PYGZsq{}}\PYG{l+s+s1}{tokenMicrosoftTranslator}\PYG{l+s+s1}{\PYGZsq{}}\PYG{p}{,} \PYG{n}{res}\PYG{p}{)}\PYG{p}{;}
	\PYG{p}{\PYGZcb{}}\PYG{p}{,} \PYG{p}{(}\PYG{n}{err}\PYG{p}{)} \PYG{o}{=}\PYG{o}{\PYGZgt{}} \PYG{p}{\PYGZob{}}
	 	\PYG{o}{/}\PYG{o}{/} \PYG{n}{Do} \PYG{n}{something} \PYG{n}{when} \PYG{n}{error}
	\PYG{p}{\PYGZcb{}}\PYG{p}{)}\PYG{p}{)}\PYG{p}{;}
\end{sphinxVerbatim}

\begin{sphinxVerbatim}[commandchars=\\\{\}]
\PYG{n}{this}\PYG{o}{.}\PYG{n}{subscription}\PYG{o}{.}\PYG{n}{add}\PYG{p}{(} \PYG{n}{this}\PYG{o}{.}\PYG{n}{http}\PYG{o}{.}\PYG{n}{get}\PYG{p}{(}\PYG{l+s+s1}{\PYGZsq{}}\PYG{l+s+s1}{https://api.microsofttranslator.com/V2/Http.svc/Translate?appid=Bearer}\PYG{l+s+s1}{\PYGZsq{}} \PYG{o}{+} \PYG{l+s+s1}{\PYGZsq{}}\PYG{l+s+s1}{ }\PYG{l+s+s1}{\PYGZsq{}} \PYG{o}{+} \PYG{n}{sessionStorage}\PYG{o}{.}\PYG{n}{getItem}\PYG{p}{(}\PYG{l+s+s1}{\PYGZsq{}}\PYG{l+s+s1}{tokenMicrosoftTranslator}\PYG{l+s+s1}{\PYGZsq{}}\PYG{p}{)}\PYG{o}{+}\PYG{l+s+s1}{\PYGZsq{}}\PYG{l+s+s1}{\PYGZam{}text=}\PYG{l+s+s1}{\PYGZsq{}}\PYG{o}{+}\PYG{o}{\PYGZlt{}}\PYG{n}{text}\PYG{o}{\PYGZgt{}}\PYG{o}{+}\PYG{l+s+s1}{\PYGZsq{}}\PYG{l+s+s1}{\PYGZam{}to=en\PYGZam{}Authorization=Bearer}\PYG{l+s+s1}{\PYGZsq{}} \PYG{o}{+} \PYG{l+s+s1}{\PYGZsq{}}\PYG{l+s+s1}{ }\PYG{l+s+s1}{\PYGZsq{}} \PYG{o}{+} \PYG{n}{sessionStorage}\PYG{o}{.}\PYG{n}{getItem}\PYG{p}{(}\PYG{l+s+s1}{\PYGZsq{}}\PYG{l+s+s1}{tokenMicrosoftTranslator}\PYG{l+s+s1}{\PYGZsq{}}\PYG{p}{)}\PYG{p}{)}
	\PYG{o}{.}\PYG{n}{subscribe}\PYG{p}{(} \PYG{p}{(}\PYG{n}{res} \PYG{p}{:} \PYG{n+nb}{any}\PYG{p}{)} \PYG{o}{=}\PYG{o}{\PYGZgt{}} \PYG{p}{\PYGZob{}}
		\PYG{o}{/}\PYG{o}{/} \PYG{n}{Do} \PYG{n}{something} \PYG{n}{when} \PYG{n}{it} \PYG{n}{returns} \PYG{n}{the} \PYG{n}{text} \PYG{k+kn}{from} \PYG{o}{\PYGZlt{}}\PYG{n}{text}\PYG{o}{\PYGZgt{}} \PYG{n}{translated}
	\PYG{p}{\PYGZcb{}}\PYG{p}{,} \PYG{p}{(}\PYG{n}{err}\PYG{p}{)} \PYG{o}{=}\PYG{o}{\PYGZgt{}} \PYG{p}{\PYGZob{}}
	 	\PYG{o}{/}\PYG{o}{/} \PYG{n}{Do} \PYG{n}{something} \PYG{n}{when} \PYG{n}{error}
	\PYG{p}{\PYGZcb{}}\PYG{p}{)}\PYG{p}{)}\PYG{p}{;}
\end{sphinxVerbatim}

And the configuration of the headers (auth.interceptor.ts) to use this service is:

\begin{sphinxVerbatim}[commandchars=\\\{\}]
\PYG{k}{if}\PYG{p}{(}\PYG{n}{req}\PYG{o}{.}\PYG{n}{url}\PYG{o}{.}\PYG{n}{indexOf}\PYG{p}{(}\PYG{l+s+s1}{\PYGZsq{}}\PYG{l+s+s1}{https://api.cognitive.microsoft.com/sts/v1.0/issueToken}\PYG{l+s+s1}{\PYGZsq{}}\PYG{p}{)}\PYG{o}{!=}\PYG{o}{=}\PYG{o}{\PYGZhy{}}\PYG{l+m+mi}{1}\PYG{p}{)}\PYG{p}{\PYGZob{}}
  \PYG{n}{isExternalReq} \PYG{o}{=} \PYG{n}{true}\PYG{p}{;}
  \PYG{n}{authReq} \PYG{o}{=} \PYG{n}{req}\PYG{o}{.}\PYG{n}{clone}\PYG{p}{(}\PYG{p}{\PYGZob{}} \PYG{n}{headers}\PYG{p}{:} \PYG{n}{req}\PYG{o}{.}\PYG{n}{headers}\PYG{o}{.}\PYG{n}{set}\PYG{p}{(}\PYG{l+s+s1}{\PYGZsq{}}\PYG{l+s+s1}{Ocp\PYGZhy{}Apim\PYGZhy{}Subscription\PYGZhy{}Key}\PYG{l+s+s1}{\PYGZsq{}}\PYG{p}{,}  \PYG{n}{environment}\PYG{o}{.}\PYG{n}{keyCognitiveMicrosoft} \PYG{p}{)}\PYG{p}{,} \PYG{n}{responseType}\PYG{p}{:} \PYG{l+s+s1}{\PYGZsq{}}\PYG{l+s+s1}{text}\PYG{l+s+s1}{\PYGZsq{}}\PYG{p}{\PYGZcb{}}\PYG{p}{)}\PYG{p}{;}
\PYG{p}{\PYGZcb{}}

\PYG{k}{if}\PYG{p}{(}\PYG{n}{req}\PYG{o}{.}\PYG{n}{url}\PYG{o}{.}\PYG{n}{indexOf}\PYG{p}{(}\PYG{l+s+s1}{\PYGZsq{}}\PYG{l+s+s1}{https://api.microsofttranslator.com}\PYG{l+s+s1}{\PYGZsq{}}\PYG{p}{)}\PYG{o}{!=}\PYG{o}{=}\PYG{o}{\PYGZhy{}}\PYG{l+m+mi}{1}\PYG{p}{)}\PYG{p}{\PYGZob{}}
  \PYG{n}{isExternalReq} \PYG{o}{=} \PYG{n}{true}\PYG{p}{;}
  \PYG{n}{authReq} \PYG{o}{=} \PYG{n}{req}\PYG{o}{.}\PYG{n}{clone}\PYG{p}{(}\PYG{p}{\PYGZob{}} \PYG{n}{responseType}\PYG{p}{:} \PYG{l+s+s1}{\PYGZsq{}}\PYG{l+s+s1}{text}\PYG{l+s+s1}{\PYGZsq{}} \PYG{p}{\PYGZcb{}}\PYG{p}{)}\PYG{p}{;}
\PYG{p}{\PYGZcb{}}
\PYG{k}{if}\PYG{p}{(}\PYG{n}{req}\PYG{o}{.}\PYG{n}{url}\PYG{o}{.}\PYG{n}{indexOf}\PYG{p}{(}\PYG{l+s+s1}{\PYGZsq{}}\PYG{l+s+s1}{https://api.cognitive.microsofttranslator.com}\PYG{l+s+s1}{\PYGZsq{}}\PYG{p}{)}\PYG{o}{!=}\PYG{o}{=}\PYG{o}{\PYGZhy{}}\PYG{l+m+mi}{1}\PYG{p}{)}\PYG{p}{\PYGZob{}}
  \PYG{n}{isExternalReq} \PYG{o}{=} \PYG{n}{true}\PYG{p}{;}
  \PYG{n}{authReq} \PYG{o}{=} \PYG{n}{req}\PYG{o}{.}\PYG{n}{clone}\PYG{p}{(}\PYG{p}{\PYGZob{}} \PYG{n}{headers}\PYG{p}{:} \PYG{n}{req}\PYG{o}{.}\PYG{n}{headers}\PYG{o}{.}\PYG{n}{set}\PYG{p}{(}\PYG{l+s+s1}{\PYGZsq{}}\PYG{l+s+s1}{Ocp\PYGZhy{}Apim\PYGZhy{}Subscription\PYGZhy{}Key}\PYG{l+s+s1}{\PYGZsq{}}\PYG{p}{,}  \PYG{n}{environment}\PYG{o}{.}\PYG{n}{keyCognitiveMicrosoft} \PYG{p}{)} \PYG{p}{\PYGZcb{}}\PYG{p}{)}\PYG{p}{;}
\PYG{p}{\PYGZcb{}}
\end{sphinxVerbatim}

The first of these is used to translate the datapoints and to add a new language to the platform, while the second is used in the symptom section (phenotype) to translate into English the text obtained from the vision service.

All the commands and settings for establishing this communication are described in the \sphinxhref{https://docs.microsoft.com/bs-cyrl-ba/azure/cognitive-services/translator/}{Microsoft documentation}.


\section{2.4.4. Azure Healthbot}
\label{\detokenize{pages/SW/Code:azure-healthbot}}
To create it from azure you just have to select the Healthcare Bot in the marketplace.
To create and configure it, just follow the steps in the \sphinxhref{https://docs.microsoft.com/en-us/healthbot/quickstart-createyourhealthcarebot}{Microsoft guide}.

In the case of Health29, the App service \sphinxhref{https://healthbotcontainersamplef666.scm.azurewebsites.net/dev/wwwroot/server.js}{server} has been configured to be able to deploy the three bots depending on the webapp client that is running it. So that:

\begin{sphinxVerbatim}[commandchars=\\\{\}]
\PYG{n}{var} \PYG{n}{WEBCHAT\PYGZus{}SECRET} \PYG{o}{=} \PYG{n}{process}\PYG{o}{.}\PYG{n}{env}\PYG{o}{.}\PYG{n}{WEBCHAT\PYGZus{}SECRET}\PYG{p}{;}
\PYG{n}{var} \PYG{n}{APP\PYGZus{}SECRET} \PYG{o}{=} \PYG{n}{process}\PYG{o}{.}\PYG{n}{env}\PYG{o}{.}\PYG{n}{APP\PYGZus{}SECRET}\PYG{p}{;}
\PYG{k}{if}\PYG{p}{(}\PYG{n}{req}\PYG{o}{.}\PYG{n}{headers}\PYG{o}{.}\PYG{n}{origin} \PYG{o}{==} \PYG{l+s+s1}{\PYGZsq{}}\PYG{l+s+s1}{https://health29.org}\PYG{l+s+s1}{\PYGZsq{}}\PYG{p}{)}\PYG{p}{\PYGZob{}}
    
\PYG{p}{\PYGZcb{}}\PYG{k}{else} \PYG{k}{if}\PYG{p}{(}\PYG{n}{req}\PYG{o}{.}\PYG{n}{headers}\PYG{o}{.}\PYG{n}{origin} \PYG{o}{==} \PYG{l+s+s1}{\PYGZsq{}}\PYG{l+s+s1}{https://health29\PYGZhy{}test.azurewebsites.net}\PYG{l+s+s1}{\PYGZsq{}}\PYG{p}{)}\PYG{p}{\PYGZob{}}
    \PYG{n}{WEBCHAT\PYGZus{}SECRET} \PYG{o}{=} \PYG{n}{process}\PYG{o}{.}\PYG{n}{env}\PYG{o}{.}\PYG{n}{WEBCHAT\PYGZus{}SECRET\PYGZus{}TEST}\PYG{p}{;}
    \PYG{n}{APP\PYGZus{}SECRET} \PYG{o}{=} \PYG{n}{process}\PYG{o}{.}\PYG{n}{env}\PYG{o}{.}\PYG{n}{APP\PYGZus{}SECRET\PYGZus{}TEST}\PYG{p}{;}
\PYG{p}{\PYGZcb{}}\PYG{k}{else} \PYG{k}{if}\PYG{p}{(}\PYG{n}{req}\PYG{o}{.}\PYG{n}{headers}\PYG{o}{.}\PYG{n}{origin} \PYG{o}{==} \PYG{l+s+s1}{\PYGZsq{}}\PYG{l+s+s1}{http://localhost:4200}\PYG{l+s+s1}{\PYGZsq{}} \PYG{o}{|}\PYG{o}{|} \PYG{n}{req}\PYG{o}{.}\PYG{n}{headers}\PYG{o}{.}\PYG{n}{origin} \PYG{o}{==}\PYG{l+s+s1}{\PYGZsq{}}\PYG{l+s+s1}{https://health29\PYGZhy{}dev.azurewebsites.net}\PYG{l+s+s1}{\PYGZsq{}}\PYG{p}{)}\PYG{p}{\PYGZob{}}
    \PYG{n}{WEBCHAT\PYGZus{}SECRET} \PYG{o}{=} \PYG{n}{process}\PYG{o}{.}\PYG{n}{env}\PYG{o}{.}\PYG{n}{WEBCHAT\PYGZus{}SECRET\PYGZus{}DEV}\PYG{p}{;}
    \PYG{n}{APP\PYGZus{}SECRET} \PYG{o}{=} \PYG{n}{process}\PYG{o}{.}\PYG{n}{env}\PYG{o}{.}\PYG{n}{APP\PYGZus{}SECRET\PYGZus{}DEV}\PYG{p}{;}
\PYG{p}{\PYGZcb{}}
\end{sphinxVerbatim}

It should be noted that this service has a monthly message limit. Depending on the rate chosen in each of the Healthbots, a greater or lesser flow of information exchange or interactions of the Health29 assistant will be allowed.

\sphinxstyleemphasis{NOTE: if at any time the limit is exceeded, the bot will return error 429 (Too many request).}

The incorporation of this assistant to the webapp is done from the client, in particular, from the customizer component.
This way, it is added to the HTML:

\begin{sphinxVerbatim}[commandchars=\\\{\}]
\PYG{o}{\PYGZlt{}}\PYG{n}{div} \PYG{n+nb}{id}\PYG{o}{=}\PYG{l+s+s2}{\PYGZdq{}}\PYG{l+s+s2}{botContainer}\PYG{l+s+s2}{\PYGZdq{}} \PYG{n}{style}\PYG{o}{=}\PYG{l+s+s2}{\PYGZdq{}}\PYG{l+s+s2}{width:95}\PYG{l+s+s2}{\PYGZpc{}}\PYG{l+s+s2}{\PYGZdq{}}\PYG{o}{\PYGZgt{}}\PYG{o}{\PYGZlt{}}\PYG{o}{/}\PYG{n}{div}\PYG{o}{\PYGZgt{}}
\end{sphinxVerbatim}

And the style is added from the scss file.

As for the functionality of the bot, this is included in the “.ts” file, taking into account:



Health Bot uses \sphinxhref{https://dev.botframework.com/}{Bot Framework} under the hood as a messaging and routing platform to deliver messages to and from the end user.

So, the steps to follow are:
\begin{enumerate}
\sphinxsetlistlabels{\arabic}{enumi}{enumii}{}{.}%
\item {} 
Get the token for the connection with the azure service:

\end{enumerate}

\begin{sphinxVerbatim}[commandchars=\\\{\}]
\PYG{n}{var} \PYG{n}{paramssend} \PYG{o}{=} \PYG{p}{\PYGZob{}} \PYG{n}{userName}\PYG{p}{:} \PYG{n}{this}\PYG{o}{.}\PYG{n}{user}\PYG{o}{.}\PYG{n}{userName}\PYG{p}{,} \PYG{n}{userId}\PYG{p}{:} \PYG{n}{this}\PYG{o}{.}\PYG{n}{authService}\PYG{o}{.}\PYG{n}{getIdUser}\PYG{p}{(}\PYG{p}{)}\PYG{p}{,} \PYG{n}{token}\PYG{p}{:} \PYG{n}{this}\PYG{o}{.}\PYG{n}{authService}\PYG{o}{.}\PYG{n}{getToken}\PYG{p}{(}\PYG{p}{)}\PYG{p}{,} \PYG{n}{groupId}\PYG{p}{:} \PYG{n}{this}\PYG{o}{.}\PYG{n}{groupId}\PYG{p}{,} \PYG{n}{lang}\PYG{p}{:} \PYG{n}{this}\PYG{o}{.}\PYG{n}{user}\PYG{o}{.}\PYG{n}{lang}\PYG{p}{,} \PYG{n}{knowledgeBaseID}\PYG{p}{:} \PYG{n}{this}\PYG{o}{.}\PYG{n}{knowledgeBaseID}\PYG{p}{\PYGZcb{}}\PYG{p}{;} 
\PYG{n}{this}\PYG{o}{.}\PYG{n}{subscription}\PYG{o}{.}\PYG{n}{add}\PYG{p}{(} \PYG{n}{this}\PYG{o}{.}\PYG{n}{http}\PYG{o}{.}\PYG{n}{get}\PYG{p}{(}\PYG{l+s+s1}{\PYGZsq{}}\PYG{l+s+s1}{https://healthbotcontainersamplef666.azurewebsites.net:443/chatBot}\PYG{l+s+s1}{\PYGZsq{}}\PYG{p}{,}\PYG{p}{\PYGZob{}}\PYG{n}{params}\PYG{p}{:} \PYG{n}{paramssend}\PYG{p}{\PYGZcb{}}\PYG{p}{)}
	\PYG{o}{.}\PYG{n}{subscribe}\PYG{p}{(} \PYG{p}{(}\PYG{n}{res} \PYG{p}{:} \PYG{n+nb}{any}\PYG{p}{)} \PYG{o}{=}\PYG{o}{\PYGZgt{}} \PYG{p}{\PYGZob{}}
		\PYG{o}{/}\PYG{o}{/} \PYG{n}{Get} \PYG{n}{the} \PYG{n}{token} \PYG{n}{OK}
	\PYG{p}{\PYGZcb{}}\PYG{p}{,} \PYG{p}{(}\PYG{n}{err}\PYG{p}{)} \PYG{o}{=}\PYG{o}{\PYGZgt{}} \PYG{p}{\PYGZob{}}
		\PYG{o}{/}\PYG{o}{/} \PYG{n}{Get} \PYG{n}{the} \PYG{n}{token} \PYG{n}{Error}
\PYG{p}{\PYGZcb{}}\PYG{p}{)}\PYG{p}{)}\PYG{p}{;}
\end{sphinxVerbatim}
\begin{enumerate}
\sphinxsetlistlabels{\arabic}{enumi}{enumii}{}{.}%
\item {} 
Create a bot connection:

\end{enumerate}

\begin{sphinxVerbatim}[commandchars=\\\{\}]
\PYG{n}{const} \PYG{n}{jsonWebToken} \PYG{o}{=} \PYG{n}{token}\PYG{p}{;}
\PYG{n}{const} \PYG{n}{tokenPayload} \PYG{o}{=} \PYG{n}{JSON}\PYG{o}{.}\PYG{n}{parse}\PYG{p}{(}\PYG{n}{atob}\PYG{p}{(}\PYG{n}{jsonWebToken}\PYG{o}{.}\PYG{n}{split}\PYG{p}{(}\PYG{l+s+s1}{\PYGZsq{}}\PYG{l+s+s1}{.}\PYG{l+s+s1}{\PYGZsq{}}\PYG{p}{)}\PYG{p}{[}\PYG{l+m+mi}{1}\PYG{p}{]}\PYG{p}{)}\PYG{p}{)}\PYG{p}{;}
\PYG{n}{const} \PYG{n}{user} \PYG{o}{=} \PYG{p}{\PYGZob{}}
  \PYG{n+nb}{id}\PYG{p}{:} \PYG{n}{tokenPayload}\PYG{o}{.}\PYG{n}{userId}\PYG{p}{,}
  \PYG{n}{name}\PYG{p}{:} \PYG{n}{tokenPayload}\PYG{o}{.}\PYG{n}{userName}
\PYG{p}{\PYGZcb{}}\PYG{p}{;}
\PYG{n}{const} \PYG{n}{botConnection} \PYG{o}{=} \PYG{n}{new} \PYG{n}{BotChat}\PYG{o}{.}\PYG{n}{DirectLine}\PYG{p}{(}\PYG{p}{\PYGZob{}}
  \PYG{o}{/}\PYG{o}{/}\PYG{n}{secret}\PYG{p}{:} \PYG{n}{botSecret}\PYG{p}{,}
  \PYG{n}{token}\PYG{p}{:} \PYG{n}{tokenPayload}\PYG{o}{.}\PYG{n}{connectorToken}\PYG{p}{,}
  \PYG{o}{/}\PYG{o}{/}\PYG{n}{domain}\PYG{p}{:} \PYG{l+s+s2}{\PYGZdq{}}\PYG{l+s+s2}{\PYGZdq{}}\PYG{p}{,}
  \PYG{n}{webSocket}\PYG{p}{:} \PYG{n}{true}
\PYG{p}{\PYGZcb{}}\PYG{p}{)}\PYG{p}{;}
\end{sphinxVerbatim}
\begin{enumerate}
\sphinxsetlistlabels{\arabic}{enumi}{enumii}{}{.}%
\item {} 
Start conversation

\end{enumerate}

\begin{sphinxVerbatim}[commandchars=\\\{\}]
const botContainer = document.getElementById(\PYGZsq{}botContainer\PYGZsq{});
\PYGZdl{}(\PYGZsq{}\PYGZsh{}botContainer\PYGZsq{}).empty();
botContainer.classList.add(\PYGZdq{}wc\PYGZhy{}display\PYGZdq{});   
BotChat.App(\PYGZob{}
    botConnection: botConnection,
    user: user,
    locale: this.user.lang,
    bot: \PYGZob{}id: \PYGZsq{}\PYGZsq{}\PYGZcb{},
    resize: \PYGZsq{}detect\PYGZsq{}
    // sendTyping: true,    // defaults to false. set to true to send \PYGZsq{}typing\PYGZsq{} activities to bot (and other users) when user is typing
\PYGZcb{}, botContainer);
\end{sphinxVerbatim}
\begin{enumerate}
\sphinxsetlistlabels{\arabic}{enumi}{enumii}{}{.}%
\item {} 
A communication will be established with it to exchange information and perform the relevant actions:

\end{enumerate}

\begin{sphinxVerbatim}[commandchars=\\\{\}]
\PYG{n}{this}\PYG{o}{.}\PYG{n}{subscription}\PYG{o}{.}\PYG{n}{add}\PYG{p}{(} \PYG{n}{botConnection}\PYG{o}{.}\PYG{n}{postActivity}\PYG{p}{(}\PYG{p}{\PYGZob{}}\PYG{n+nb}{type}\PYG{p}{:} \PYG{l+s+s2}{\PYGZdq{}}\PYG{l+s+s2}{event}\PYG{l+s+s2}{\PYGZdq{}}\PYG{p}{,} \PYG{n}{value}\PYG{p}{:} \PYG{n}{jsonWebToken}\PYG{p}{,} \PYG{n}{from}\PYG{p}{:} \PYG{n}{user}\PYG{p}{,} \PYG{n}{name}\PYG{p}{:} \PYG{l+s+s2}{\PYGZdq{}}\PYG{l+s+s2}{InitAuthenticatedConversation}\PYG{l+s+s2}{\PYGZdq{}}\PYG{p}{\PYGZcb{}}\PYG{p}{)}\PYG{o}{.}\PYG{n}{subscribe}\PYG{p}{(}\PYG{n}{function} \PYG{p}{(}\PYG{n+nb}{id}\PYG{p}{)} \PYG{p}{\PYGZob{}}
     \PYG{o}{/}\PYG{o}{/} \PYG{n}{Start} \PYG{n}{scenarios}
\PYG{p}{\PYGZcb{}}\PYG{p}{,} \PYG{p}{(}\PYG{n}{err}\PYG{p}{)} \PYG{o}{=}\PYG{o}{\PYGZgt{}} \PYG{p}{\PYGZob{}}
		\PYG{o}{/}\PYG{o}{/} \PYG{n}{Do} \PYG{n}{something} \PYG{n}{when} \PYG{n}{error}
\PYG{p}{\PYGZcb{}}\PYG{p}{)}\PYG{p}{)}\PYG{p}{;}
\end{sphinxVerbatim}

All the commands and settings for using this Healthbot service ara available in \sphinxhref{https://docs.microsoft.com/en-us/healthbot/integrations/embed}{Microsoft documentation}.

In addition to this, the HTTP headers have to be configured to be able to use this service in the auth.interceptor.ts file:

\begin{sphinxVerbatim}[commandchars=\\\{\}]
\PYG{k}{if}\PYG{p}{(}\PYG{n}{req}\PYG{o}{.}\PYG{n}{url}\PYG{o}{.}\PYG{n}{indexOf}\PYG{p}{(}\PYG{l+s+s1}{\PYGZsq{}}\PYG{l+s+s1}{healthbot}\PYG{l+s+s1}{\PYGZsq{}}\PYG{p}{)}\PYG{o}{!=}\PYG{o}{=}\PYG{o}{\PYGZhy{}}\PYG{l+m+mi}{1}\PYG{p}{)}\PYG{p}{\PYGZob{}}
  \PYG{n}{isExternalReq} \PYG{o}{=} \PYG{n}{true}\PYG{p}{;}
  \PYG{n}{const} \PYG{n}{headers} \PYG{o}{=} \PYG{n}{new} \PYG{n}{HttpHeaders}\PYG{p}{(}\PYG{p}{\PYGZob{}}
    \PYG{l+s+s1}{\PYGZsq{}}\PYG{l+s+s1}{Content\PYGZhy{}Type}\PYG{l+s+s1}{\PYGZsq{}}\PYG{p}{:}  \PYG{l+s+s1}{\PYGZsq{}}\PYG{l+s+s1}{text/html; charset=utf\PYGZhy{}8}\PYG{l+s+s1}{\PYGZsq{}}\PYG{p}{,}
    \PYG{l+s+s1}{\PYGZsq{}}\PYG{l+s+s1}{Access\PYGZhy{}Control\PYGZhy{}Allow\PYGZhy{}Methods}\PYG{l+s+s1}{\PYGZsq{}}\PYG{p}{:} \PYG{l+s+s1}{\PYGZsq{}}\PYG{l+s+s1}{GET}\PYG{l+s+s1}{\PYGZsq{}}
  \PYG{p}{\PYGZcb{}}\PYG{p}{)}\PYG{p}{;}
  \PYG{n}{authReq} \PYG{o}{=} \PYG{n}{req}\PYG{o}{.}\PYG{n}{clone}\PYG{p}{(}\PYG{p}{\PYGZob{}} \PYG{n}{headers}\PYG{p}{,} \PYG{n}{responseType}\PYG{p}{:} \PYG{l+s+s1}{\PYGZsq{}}\PYG{l+s+s1}{text}\PYG{l+s+s1}{\PYGZsq{}}\PYG{p}{\PYGZcb{}}\PYG{p}{)}\PYG{p}{;}\PYG{o}{/}\PYG{o}{/}\PYG{l+s+s1}{\PYGZsq{}}\PYG{l+s+s1}{Content\PYGZhy{}Type}\PYG{l+s+s1}{\PYGZsq{}}\PYG{p}{,}  \PYG{l+s+s1}{\PYGZsq{}}\PYG{l+s+s1}{application/json}\PYG{l+s+s1}{\PYGZsq{}}
\PYG{p}{\PYGZcb{}}
\end{sphinxVerbatim}

As already indicated, once the bot is embedded in the webapp you will have to indicate it to initialize the desired scenarios in each case.

In the microsoft documentation there are \sphinxhref{https://docs.microsoft.com/en-us/healthbot/quickstart-createyourfirstscenario}{guides} to help you create these scenarios.

The general workflow of the Healthbot scenarios defined for Health29 is



Different scenarios could be designed for each patient group. Thus, for example, the flow of the main scenario of the Health29 assistant for the Duchenne group can be schematized as:



Except for the control scenarios that do not send messages to the user, the rest are replicated: one is created per user language available in Health29 (as can be seen in the previous diagram, the nomenclature “scenarioName\_lang” has been used).
In this way, the scenarios that will be executed depend on the language selected by the user on the Health29 platform.

A modification of this was started to use the Healthbot localization service to avoid scenario replications. However, there is still work to be done since this translation could not be managed for some messages sent by the bot to notify of problems or errors. This has been solved so far from the Health29 client code.
For example:

\begin{sphinxVerbatim}[commandchars=\\\{\}]
 if((activity.text ===\PYGZdq{}I didn\PYGZsq{}t understand. Please choose an option from the list.\PYGZdq{}))\PYGZob{}
  var x = document.getElementsByClassName(\PYGZdq{}format\PYGZhy{}markdown\PYGZdq{});
  for (var element=0;element\PYGZlt{}x.length;element++)\PYGZob{}
    var stringElement=x[element].firstElementChild.childNodes[0].textContent;
    if((stringElement ==\PYGZdq{}I didn’t understand. Please choose an option from the list.\PYGZdq{})||
    (stringElement==\PYGZdq{}No se pudo reconocer su respuesta. Por favor seleccione una opción de la lista.\PYGZdq{}))\PYGZob{}
      var tempElement=x[element].firstElementChild.childNodes[0];
      tempElement.textContent = this.translate.instant(\PYGZdq{}generics.Understand\PYGZdq{});
      \PYGZdl{}(x[element].firstElementChild.childNodes[0]).replaceWith(tempElement.textContent);
    \PYGZcb{}
  \PYGZcb{}
\end{sphinxVerbatim}

In this case, the messages shown by the bot are read and translated using the translation system used in Health29 that is explained in the “Multilanguage” section of this document.

Finally, it should also be noted that an action flow has been defined for the task of closing or minimizing the platform’s assistant, taking into account that
\begin{itemize}
\item {} 
This will be displayed whenever a user logs in.

\item {} 
The user will be given the option to end the conversation, so previous messages will be deleted and the wizard will be minimized.

\item {} 
The behavior of the buttons to maximize/minimize and open/close the assistant in Health29 is maintained.

\end{itemize}


\section{2.4.5. Azure blobs}
\label{\detokenize{pages/SW/Code:azure-blobs}}
To create it from azure you just have to search in the marketplace: “Storage account”.
The configuration has no complexity,
\begin{itemize}
\item {} 
Fill project details: subscription and resource group

\item {} 
Fill instance details:

\end{itemize}
\begin{itemize}
\item {} 
Storage account name

\item {} 
Localization

\item {} 
Performance (default standard)

\item {} 
Account kind (blobstorage)

\item {} 
Replication

\item {} 
Access tier (hot)

\end{itemize}

It is used in the webapp client establishing a REST communication.
In this project, the Javascript library Azure storage is used. In this way, some services have been created in the client project of Angular that will allow working with the reading and writing functions of the Azure blob.
Thus, the following services have been defined:
\begin{itemize}
\item {} 
blob\sphinxhyphen{}storage\sphinxhyphen{}medical\sphinxhyphen{}care.service.ts

\item {} 
blob\sphinxhyphen{}storage\sphinxhyphen{}support.service.ts

\item {} 
blob\sphinxhyphen{}storage.service.ts

\end{itemize}

In all cases, an interface to work with blobs like this will be exported:

\begin{sphinxVerbatim}[commandchars=\\\{\}]
\PYG{n}{export} \PYG{n}{interface} \PYG{n}{IBlobAccessToken} \PYG{p}{\PYGZob{}}
  \PYG{n}{blobAccountUrl}\PYG{p}{:} \PYG{n}{string}\PYG{p}{;}
  \PYG{n}{sasToken}\PYG{p}{:} \PYG{n}{string}\PYG{p}{;}
  \PYG{n}{containerName}\PYG{p}{:} \PYG{n}{string}\PYG{p}{;}
  \PYG{n}{patientId}\PYG{p}{:} \PYG{n}{string}\PYG{p}{;} \PYG{p}{(}\PYG{n}{optional}\PYG{p}{)}
\PYG{p}{\PYGZcb{}}
\end{sphinxVerbatim}

and a class will be defined with the possible functions that can be performed with the blob:

\begin{sphinxVerbatim}[commandchars=\\\{\}]
\PYG{n+nd}{@Injectable}\PYG{p}{(}\PYG{p}{)}
\PYG{n}{export} \PYG{k}{class} \PYG{n+nc}{BlobStorage}\PYG{o}{\PYGZlt{}}\PYG{n}{name}\PYG{o}{\PYGZgt{}}\PYG{p}{\PYGZob{}}\PYG{o}{.}\PYG{o}{.}\PYG{o}{.}\PYG{p}{\PYGZcb{}}
\end{sphinxVerbatim}

With this, from the different sections of Health29 it will be possible to create the containers in the blob and store the relevant information.
An example of use would be:

\begin{sphinxVerbatim}[commandchars=\\\{\}]
\PYG{n}{constructor} \PYG{o}{.}\PYG{o}{.}\PYG{o}{.} \PYG{p}{(}\PYG{o}{.}\PYG{o}{.}\PYG{o}{.}\PYG{p}{,}\PYG{n}{private} \PYG{n}{blob}\PYG{p}{:} \PYG{n}{BlobStorageMedicalCareService}\PYG{p}{,}\PYG{o}{.}\PYG{o}{.}\PYG{o}{.}\PYG{p}{)}
\PYG{o}{.}\PYG{o}{.}\PYG{o}{.}
\PYG{n}{this}\PYG{o}{.}\PYG{n}{blob}\PYG{o}{.}\PYG{n}{uploadToBlobStorage}\PYG{p}{(}\PYG{n}{this}\PYG{o}{.}\PYG{n}{accessToken}\PYG{p}{,} \PYG{o}{\PYGZlt{}}\PYG{n}{files}\PYG{o}{\PYGZgt{}}\PYG{p}{,} \PYG{n}{filename}\PYG{p}{,} \PYG{n}{index1}\PYG{p}{,} \PYG{n}{index2}\PYG{p}{)}\PYG{p}{;}
\end{sphinxVerbatim}

In addition to this, In order to perform operations with the Blobs, we need the sasToken. src\textbackslash{}app\textbackslash{}shared\textbackslash{}services\textbackslash{}api\sphinxhyphen{}dx29\sphinxhyphen{}server.service.ts file on the client:

\begin{sphinxVerbatim}[commandchars=\\\{\}]
\PYG{n}{getAzureBlobSasToken}\PYG{p}{(}\PYG{n}{containerName}\PYG{p}{)}\PYG{p}{\PYGZob{}}
        \PYG{k}{return} \PYG{n}{this}\PYG{o}{.}\PYG{n}{http}\PYG{o}{.}\PYG{n}{get}\PYG{p}{(}\PYG{n}{environment}\PYG{o}{.}\PYG{n}{api}\PYG{o}{+}\PYG{l+s+s1}{\PYGZsq{}}\PYG{l+s+s1}{/api/getAzureBlobSasTokenWithContainer/}\PYG{l+s+s1}{\PYGZsq{}}\PYG{o}{+}\PYG{n}{containerName}\PYG{p}{)}
        \PYG{o}{.}\PYG{n}{map}\PYG{p}{(} \PYG{p}{(}\PYG{n}{res} \PYG{p}{:} \PYG{n+nb}{any}\PYG{p}{)} \PYG{o}{=}\PYG{o}{\PYGZgt{}} \PYG{p}{\PYGZob{}}
            \PYG{k}{return} \PYG{n}{res}\PYG{o}{.}\PYG{n}{containerSAS}\PYG{p}{;}
        \PYG{p}{\PYGZcb{}}\PYG{p}{,} \PYG{p}{(}\PYG{n}{err}\PYG{p}{)} \PYG{o}{=}\PYG{o}{\PYGZgt{}} \PYG{p}{\PYGZob{}}
            \PYG{n}{console}\PYG{o}{.}\PYG{n}{log}\PYG{p}{(}\PYG{n}{err}\PYG{p}{)}\PYG{p}{;}
            \PYG{k}{return} \PYG{n}{err}\PYG{p}{;}
        \PYG{p}{\PYGZcb{}}\PYG{p}{)}
    \PYG{p}{\PYGZcb{}}
\end{sphinxVerbatim}

On the server:

\begin{sphinxVerbatim}[commandchars=\\\{\}]
\PYG{n}{function} \PYG{n}{getAzureBlobSasTokenWithContainer} \PYG{p}{(}\PYG{n}{req}\PYG{p}{,} \PYG{n}{res}\PYG{p}{)}\PYG{p}{\PYGZob{}}
  \PYG{n}{var} \PYG{n}{containerName} \PYG{o}{=} \PYG{n}{req}\PYG{o}{.}\PYG{n}{params}\PYG{o}{.}\PYG{n}{containerName}\PYG{p}{;}

  \PYG{n}{var} \PYG{n}{startDate} \PYG{o}{=} \PYG{n}{new} \PYG{n}{Date}\PYG{p}{(}\PYG{p}{)}\PYG{p}{;}
  \PYG{n}{var} \PYG{n}{expiryDate} \PYG{o}{=} \PYG{n}{new} \PYG{n}{Date}\PYG{p}{(}\PYG{p}{)}\PYG{p}{;}
  \PYG{n}{startDate}\PYG{o}{.}\PYG{n}{setTime}\PYG{p}{(}\PYG{n}{startDate}\PYG{o}{.}\PYG{n}{getTime}\PYG{p}{(}\PYG{p}{)} \PYG{o}{\PYGZhy{}} \PYG{l+m+mi}{5}\PYG{o}{*}\PYG{l+m+mi}{60}\PYG{o}{*}\PYG{l+m+mi}{1000}\PYG{p}{)}\PYG{p}{;}
  \PYG{n}{expiryDate}\PYG{o}{.}\PYG{n}{setTime}\PYG{p}{(}\PYG{n}{expiryDate}\PYG{o}{.}\PYG{n}{getTime}\PYG{p}{(}\PYG{p}{)} \PYG{o}{+} \PYG{l+m+mi}{24}\PYG{o}{*}\PYG{l+m+mi}{60}\PYG{o}{*}\PYG{l+m+mi}{60}\PYG{o}{*}\PYG{l+m+mi}{1000}\PYG{p}{)}\PYG{p}{;}

  \PYG{n}{var} \PYG{n}{containerSAS} \PYG{o}{=} \PYG{n}{storage}\PYG{o}{.}\PYG{n}{generateBlobSASQueryParameters}\PYG{p}{(}\PYG{p}{\PYGZob{}}
      \PYG{n}{expiresOn} \PYG{p}{:} \PYG{n}{expiryDate}\PYG{p}{,}
      \PYG{n}{permissions}\PYG{p}{:} \PYG{n}{storage}\PYG{o}{.}\PYG{n}{ContainerSASPermissions}\PYG{o}{.}\PYG{n}{parse}\PYG{p}{(}\PYG{l+s+s2}{\PYGZdq{}}\PYG{l+s+s2}{rwdlac}\PYG{l+s+s2}{\PYGZdq{}}\PYG{p}{)}\PYG{p}{,}
      \PYG{n}{protocol}\PYG{p}{:} \PYG{n}{storage}\PYG{o}{.}\PYG{n}{SASProtocol}\PYG{o}{.}\PYG{n}{Https}\PYG{p}{,}
      \PYG{n}{containerName}\PYG{p}{:} \PYG{n}{containerName}\PYG{p}{,}
      \PYG{n}{startsOn}\PYG{p}{:} \PYG{n}{startDate}\PYG{p}{,}
      \PYG{n}{version}\PYG{p}{:}\PYG{o}{\PYGZlt{}}\PYG{n}{version}\PYG{o}{\PYGZgt{}}

    \PYG{p}{\PYGZcb{}}\PYG{p}{,}\PYG{n}{sharedKeyCredentialGenomics}\PYG{p}{)}\PYG{o}{.}\PYG{n}{toString}\PYG{p}{(}\PYG{p}{)}\PYG{p}{;}
  \PYG{n}{res}\PYG{o}{.}\PYG{n}{status}\PYG{p}{(}\PYG{l+m+mi}{200}\PYG{p}{)}\PYG{o}{.}\PYG{n}{send}\PYG{p}{(}\PYG{p}{\PYGZob{}}\PYG{n}{containerSAS}\PYG{p}{:} \PYG{n}{containerSAS}\PYG{p}{\PYGZcb{}}\PYG{p}{)}
\PYG{p}{\PYGZcb{}}
\end{sphinxVerbatim}


\section{2.4.6. Databases}
\label{\detokenize{pages/SW/Code:databases}}
The databases are created in Azure as \sphinxhref{https://docs.microsoft.com/en-US/azure/cosmos-db/}{Azure Cosmos DB account}.To create it you just have to select in the marketplace: “Azure Cosmo db” and configure it.

As mentioned throughout the development of this document, there will be 4 databases: two for the test and development environments, and two for the production environment.
The structure of the databases of the two environments will be the same, so we have:



Each database will be composed of some collections that will be accessed from the Health29 platform server.

In the following sections we will explain how this connection between Health29 and Azure’s databases is made, and we will go deeper into each of them by explaining the structure and layout of the collections they contain in each case.


\subsection{2.4.6.1. Databases and health29 communication}
\label{\detokenize{pages/SW/Code:databases-and-health29-communication}}
As indicated above, the communication is established using mongoose, so for each model or scheme in a database collection, we will have

\begin{sphinxVerbatim}[commandchars=\\\{\}]
\PYG{n}{const} \PYG{n}{mongoose} \PYG{o}{=} \PYG{n}{require} \PYG{p}{(}\PYG{l+s+s1}{\PYGZsq{}}\PYG{l+s+s1}{mongoose}\PYG{l+s+s1}{\PYGZsq{}}\PYG{p}{)}\PYG{p}{;}
\PYG{n}{const} \PYG{p}{\PYGZob{}} \PYG{n}{conndbaccounts} \PYG{p}{\PYGZcb{}} \PYG{o}{=} \PYG{n}{require}\PYG{p}{(}\PYG{l+s+s1}{\PYGZsq{}}\PYG{l+s+s1}{../db\PYGZus{}connect}\PYG{l+s+s1}{\PYGZsq{}}\PYG{p}{)}

\PYG{p}{[}\PYG{o}{.}\PYG{o}{.}\PYG{o}{.}\PYG{p}{]} \PYG{p}{\PYGZob{}}\PYG{n}{Schema} \PYG{n}{definition}\PYG{p}{\PYGZcb{}}

\PYG{n}{module}\PYG{o}{.}\PYG{n}{exports} \PYG{o}{=} \PYG{n}{conndbaccounts}\PYG{o}{.}\PYG{n}{model}\PYG{p}{(}\PYG{o}{\PYGZlt{}}\PYG{n}{Collection} \PYG{n}{name}\PYG{o}{\PYGZgt{}}\PYG{p}{,}\PYG{o}{\PYGZlt{}}\PYG{n}{schema} \PYG{n}{name}\PYG{o}{\PYGZgt{}}\PYG{p}{)}
\end{sphinxVerbatim}


\subsection{2.4.6.2. DDBB: Accounts}
\label{\detokenize{pages/SW/Code:ddbb-accounts}}
\sphinxstylestrong{Alerts}
In this collection are saved the notifications or alerts created by the administrator or super administrator profiles, indicating the group of users (receivers) to which they will be sent/displayed. It also includes fields to manage the language of the notification.

\begin{sphinxVerbatim}[commandchars=\\\{\}]
\PYG{n}{const} \PYG{n}{AlertsSchema} \PYG{o}{=} \PYG{n}{Schema}\PYG{p}{(}\PYG{p}{\PYGZob{}}
    \PYG{n}{groupId}\PYG{p}{:} \PYG{p}{\PYGZob{}}\PYG{n+nb}{type}\PYG{p}{:}\PYG{n}{String}\PYG{p}{,} \PYG{n}{default}\PYG{p}{:}\PYG{l+s+s2}{\PYGZdq{}}\PYG{l+s+s2}{None}\PYG{l+s+s2}{\PYGZdq{}}\PYG{p}{\PYGZcb{}}\PYG{p}{,}
    \PYG{n+nb}{type}\PYG{p}{:} \PYG{p}{\PYGZob{}}\PYG{n+nb}{type}\PYG{p}{:}\PYG{n}{String}\PYG{p}{,} \PYG{n}{default}\PYG{p}{:}\PYG{l+s+s2}{\PYGZdq{}}\PYG{l+s+s2}{\PYGZdq{}}\PYG{p}{\PYGZcb{}}\PYG{p}{,}
    \PYG{n}{identifier}\PYG{p}{:} \PYG{p}{\PYGZob{}}\PYG{n+nb}{type}\PYG{p}{:}\PYG{n}{String}\PYG{p}{,} \PYG{n}{default}\PYG{p}{:}\PYG{l+s+s2}{\PYGZdq{}}\PYG{l+s+s2}{\PYGZdq{}}\PYG{p}{\PYGZcb{}}\PYG{p}{,}
    \PYG{n}{translatedName}\PYG{p}{:} \PYG{p}{\PYGZob{}}\PYG{n+nb}{type}\PYG{p}{:} \PYG{n}{Object}\PYG{p}{,} \PYG{n}{default}\PYG{p}{:} \PYG{p}{[}\PYG{p}{]}\PYG{p}{\PYGZcb{}}\PYG{p}{,}
    \PYG{n}{launchDate}\PYG{p}{:} \PYG{p}{\PYGZob{}}\PYG{n+nb}{type}\PYG{p}{:} \PYG{n}{Date}\PYG{p}{,} \PYG{n}{default}\PYG{p}{:} \PYG{n}{Date}\PYG{o}{.}\PYG{n}{now}\PYG{p}{\PYGZcb{}}\PYG{p}{,}
    \PYG{n}{endDate}\PYG{p}{:} \PYG{p}{\PYGZob{}}\PYG{n+nb}{type}\PYG{p}{:} \PYG{n}{Date}\PYG{p}{\PYGZcb{}}\PYG{p}{,}
    \PYG{n}{url}\PYG{p}{:} \PYG{p}{\PYGZob{}}\PYG{n+nb}{type}\PYG{p}{:}\PYG{n}{Object}\PYG{p}{,} \PYG{n}{default}\PYG{p}{:}\PYG{p}{[}\PYG{p}{]}\PYG{p}{\PYGZcb{}}\PYG{p}{,}
    \PYG{n}{role}\PYG{p}{:} \PYG{p}{\PYGZob{}} \PYG{n+nb}{type}\PYG{p}{:} \PYG{n}{String}\PYG{p}{,} \PYG{n}{enum}\PYG{p}{:} \PYG{p}{[}\PYG{l+s+s1}{\PYGZsq{}}\PYG{l+s+s1}{SuperAdmin}\PYG{l+s+s1}{\PYGZsq{}}\PYG{p}{,} \PYG{l+s+s1}{\PYGZsq{}}\PYG{l+s+s1}{Admin}\PYG{l+s+s1}{\PYGZsq{}}\PYG{p}{,} \PYG{l+s+s1}{\PYGZsq{}}\PYG{l+s+s1}{User}\PYG{l+s+s1}{\PYGZsq{}}\PYG{p}{]}\PYG{p}{,} \PYG{n}{default}\PYG{p}{:} \PYG{l+s+s1}{\PYGZsq{}}\PYG{l+s+s1}{User}\PYG{l+s+s1}{\PYGZsq{}}\PYG{p}{\PYGZcb{}}\PYG{p}{,}
    \PYG{n}{color}\PYG{p}{:} \PYG{p}{\PYGZob{}}\PYG{n+nb}{type}\PYG{p}{:}\PYG{n}{String}\PYG{p}{,} \PYG{n}{default}\PYG{p}{:}\PYG{l+s+s2}{\PYGZdq{}}\PYG{l+s+s2}{\PYGZdq{}}\PYG{p}{\PYGZcb{}}\PYG{p}{,}
    \PYG{n}{logo}\PYG{p}{:} \PYG{p}{\PYGZob{}}\PYG{n+nb}{type}\PYG{p}{:}\PYG{n}{String}\PYG{p}{,} \PYG{n}{default}\PYG{p}{:}\PYG{l+s+s2}{\PYGZdq{}}\PYG{l+s+s2}{\PYGZdq{}}\PYG{p}{\PYGZcb{}}\PYG{p}{,}
    \PYG{n}{importance}\PYG{p}{:} \PYG{p}{\PYGZob{}}\PYG{n+nb}{type}\PYG{p}{:}\PYG{n}{String}\PYG{p}{,} \PYG{n}{default}\PYG{p}{:}\PYG{l+s+s2}{\PYGZdq{}}\PYG{l+s+s2}{\PYGZdq{}}\PYG{p}{\PYGZcb{}}\PYG{p}{,}
    \PYG{n}{receiver}\PYG{p}{:} \PYG{p}{\PYGZob{}}\PYG{n+nb}{type}\PYG{p}{:}\PYG{n}{Object}\PYG{p}{,} \PYG{n}{default}\PYG{p}{:}\PYG{p}{[}\PYG{p}{]}\PYG{p}{\PYGZcb{}}\PYG{p}{,}
    \PYG{n}{createdBy}\PYG{p}{:} \PYG{p}{\PYGZob{}} \PYG{n+nb}{type}\PYG{p}{:} \PYG{n}{Schema}\PYG{o}{.}\PYG{n}{Types}\PYG{o}{.}\PYG{n}{ObjectId}\PYG{p}{,} \PYG{n}{ref}\PYG{p}{:} \PYG{l+s+s2}{\PYGZdq{}}\PYG{l+s+s2}{Group}\PYG{l+s+s2}{\PYGZdq{}}\PYG{p}{\PYGZcb{}}
\PYG{p}{\PYGZcb{}}\PYG{p}{)}
\end{sphinxVerbatim}

\sphinxstylestrong{Bot}
This collection stores the information exchanged between the bot and a user when using the Healthbot direct question scenario.

\begin{sphinxVerbatim}[commandchars=\\\{\}]
\PYG{n}{const} \PYG{n}{BotSchema} \PYG{o}{=} \PYG{n}{Schema}\PYG{p}{(}\PYG{p}{\PYGZob{}}
    \PYG{n}{lang}\PYG{p}{:} \PYG{p}{\PYGZob{}}\PYG{n+nb}{type}\PYG{p}{:}\PYG{n}{String}\PYG{p}{,} \PYG{n}{default}\PYG{p}{:}\PYG{l+s+s2}{\PYGZdq{}}\PYG{l+s+s2}{\PYGZdq{}}\PYG{p}{\PYGZcb{}}\PYG{p}{,}
    \PYG{n}{data}\PYG{p}{:} \PYG{p}{\PYGZob{}}\PYG{n+nb}{type}\PYG{p}{:} \PYG{n}{Object}\PYG{p}{,} \PYG{n}{default}\PYG{p}{:} \PYG{p}{[}\PYG{p}{]}\PYG{p}{\PYGZcb{}}\PYG{p}{,}
    \PYG{n}{date}\PYG{p}{:} \PYG{p}{\PYGZob{}}\PYG{n+nb}{type}\PYG{p}{:} \PYG{n}{Date}\PYG{p}{,} \PYG{n}{default}\PYG{p}{:} \PYG{n}{Date}\PYG{o}{.}\PYG{n}{now}\PYG{p}{\PYGZcb{}}\PYG{p}{,}
    \PYG{n}{curatedBy}\PYG{p}{:} \PYG{p}{\PYGZob{}}\PYG{n+nb}{type}\PYG{p}{:}\PYG{n}{String}\PYG{p}{,} \PYG{n}{default}\PYG{p}{:}\PYG{l+s+s2}{\PYGZdq{}}\PYG{l+s+s2}{\PYGZdq{}}\PYG{p}{\PYGZcb{}}\PYG{p}{,}
    \PYG{n+nb}{type}\PYG{p}{:} \PYG{n}{String}\PYG{p}{,}
    \PYG{n}{createdBy}\PYG{p}{:} \PYG{p}{\PYGZob{}} \PYG{n+nb}{type}\PYG{p}{:} \PYG{n}{Schema}\PYG{o}{.}\PYG{n}{Types}\PYG{o}{.}\PYG{n}{ObjectId}\PYG{p}{,} \PYG{n}{ref}\PYG{p}{:} \PYG{l+s+s2}{\PYGZdq{}}\PYG{l+s+s2}{Group}\PYG{l+s+s2}{\PYGZdq{}}\PYG{p}{\PYGZcb{}}
\PYG{p}{\PYGZcb{}}\PYG{p}{)}
\end{sphinxVerbatim}

\sphinxstylestrong{Group}
In this collection, the patient groups available at Health29 are stored.

\begin{sphinxVerbatim}[commandchars=\\\{\}]
\PYG{n}{const} \PYG{n}{GroupSchema} \PYG{o}{=} \PYG{n}{Schema}\PYG{p}{(}\PYG{p}{\PYGZob{}}
	\PYG{n}{name}\PYG{p}{:} \PYG{p}{\PYGZob{}}
		\PYG{n+nb}{type}\PYG{p}{:} \PYG{n}{String}
  	\PYG{p}{\PYGZcb{}}\PYG{p}{,}
	\PYG{n}{subscription}\PYG{p}{:} \PYG{n}{String}\PYG{p}{,}
	\PYG{n}{email}\PYG{p}{:} \PYG{n}{String}\PYG{p}{,}
	\PYG{n}{defaultLang}\PYG{p}{:} \PYG{p}{\PYGZob{}}\PYG{n+nb}{type}\PYG{p}{:} \PYG{n}{String}\PYG{p}{,} \PYG{n}{default}\PYG{p}{:} \PYG{l+s+s1}{\PYGZsq{}}\PYG{l+s+s1}{en}\PYG{l+s+s1}{\PYGZsq{}}\PYG{p}{\PYGZcb{}}\PYG{p}{,}
	\PYG{n}{phenotype}\PYG{p}{:} \PYG{p}{\PYGZob{}}\PYG{n+nb}{type}\PYG{p}{:} \PYG{n}{Object}\PYG{p}{,} \PYG{n}{default}\PYG{p}{:} \PYG{p}{[}\PYG{p}{]}\PYG{p}{\PYGZcb{}}\PYG{p}{,}
	\PYG{n}{medications}\PYG{p}{:} \PYG{p}{\PYGZob{}}\PYG{n+nb}{type}\PYG{p}{:} \PYG{n}{Object}\PYG{p}{,} \PYG{n}{default}\PYG{p}{:} \PYG{p}{[}\PYG{p}{]}\PYG{p}{\PYGZcb{}}
\PYG{p}{\PYGZcb{}}\PYG{p}{)}
\end{sphinxVerbatim}

\sphinxstylestrong{Lang}
There will be one entry in this collection for each language you want to use on the Health29 platform

\begin{sphinxVerbatim}[commandchars=\\\{\}]
\PYG{n}{const} \PYG{n}{LangSchema} \PYG{o}{=} \PYG{n}{Schema}\PYG{p}{(}\PYG{p}{\PYGZob{}}
    \PYG{n}{name}\PYG{p}{:} \PYG{p}{\PYGZob{}}\PYG{n+nb}{type}\PYG{p}{:} \PYG{n}{String}\PYG{p}{,} \PYG{n}{unique}\PYG{p}{:} \PYG{n}{true}\PYG{p}{,} \PYG{n}{required}\PYG{p}{:} \PYG{n}{true} \PYG{p}{\PYGZcb{}}\PYG{p}{,}
    \PYG{n}{code}\PYG{p}{:} \PYG{p}{\PYGZob{}} \PYG{n+nb}{type}\PYG{p}{:} \PYG{n}{String}\PYG{p}{,}	\PYG{n}{index}\PYG{p}{:} \PYG{n}{true}\PYG{p}{,} \PYG{n}{unique}\PYG{p}{:} \PYG{n}{true}\PYG{p}{,} \PYG{n}{required}\PYG{p}{:} \PYG{n}{true} \PYG{p}{\PYGZcb{}}
\PYG{p}{\PYGZcb{}}\PYG{p}{,} \PYG{p}{\PYGZob{}} \PYG{n}{versionKey}\PYG{p}{:} \PYG{n}{false} \PYG{o}{/}\PYG{o}{/} \PYG{n}{You} \PYG{n}{should} \PYG{n}{be} \PYG{n}{aware} \PYG{n}{of} \PYG{n}{the} \PYG{n}{outcome} \PYG{n}{after} \PYG{n+nb}{set} \PYG{n}{to} \PYG{n}{false}
\PYG{p}{\PYGZcb{}}\PYG{p}{)}
\end{sphinxVerbatim}

\sphinxstylestrong{Patient}
In this collection the data of the patients registered in the platform are stored.

\begin{sphinxVerbatim}[commandchars=\\\{\}]
\PYG{n}{const} \PYG{n}{SiblingSchema} \PYG{o}{=} \PYG{n}{Schema}\PYG{p}{(}\PYG{p}{\PYGZob{}}
	\PYG{n}{gender}\PYG{p}{:} \PYG{n}{String}\PYG{p}{,}
	\PYG{n}{affected}\PYG{p}{:} \PYG{n}{String} \PYG{o}{/}\PYG{o}{/}\PYG{n}{affected}\PYG{p}{:} \PYG{p}{\PYGZob{}} \PYG{n+nb}{type}\PYG{p}{:} \PYG{n}{String}\PYG{p}{,} \PYG{n}{enum}\PYG{p}{:} \PYG{p}{[}\PYG{l+s+s1}{\PYGZsq{}}\PYG{l+s+s1}{yes}\PYG{l+s+s1}{\PYGZsq{}}\PYG{p}{,} \PYG{l+s+s1}{\PYGZsq{}}\PYG{l+s+s1}{no}\PYG{l+s+s1}{\PYGZsq{}}\PYG{p}{]}\PYG{p}{\PYGZcb{}} \PYG{n}{si} \PYG{n}{hacemos} \PYG{n}{validacion} \PYG{n}{de} \PYG{n}{que} \PYG{n}{no} \PYG{n}{pueda} \PYG{n}{ser} \PYG{n}{null}\PYG{p}{,} \PYG{n}{igual} \PYG{n}{poner} \PYG{n}{el} \PYG{n}{enum}
\PYG{p}{\PYGZcb{}}\PYG{p}{)}

\PYG{n}{const} \PYG{n}{ParentSchema} \PYG{o}{=} \PYG{n}{Schema}\PYG{p}{(}\PYG{p}{\PYGZob{}}
	\PYG{n}{highEducation}\PYG{p}{:} \PYG{n}{String}\PYG{p}{,}
	\PYG{n}{profession}\PYG{p}{:} \PYG{n}{String}\PYG{p}{,}
	\PYG{n}{relationship}\PYG{p}{:} \PYG{n}{String}\PYG{p}{,}
	\PYG{n}{nameCaregiver}\PYG{p}{:} \PYG{n}{String}
\PYG{p}{\PYGZcb{}}\PYG{p}{)}

\PYG{n}{const} \PYG{n}{PatientSchema} \PYG{o}{=} \PYG{n}{Schema}\PYG{p}{(}\PYG{p}{\PYGZob{}}
    \PYG{n}{patientName}\PYG{p}{:} \PYG{n}{String}\PYG{p}{,}
    \PYG{n}{surname}\PYG{p}{:} \PYG{n}{String}\PYG{p}{,}
    \PYG{n}{birthDate}\PYG{p}{:} \PYG{n}{Date}\PYG{p}{,}
    \PYG{n}{citybirth}\PYG{p}{:} \PYG{n}{String}\PYG{p}{,}
    \PYG{n}{provincebirth}\PYG{p}{:} \PYG{n}{String}\PYG{p}{,}
    \PYG{n}{countrybirth}\PYG{p}{:} \PYG{n}{String}\PYG{p}{,}
    \PYG{n}{street}\PYG{p}{:} \PYG{n}{String}\PYG{p}{,}
    \PYG{n}{postalCode}\PYG{p}{:} \PYG{n}{String}\PYG{p}{,}
    \PYG{n}{city}\PYG{p}{:} \PYG{n}{String}\PYG{p}{,}
    \PYG{n}{province}\PYG{p}{:} \PYG{n}{String}\PYG{p}{,}
    \PYG{n}{country}\PYG{p}{:} \PYG{n}{String}\PYG{p}{,}
    \PYG{n}{phone1}\PYG{p}{:} \PYG{n}{String}\PYG{p}{,}
    \PYG{n}{phone2}\PYG{p}{:} \PYG{n}{String}\PYG{p}{,}
    \PYG{n}{gender}\PYG{p}{:} \PYG{n}{String}\PYG{p}{,}
    \PYG{n}{siblings}\PYG{p}{:} \PYG{p}{[}\PYG{n}{SiblingSchema}\PYG{p}{]}\PYG{p}{,}
    \PYG{n}{parents}\PYG{p}{:} \PYG{p}{[}\PYG{n}{ParentSchema}\PYG{p}{]}\PYG{p}{,}
    \PYG{n}{createdBy}\PYG{p}{:} \PYG{p}{\PYGZob{}} \PYG{n+nb}{type}\PYG{p}{:} \PYG{n}{Schema}\PYG{o}{.}\PYG{n}{Types}\PYG{o}{.}\PYG{n}{ObjectId}\PYG{p}{,} \PYG{n}{ref}\PYG{p}{:} \PYG{l+s+s2}{\PYGZdq{}}\PYG{l+s+s2}{User}\PYG{l+s+s2}{\PYGZdq{}}\PYG{p}{\PYGZcb{}}\PYG{p}{,}
    \PYG{n}{death}\PYG{p}{:} \PYG{n}{Date}\PYG{p}{,}
    \PYG{n}{notes}\PYG{p}{:} \PYG{p}{\PYGZob{}}\PYG{n+nb}{type}\PYG{p}{:} \PYG{n}{String}\PYG{p}{,} \PYG{n}{default}\PYG{p}{:} \PYG{l+s+s1}{\PYGZsq{}}\PYG{l+s+s1}{\PYGZsq{}}\PYG{p}{\PYGZcb{}}\PYG{p}{,}
    \PYG{n}{answers}\PYG{p}{:} \PYG{p}{\PYGZob{}}\PYG{n+nb}{type}\PYG{p}{:} \PYG{n}{Object}\PYG{p}{,} \PYG{n}{default}\PYG{p}{:} \PYG{p}{[}\PYG{p}{]}\PYG{p}{\PYGZcb{}}\PYG{p}{,}
    \PYG{n}{subscriptionToGroupAlerts}\PYG{p}{:}\PYG{p}{\PYGZob{}}\PYG{n+nb}{type}\PYG{p}{:}\PYG{n}{Boolean}\PYG{p}{,}\PYG{n}{default}\PYG{p}{:}\PYG{n}{true}\PYG{p}{\PYGZcb{}}
\PYG{p}{\PYGZcb{}}\PYG{p}{)}
\end{sphinxVerbatim}

\sphinxstylestrong{Prom}
The datapoints associated with a group of patients are stored here.

\begin{sphinxVerbatim}[commandchars=\\\{\}]
\PYG{n}{const} \PYG{n}{PromSchema} \PYG{o}{=} \PYG{n}{Schema}\PYG{p}{(}\PYG{p}{\PYGZob{}}
	\PYG{n}{name}\PYG{p}{:} \PYG{n}{String}\PYG{p}{,}
	\PYG{n}{responseType}\PYG{p}{:} \PYG{n}{String}\PYG{p}{,}
	\PYG{n}{question}\PYG{p}{:} \PYG{n}{String}\PYG{p}{,}
	\PYG{n}{hideQuestion}\PYG{p}{:} \PYG{p}{\PYGZob{}}\PYG{n+nb}{type}\PYG{p}{:} \PYG{n}{Boolean}\PYG{p}{,} \PYG{n}{default}\PYG{p}{:} \PYG{n}{false}\PYG{p}{\PYGZcb{}}\PYG{p}{,}
	\PYG{n}{marginTop}\PYG{p}{:} \PYG{p}{\PYGZob{}}\PYG{n+nb}{type}\PYG{p}{:} \PYG{n}{Boolean}\PYG{p}{,} \PYG{n}{default}\PYG{p}{:} \PYG{n}{false}\PYG{p}{\PYGZcb{}}\PYG{p}{,}
	\PYG{n}{values}\PYG{p}{:} \PYG{n}{Array}\PYG{p}{,}
	\PYG{n}{section}\PYG{p}{:} \PYG{p}{\PYGZob{}} \PYG{n+nb}{type}\PYG{p}{:} \PYG{n}{Schema}\PYG{o}{.}\PYG{n}{Types}\PYG{o}{.}\PYG{n}{ObjectId}\PYG{p}{,} \PYG{n}{ref}\PYG{p}{:} \PYG{l+s+s2}{\PYGZdq{}}\PYG{l+s+s2}{PromSection}\PYG{l+s+s2}{\PYGZdq{}}\PYG{p}{\PYGZcb{}}\PYG{p}{,}
	\PYG{n}{order}\PYG{p}{:}\PYG{n}{Number}\PYG{p}{,}
	\PYG{n}{periodicity}\PYG{p}{:} \PYG{n}{Number}\PYG{p}{,}
	\PYG{n}{isRequired}\PYG{p}{:} \PYG{p}{\PYGZob{}}\PYG{n+nb}{type}\PYG{p}{:} \PYG{n}{Boolean}\PYG{p}{,} \PYG{n}{default}\PYG{p}{:} \PYG{n}{false}\PYG{p}{\PYGZcb{}}\PYG{p}{,}
	\PYG{n}{enabled}\PYG{p}{:} \PYG{p}{\PYGZob{}}\PYG{n+nb}{type}\PYG{p}{:} \PYG{n}{Boolean}\PYG{p}{,} \PYG{n}{default}\PYG{p}{:} \PYG{n}{false}\PYG{p}{\PYGZcb{}}\PYG{p}{,}
	\PYG{n}{createdBy}\PYG{p}{:} \PYG{p}{\PYGZob{}} \PYG{n+nb}{type}\PYG{p}{:} \PYG{n}{Schema}\PYG{o}{.}\PYG{n}{Types}\PYG{o}{.}\PYG{n}{ObjectId}\PYG{p}{,} \PYG{n}{ref}\PYG{p}{:} \PYG{l+s+s2}{\PYGZdq{}}\PYG{l+s+s2}{Group}\PYG{l+s+s2}{\PYGZdq{}}\PYG{p}{\PYGZcb{}}\PYG{p}{,}
	\PYG{n}{width}\PYG{p}{:} \PYG{n}{String}\PYG{p}{,}
	\PYG{n}{hpo}\PYG{p}{:} \PYG{n}{String}\PYG{p}{,}
	\PYG{n}{relatedTo}\PYG{p}{:} \PYG{p}{\PYGZob{}} \PYG{n+nb}{type}\PYG{p}{:} \PYG{n}{Schema}\PYG{o}{.}\PYG{n}{Types}\PYG{o}{.}\PYG{n}{ObjectId}\PYG{p}{,} \PYG{n}{ref}\PYG{p}{:} \PYG{l+s+s2}{\PYGZdq{}}\PYG{l+s+s2}{PromSchema}\PYG{l+s+s2}{\PYGZdq{}}\PYG{p}{\PYGZcb{}}\PYG{p}{,}
	\PYG{n}{disableDataPoints}\PYG{p}{:} \PYG{p}{\PYGZob{}} \PYG{n+nb}{type}\PYG{p}{:} \PYG{n}{Schema}\PYG{o}{.}\PYG{n}{Types}\PYG{o}{.}\PYG{n}{ObjectId}\PYG{p}{,} \PYG{n}{ref}\PYG{p}{:} \PYG{l+s+s2}{\PYGZdq{}}\PYG{l+s+s2}{PromSchema}\PYG{l+s+s2}{\PYGZdq{}}\PYG{p}{\PYGZcb{}}

\PYG{p}{\PYGZcb{}}\PYG{p}{)}
\end{sphinxVerbatim}

\sphinxstylestrong{PromSection}
Each of the above datapoints is encapsulated within a section. The different sections or groupings of datapoints that can appear on the platform for each group of patients are saved in this collection.

\begin{sphinxVerbatim}[commandchars=\\\{\}]
\PYG{n}{const} \PYG{n}{PromSectionSchema} \PYG{o}{=} \PYG{n}{Schema}\PYG{p}{(}\PYG{p}{\PYGZob{}}
    \PYG{n}{name}\PYG{p}{:} \PYG{n}{String}\PYG{p}{,}
    \PYG{n}{description}\PYG{p}{:} \PYG{n}{String}\PYG{p}{,}
    \PYG{n}{enabled}\PYG{p}{:} \PYG{p}{\PYGZob{}}\PYG{n+nb}{type}\PYG{p}{:} \PYG{n}{Boolean}\PYG{p}{,} \PYG{n}{default}\PYG{p}{:} \PYG{n}{true}\PYG{p}{\PYGZcb{}}\PYG{p}{,}
    \PYG{n}{order}\PYG{p}{:} \PYG{p}{\PYGZob{}}\PYG{n+nb}{type}\PYG{p}{:} \PYG{n}{Number}\PYG{p}{,} \PYG{n}{default}\PYG{p}{:} \PYG{l+m+mi}{0}\PYG{p}{\PYGZcb{}}\PYG{p}{,}
    \PYG{n}{createdBy}\PYG{p}{:} \PYG{p}{\PYGZob{}} \PYG{n+nb}{type}\PYG{p}{:} \PYG{n}{Schema}\PYG{o}{.}\PYG{n}{Types}\PYG{o}{.}\PYG{n}{ObjectId}\PYG{p}{,} \PYG{n}{ref}\PYG{p}{:} \PYG{l+s+s2}{\PYGZdq{}}\PYG{l+s+s2}{Group}\PYG{l+s+s2}{\PYGZdq{}}\PYG{p}{\PYGZcb{}}
\PYG{p}{\PYGZcb{}}\PYG{p}{)}
\end{sphinxVerbatim}

\sphinxstylestrong{Qna}
To access the information contained in qnaMaker, it is necessary to identify each of the databases created there, according to language and patient group. This collection contains this identifying information.

\begin{sphinxVerbatim}[commandchars=\\\{\}]
\PYG{n}{const} \PYG{n}{QnaSchema} \PYG{o}{=} \PYG{n}{Schema}\PYG{p}{(}\PYG{p}{\PYGZob{}}
	\PYG{n}{group}\PYG{p}{:} \PYG{p}{\PYGZob{}} \PYG{n+nb}{type}\PYG{p}{:} \PYG{n}{String}\PYG{p}{,} \PYG{n}{required}\PYG{p}{:} \PYG{n}{true}\PYG{p}{\PYGZcb{}}\PYG{p}{,}
	\PYG{n}{knowledgeBaseID}\PYG{p}{:} \PYG{p}{\PYGZob{}} \PYG{n+nb}{type}\PYG{p}{:} \PYG{n}{String}\PYG{p}{,} \PYG{n}{required}\PYG{p}{:} \PYG{n}{true}\PYG{p}{\PYGZcb{}}\PYG{p}{,}
	\PYG{n}{lang}\PYG{p}{:} \PYG{p}{\PYGZob{}} \PYG{n+nb}{type}\PYG{p}{:} \PYG{n}{String}\PYG{p}{,} \PYG{n}{required}\PYG{p}{:} \PYG{n}{true}\PYG{p}{\PYGZcb{}}\PYG{p}{,}
	\PYG{n}{categories}\PYG{p}{:}\PYG{p}{\PYGZob{}}\PYG{n+nb}{type}\PYG{p}{:}\PYG{n}{Object}\PYG{p}{,}\PYG{n}{default}\PYG{p}{:}\PYG{p}{[}\PYG{p}{]}\PYG{p}{\PYGZcb{}}
\PYG{p}{\PYGZcb{}}\PYG{p}{)}
\end{sphinxVerbatim}

\sphinxstylestrong{StructureProms}
Again this collection provides functionality to work with the different datapoints of a group of patients. It stores the structure of the datapoints for each patient group for each available language.
Highlight here the data field: it contains a lot of information, for example for the duchenne patient group it is 2,000 lines of a JSON. It has been done this way because we have a section to translate, and so we load everything in the language of the patient

\begin{sphinxVerbatim}[commandchars=\\\{\}]
\PYG{n}{const} \PYG{n}{StructurePromSchema} \PYG{o}{=} \PYG{n}{Schema}\PYG{p}{(}\PYG{p}{\PYGZob{}}
	\PYG{n}{data}\PYG{p}{:} \PYG{n}{Schema}\PYG{o}{.}\PYG{n}{Types}\PYG{o}{.}\PYG{n}{Mixed}\PYG{p}{,}
	\PYG{n}{lang}\PYG{p}{:} \PYG{p}{\PYGZob{}} \PYG{n+nb}{type}\PYG{p}{:} \PYG{n}{String}\PYG{p}{,} \PYG{n}{required}\PYG{p}{:} \PYG{n}{true}\PYG{p}{\PYGZcb{}}\PYG{p}{,}
	\PYG{n}{createdBy}\PYG{p}{:} \PYG{p}{\PYGZob{}} \PYG{n+nb}{type}\PYG{p}{:} \PYG{n}{Schema}\PYG{o}{.}\PYG{n}{Types}\PYG{o}{.}\PYG{n}{ObjectId}\PYG{p}{,} \PYG{n}{ref}\PYG{p}{:} \PYG{l+s+s2}{\PYGZdq{}}\PYG{l+s+s2}{Group}\PYG{l+s+s2}{\PYGZdq{}}\PYG{p}{\PYGZcb{}}
\PYG{p}{\PYGZcb{}}\PYG{p}{)}
\end{sphinxVerbatim}

\sphinxstylestrong{Support}
In this collection is stored the information related to the requests made by users to the support of the platform.

\begin{sphinxVerbatim}[commandchars=\\\{\}]
\PYG{n}{const} \PYG{n}{SupportSchema} \PYG{o}{=} \PYG{n}{Schema}\PYG{p}{(}\PYG{p}{\PYGZob{}}
	\PYG{n}{platform}\PYG{p}{:} \PYG{n}{String}\PYG{p}{,}
	\PYG{n}{subject}\PYG{p}{:} \PYG{n}{String}\PYG{p}{,}
	\PYG{n}{description}\PYG{p}{:} \PYG{n}{String}\PYG{p}{,}
	\PYG{n+nb}{type}\PYG{p}{:} \PYG{n}{String}\PYG{p}{,}
	\PYG{n}{status}\PYG{p}{:} \PYG{p}{\PYGZob{}}\PYG{n+nb}{type}\PYG{p}{:} \PYG{n}{String}\PYG{p}{,} \PYG{n}{default}\PYG{p}{:} \PYG{l+s+s1}{\PYGZsq{}}\PYG{l+s+s1}{unread}\PYG{l+s+s1}{\PYGZsq{}}\PYG{p}{\PYGZcb{}}\PYG{p}{,}
	\PYG{n}{statusDate}\PYG{p}{:} \PYG{p}{\PYGZob{}}\PYG{n+nb}{type}\PYG{p}{:} \PYG{n}{Date}\PYG{p}{,} \PYG{n}{default}\PYG{p}{:} \PYG{n}{Date}\PYG{o}{.}\PYG{n}{now}\PYG{p}{\PYGZcb{}}\PYG{p}{,}
	\PYG{n}{files}\PYG{p}{:} \PYG{n}{Object}\PYG{p}{,}
	\PYG{n}{date}\PYG{p}{:} \PYG{p}{\PYGZob{}}\PYG{n+nb}{type}\PYG{p}{:} \PYG{n}{Date}\PYG{p}{,} \PYG{n}{default}\PYG{p}{:} \PYG{n}{Date}\PYG{o}{.}\PYG{n}{now}\PYG{p}{\PYGZcb{}}\PYG{p}{,}
	\PYG{n}{createdBy}\PYG{p}{:} \PYG{p}{\PYGZob{}} \PYG{n+nb}{type}\PYG{p}{:} \PYG{n}{Schema}\PYG{o}{.}\PYG{n}{Types}\PYG{o}{.}\PYG{n}{ObjectId}\PYG{p}{,} \PYG{n}{ref}\PYG{p}{:} \PYG{l+s+s2}{\PYGZdq{}}\PYG{l+s+s2}{User}\PYG{l+s+s2}{\PYGZdq{}}\PYG{p}{\PYGZcb{}}
\PYG{p}{\PYGZcb{}}\PYG{p}{)}
\end{sphinxVerbatim}

\sphinxstylestrong{UserAlerts}
This collection is related to the “Alerts” collection. The content of this one will be generated from the previous one, according to the receiver and the launch date of the alert.
The information about the notifications and their status for each patient in the platform will be stored here.

\begin{sphinxVerbatim}[commandchars=\\\{\}]
\PYG{n}{const} \PYG{n}{UseralertsSchema} \PYG{o}{=} \PYG{n}{Schema}\PYG{p}{(}\PYG{p}{\PYGZob{}}
    \PYG{n}{alertId}\PYG{p}{:} \PYG{p}{\PYGZob{}}\PYG{n+nb}{type}\PYG{p}{:}\PYG{n}{String}\PYG{p}{,} \PYG{n}{default}\PYG{p}{:}\PYG{l+s+s2}{\PYGZdq{}}\PYG{l+s+s2}{\PYGZdq{}}\PYG{p}{\PYGZcb{}}\PYG{p}{,}
    \PYG{n}{patientId}\PYG{p}{:} \PYG{p}{\PYGZob{}}\PYG{n+nb}{type}\PYG{p}{:}\PYG{n}{String}\PYG{p}{,} \PYG{n}{default}\PYG{p}{:}\PYG{l+s+s2}{\PYGZdq{}}\PYG{l+s+s2}{\PYGZdq{}}\PYG{p}{\PYGZcb{}}\PYG{p}{,}
    \PYG{n}{state}\PYG{p}{:} \PYG{p}{\PYGZob{}}\PYG{n+nb}{type}\PYG{p}{:}\PYG{n}{String}\PYG{p}{,} \PYG{n}{default}\PYG{p}{:}\PYG{l+s+s2}{\PYGZdq{}}\PYG{l+s+s2}{Not read}\PYG{l+s+s2}{\PYGZdq{}}\PYG{p}{\PYGZcb{}}\PYG{p}{,}
    \PYG{n}{snooze}\PYG{p}{:} \PYG{p}{\PYGZob{}}\PYG{n+nb}{type}\PYG{p}{:}\PYG{n}{String}\PYG{p}{,} \PYG{n}{default}\PYG{p}{:}\PYG{l+s+s2}{\PYGZdq{}}\PYG{l+s+s2}{1}\PYG{l+s+s2}{\PYGZdq{}}\PYG{p}{\PYGZcb{}}\PYG{p}{,}
    \PYG{n}{showDate}\PYG{p}{:} \PYG{p}{\PYGZob{}}\PYG{n+nb}{type}\PYG{p}{:} \PYG{n}{Date}\PYG{p}{\PYGZcb{}}\PYG{p}{,}
    \PYG{n}{launch}\PYG{p}{:} \PYG{p}{\PYGZob{}}\PYG{n+nb}{type}\PYG{p}{:} \PYG{n}{Boolean}\PYG{p}{,} \PYG{n}{default}\PYG{p}{:} \PYG{n}{false}\PYG{p}{\PYGZcb{}}\PYG{p}{,}
    \PYG{n}{createdBy}\PYG{p}{:} \PYG{p}{\PYGZob{}} \PYG{n+nb}{type}\PYG{p}{:} \PYG{n}{Schema}\PYG{o}{.}\PYG{n}{Types}\PYG{o}{.}\PYG{n}{ObjectId}\PYG{p}{,} \PYG{n}{ref}\PYG{p}{:} \PYG{l+s+s2}{\PYGZdq{}}\PYG{l+s+s2}{Group}\PYG{l+s+s2}{\PYGZdq{}}\PYG{p}{\PYGZcb{}}
\PYG{p}{\PYGZcb{}}\PYG{p}{)}
\end{sphinxVerbatim}

\sphinxstylestrong{User}
Analogous to patient collection but with user data.

\begin{sphinxVerbatim}[commandchars=\\\{\}]
const SiblingSchema = Schema(\PYGZob{}
	gender: String,
	affected: \PYGZob{} type: String, enum: [\PYGZsq{}yes\PYGZsq{}, \PYGZsq{}no\PYGZsq{}]\PYGZcb{}
\PYGZcb{})

const ParentSchema = Schema(\PYGZob{}
	highEducation: String,
	profession: String
\PYGZcb{})

const UserSchema = Schema(\PYGZob{}
	email: \PYGZob{}
		type: String,
		index: true,
		trim: true,
		lowercase: true,
		unique: true,
		required: \PYGZsq{}Email address is required\PYGZsq{},
        match: [/\PYGZca{}\PYGZbs{}w+([\PYGZbs{}.\PYGZhy{}]?\PYGZbs{}w+)*@\PYGZbs{}w+([\PYGZbs{}.\PYGZhy{}]?\PYGZbs{}w+)*(\PYGZbs{}.\PYGZbs{}w\PYGZob{}2,3\PYGZcb{})+\PYGZdl{}/, \PYGZsq{}Please fill a valid email address\PYGZsq{}]
    \PYGZcb{},
	password: \PYGZob{} type: String, select: false, required: true, minlength: [8,\PYGZsq{}Password too short\PYGZsq{}]\PYGZcb{},
	role: \PYGZob{} type: String, required: true, enum: [\PYGZsq{}Researcher\PYGZsq{}, \PYGZsq{}Admin\PYGZsq{}, \PYGZsq{}User\PYGZsq{}], default: \PYGZsq{}User\PYGZsq{}\PYGZcb{},
	group: \PYGZob{} type: String, required: true, default: \PYGZsq{}None\PYGZsq{}\PYGZcb{},
	confirmed: \PYGZob{}type: Boolean, default: false\PYGZcb{},
	confirmationCode: String,
	signupDate: \PYGZob{}type: Date, default: Date.now\PYGZcb{},
	lastLogin: \PYGZob{}type: Date, default: null\PYGZcb{},
	userName: String,
	loginAttempts: \PYGZob{} type: Number, required: true, default: 0 \PYGZcb{},
  	lockUntil: \PYGZob{} type: Number \PYGZcb{},
	lang: \PYGZob{} type: String, required: true, default: \PYGZsq{}en\PYGZsq{}\PYGZcb{},
	randomCodeRecoverPass: String,
	dateTimeRecoverPass: Date,
	massunit: \PYGZob{} type: String, required: true, default: \PYGZsq{}kg\PYGZsq{}\PYGZcb{},
	lengthunit: \PYGZob{} type: String, required: true, default: \PYGZsq{}cm\PYGZsq{}\PYGZcb{},
	blockedaccount: \PYGZob{}type: Boolean, default: false\PYGZcb{},
	permissions: \PYGZob{}type: Object, default: \PYGZob{}\PYGZcb{}\PYGZcb{},
	modules: \PYGZob{}type: Object, default: []\PYGZcb{},
	platform: \PYGZob{}type: String, default: \PYGZsq{}\PYGZsq{}\PYGZcb{},
	authyId:\PYGZob{}type:String,default:null\PYGZcb{},
	authyDeviceId:\PYGZob{}type:Object,default:[]\PYGZcb{},
	phone: String,
	subgroup: Number,
\PYGZcb{})
\end{sphinxVerbatim}


\subsection{2.4.6.3. DDBB: Data}
\label{\detokenize{pages/SW/Code:ddbb-data}}
In all the collections of this database, the information entered by the patient in the Health29 platform will be saved. Thus, for each of the sections that appear in it, the information relating to it will be saved in the corresponding collection of this database.

ClinicalTrials

\begin{sphinxVerbatim}[commandchars=\\\{\}]
\PYG{n}{const} \PYG{n}{ClinicalTrialSchema} \PYG{o}{=} \PYG{n}{Schema}\PYG{p}{(}\PYG{p}{\PYGZob{}}
	\PYG{n}{nameClinicalTrial}\PYG{p}{:} \PYG{n}{String}\PYG{p}{,}
	\PYG{n}{takingClinicalTrial}\PYG{p}{:} \PYG{n}{String}\PYG{p}{,}
	\PYG{n}{drugName}\PYG{p}{:} \PYG{n}{String}\PYG{p}{,}
	\PYG{n}{center}\PYG{p}{:} \PYG{n}{String}\PYG{p}{,}
	\PYG{n}{date}\PYG{p}{:} \PYG{n}{Date}\PYG{p}{,}
	\PYG{n}{createdBy}\PYG{p}{:} \PYG{p}{\PYGZob{}} \PYG{n+nb}{type}\PYG{p}{:} \PYG{n}{Schema}\PYG{o}{.}\PYG{n}{Types}\PYG{o}{.}\PYG{n}{ObjectId}\PYG{p}{,} \PYG{n}{ref}\PYG{p}{:} \PYG{l+s+s2}{\PYGZdq{}}\PYG{l+s+s2}{Patient}\PYG{l+s+s2}{\PYGZdq{}}\PYG{p}{\PYGZcb{}}
\PYG{p}{\PYGZcb{}}\PYG{p}{)}
\end{sphinxVerbatim}

Genotype

\begin{sphinxVerbatim}[commandchars=\\\{\}]
\PYG{n}{const} \PYG{n}{GenotypeSchema} \PYG{o}{=} \PYG{n}{Schema}\PYG{p}{(}\PYG{p}{\PYGZob{}}
	\PYG{n}{inputType}\PYG{p}{:} \PYG{p}{\PYGZob{}}\PYG{n+nb}{type}\PYG{p}{:} \PYG{n}{String}\PYG{p}{,} \PYG{n}{default}\PYG{p}{:} \PYG{n}{null}\PYG{p}{\PYGZcb{}}\PYG{p}{,}
	\PYG{n}{date}\PYG{p}{:} \PYG{p}{\PYGZob{}}\PYG{n+nb}{type}\PYG{p}{:} \PYG{n}{Date}\PYG{p}{,} \PYG{n}{default}\PYG{p}{:} \PYG{n}{Date}\PYG{o}{.}\PYG{n}{now}\PYG{p}{\PYGZcb{}}\PYG{p}{,}
	\PYG{n}{data}\PYG{p}{:} \PYG{n}{Object}\PYG{p}{,}
	\PYG{n}{createdBy}\PYG{p}{:} \PYG{p}{\PYGZob{}} \PYG{n+nb}{type}\PYG{p}{:} \PYG{n}{Schema}\PYG{o}{.}\PYG{n}{Types}\PYG{o}{.}\PYG{n}{ObjectId}\PYG{p}{,} \PYG{n}{ref}\PYG{p}{:} \PYG{l+s+s2}{\PYGZdq{}}\PYG{l+s+s2}{Patient}\PYG{l+s+s2}{\PYGZdq{}}\PYG{p}{\PYGZcb{}}
\PYG{p}{\PYGZcb{}}\PYG{p}{)}
\end{sphinxVerbatim}

HeightHistory

\begin{sphinxVerbatim}[commandchars=\\\{\}]
\PYG{n}{const} \PYG{n}{HeightHistorySchema} \PYG{o}{=} \PYG{n}{Schema}\PYG{p}{(}\PYG{p}{\PYGZob{}}
	\PYG{n}{dateTime}\PYG{p}{:} \PYG{p}{\PYGZob{}}\PYG{n+nb}{type}\PYG{p}{:} \PYG{n}{Date}\PYG{p}{,} \PYG{n}{default}\PYG{p}{:} \PYG{n}{Date}\PYG{o}{.}\PYG{n}{now}\PYG{p}{\PYGZcb{}}\PYG{p}{,}
	\PYG{n}{value}\PYG{p}{:} \PYG{n}{String}\PYG{p}{,}
	\PYG{n}{technique}\PYG{p}{:} \PYG{n}{String}\PYG{p}{,}
	\PYG{n}{createdBy}\PYG{p}{:} \PYG{p}{\PYGZob{}} \PYG{n+nb}{type}\PYG{p}{:} \PYG{n}{Schema}\PYG{o}{.}\PYG{n}{Types}\PYG{o}{.}\PYG{n}{ObjectId}\PYG{p}{,} \PYG{n}{ref}\PYG{p}{:} \PYG{l+s+s2}{\PYGZdq{}}\PYG{l+s+s2}{Patient}\PYG{l+s+s2}{\PYGZdq{}}\PYG{p}{\PYGZcb{}}
\PYG{p}{\PYGZcb{}}\PYG{p}{,} \PYG{p}{\PYGZob{}}
    \PYG{n}{versionKey}\PYG{p}{:} \PYG{n}{false} \PYG{o}{/}\PYG{o}{/} \PYG{n}{You} \PYG{n}{should} \PYG{n}{be} \PYG{n}{aware} \PYG{n}{of} \PYG{n}{the} \PYG{n}{outcome} \PYG{n}{after} \PYG{n+nb}{set} \PYG{n}{to} \PYG{n}{false}
\PYG{p}{\PYGZcb{}}\PYG{p}{)}
\end{sphinxVerbatim}

Height

\begin{sphinxVerbatim}[commandchars=\\\{\}]
\PYG{n}{const} \PYG{n}{HeightSchema} \PYG{o}{=} \PYG{n}{Schema}\PYG{p}{(}\PYG{p}{\PYGZob{}}
	\PYG{n}{dateTime}\PYG{p}{:} \PYG{p}{\PYGZob{}}\PYG{n+nb}{type}\PYG{p}{:} \PYG{n}{Date}\PYG{p}{,} \PYG{n}{default}\PYG{p}{:} \PYG{n}{Date}\PYG{o}{.}\PYG{n}{now}\PYG{p}{\PYGZcb{}}\PYG{p}{,}
	\PYG{n}{value}\PYG{p}{:} \PYG{n}{String}\PYG{p}{,}
	\PYG{n}{technique}\PYG{p}{:} \PYG{n}{String}\PYG{p}{,}
	\PYG{n}{createdBy}\PYG{p}{:} \PYG{p}{\PYGZob{}} \PYG{n+nb}{type}\PYG{p}{:} \PYG{n}{Schema}\PYG{o}{.}\PYG{n}{Types}\PYG{o}{.}\PYG{n}{ObjectId}\PYG{p}{,} \PYG{n}{ref}\PYG{p}{:} \PYG{l+s+s2}{\PYGZdq{}}\PYG{l+s+s2}{Patient}\PYG{l+s+s2}{\PYGZdq{}}\PYG{p}{\PYGZcb{}}
\PYG{p}{\PYGZcb{}}\PYG{p}{,} \PYG{p}{\PYGZob{}}
    \PYG{n}{versionKey}\PYG{p}{:} \PYG{n}{false} \PYG{o}{/}\PYG{o}{/} \PYG{n}{You} \PYG{n}{should} \PYG{n}{be} \PYG{n}{aware} \PYG{n}{of} \PYG{n}{the} \PYG{n}{outcome} \PYG{n}{after} \PYG{n+nb}{set} \PYG{n}{to} \PYG{n}{false}
\PYG{p}{\PYGZcb{}}\PYG{p}{)}
\end{sphinxVerbatim}

Medicalcare

\begin{sphinxVerbatim}[commandchars=\\\{\}]
\PYG{n}{const} \PYG{n}{MedicalCareSchema} \PYG{o}{=} \PYG{n}{Schema}\PYG{p}{(}\PYG{p}{\PYGZob{}}
	\PYG{n}{date}\PYG{p}{:} \PYG{n}{Date}\PYG{p}{,}
	\PYG{n}{data}\PYG{p}{:} \PYG{n}{Object}\PYG{p}{,}
	\PYG{n}{createdBy}\PYG{p}{:} \PYG{p}{\PYGZob{}} \PYG{n+nb}{type}\PYG{p}{:} \PYG{n}{Schema}\PYG{o}{.}\PYG{n}{Types}\PYG{o}{.}\PYG{n}{ObjectId}\PYG{p}{,} \PYG{n}{ref}\PYG{p}{:} \PYG{l+s+s2}{\PYGZdq{}}\PYG{l+s+s2}{Patient}\PYG{l+s+s2}{\PYGZdq{}}\PYG{p}{\PYGZcb{}}
\PYG{p}{\PYGZcb{}}\PYG{p}{)}
\end{sphinxVerbatim}

Medication

\begin{sphinxVerbatim}[commandchars=\\\{\}]
\PYG{n}{const} \PYG{n}{MedicationSchema} \PYG{o}{=} \PYG{n}{Schema}\PYG{p}{(}\PYG{p}{\PYGZob{}}
    \PYG{n}{drug}\PYG{p}{:} \PYG{n}{Object}\PYG{p}{,}
    \PYG{n}{dose}\PYG{p}{:} \PYG{n}{String}\PYG{p}{,}
    \PYG{n}{startDate}\PYG{p}{:} \PYG{p}{\PYGZob{}}\PYG{n+nb}{type}\PYG{p}{:} \PYG{n}{Date}\PYG{p}{,} \PYG{n}{default}\PYG{p}{:} \PYG{n}{Date}\PYG{o}{.}\PYG{n}{now}\PYG{p}{\PYGZcb{}}\PYG{p}{,}
    \PYG{n}{endDate}\PYG{p}{:} \PYG{p}{\PYGZob{}}\PYG{n+nb}{type}\PYG{p}{:} \PYG{n}{Date}\PYG{p}{,} \PYG{n}{default}\PYG{p}{:} \PYG{n}{null}\PYG{p}{\PYGZcb{}}\PYG{p}{,}
    \PYG{n}{sideEffects}\PYG{p}{:} \PYG{p}{\PYGZob{}}\PYG{n+nb}{type}\PYG{p}{:} \PYG{n}{Object}\PYG{p}{,} \PYG{n}{default}\PYG{p}{:} \PYG{n}{null}\PYG{p}{\PYGZcb{}}\PYG{p}{,}
    \PYG{n}{schedule}\PYG{p}{:} \PYG{n}{String}\PYG{p}{,}
    \PYG{n}{otherSchedule}\PYG{p}{:} \PYG{n}{String}\PYG{p}{,}
    \PYG{n}{adverseEffects}\PYG{p}{:} \PYG{n}{Object}\PYG{p}{,}
    \PYG{n}{compassionateUse}\PYG{p}{:} \PYG{p}{\PYGZob{}}\PYG{n+nb}{type}\PYG{p}{:} \PYG{n}{String}\PYG{p}{,} \PYG{n}{default}\PYG{p}{:} \PYG{l+s+s1}{\PYGZsq{}}\PYG{l+s+s1}{\PYGZsq{}}\PYG{p}{\PYGZcb{}}\PYG{p}{,}
    \PYG{n}{vaccinations}\PYG{p}{:} \PYG{n}{String}\PYG{p}{,}
    \PYG{n}{vaccinationDate}\PYG{p}{:} \PYG{p}{\PYGZob{}}\PYG{n+nb}{type}\PYG{p}{:} \PYG{n}{Date}\PYG{p}{,} \PYG{n}{default}\PYG{p}{:} \PYG{n}{Date}\PYG{o}{.}\PYG{n}{now}\PYG{p}{\PYGZcb{}}\PYG{p}{,}
    \PYG{n}{notes}\PYG{p}{:} \PYG{p}{\PYGZob{}}\PYG{n+nb}{type}\PYG{p}{:} \PYG{n}{String}\PYG{p}{,} \PYG{n}{default}\PYG{p}{:} \PYG{l+s+s1}{\PYGZsq{}}\PYG{l+s+s1}{\PYGZsq{}}\PYG{p}{\PYGZcb{}}\PYG{p}{,}
    \PYG{n}{freesideEffects}\PYG{p}{:} \PYG{p}{\PYGZob{}}\PYG{n+nb}{type}\PYG{p}{:} \PYG{n}{String}\PYG{p}{,} \PYG{n}{default}\PYG{p}{:} \PYG{l+s+s1}{\PYGZsq{}}\PYG{l+s+s1}{\PYGZsq{}}\PYG{p}{\PYGZcb{}}\PYG{p}{,}
    \PYG{n}{createdBy}\PYG{p}{:} \PYG{p}{\PYGZob{}} \PYG{n+nb}{type}\PYG{p}{:} \PYG{n}{Schema}\PYG{o}{.}\PYG{n}{Types}\PYG{o}{.}\PYG{n}{ObjectId}\PYG{p}{,} \PYG{n}{ref}\PYG{p}{:} \PYG{l+s+s2}{\PYGZdq{}}\PYG{l+s+s2}{Patient}\PYG{l+s+s2}{\PYGZdq{}}\PYG{p}{\PYGZcb{}}
\PYG{p}{\PYGZcb{}}\PYG{p}{)}
\end{sphinxVerbatim}

Othermedications

\begin{sphinxVerbatim}[commandchars=\\\{\}]
\PYG{n}{const} \PYG{n}{OtherMedicationSchema} \PYG{o}{=} \PYG{n}{Schema}\PYG{p}{(}\PYG{p}{\PYGZob{}}
    \PYG{n}{name}\PYG{p}{:} \PYG{n}{String}\PYG{p}{,}
    \PYG{n}{dose}\PYG{p}{:} \PYG{n}{String}\PYG{p}{,}
    \PYG{n}{startDate}\PYG{p}{:} \PYG{p}{\PYGZob{}}\PYG{n+nb}{type}\PYG{p}{:} \PYG{n}{Date}\PYG{p}{,} \PYG{n}{default}\PYG{p}{:} \PYG{n}{Date}\PYG{o}{.}\PYG{n}{now}\PYG{p}{\PYGZcb{}}\PYG{p}{,}
    \PYG{n}{endDate}\PYG{p}{:} \PYG{p}{\PYGZob{}}\PYG{n+nb}{type}\PYG{p}{:} \PYG{n}{Date}\PYG{p}{,} \PYG{n}{default}\PYG{p}{:} \PYG{n}{null}\PYG{p}{\PYGZcb{}}\PYG{p}{,}
    \PYG{n+nb}{type}\PYG{p}{:} \PYG{n}{String}\PYG{p}{,}
    \PYG{n}{compassionateUse}\PYG{p}{:} \PYG{p}{\PYGZob{}}\PYG{n+nb}{type}\PYG{p}{:} \PYG{n}{String}\PYG{p}{,} \PYG{n}{default}\PYG{p}{:} \PYG{l+s+s1}{\PYGZsq{}}\PYG{l+s+s1}{\PYGZsq{}}\PYG{p}{\PYGZcb{}}\PYG{p}{,}
    \PYG{n}{freesideEffects}\PYG{p}{:} \PYG{p}{\PYGZob{}}\PYG{n+nb}{type}\PYG{p}{:} \PYG{n}{String}\PYG{p}{,} \PYG{n}{default}\PYG{p}{:} \PYG{l+s+s1}{\PYGZsq{}}\PYG{l+s+s1}{\PYGZsq{}}\PYG{p}{\PYGZcb{}}\PYG{p}{,}
    \PYG{n}{schedule}\PYG{p}{:} \PYG{n}{String}\PYG{p}{,}
    \PYG{n}{otherSchedule}\PYG{p}{:} \PYG{n}{String}\PYG{p}{,}
    \PYG{n}{notes}\PYG{p}{:} \PYG{n}{String}\PYG{p}{,}
    \PYG{n}{createdBy}\PYG{p}{:} \PYG{p}{\PYGZob{}} \PYG{n+nb}{type}\PYG{p}{:} \PYG{n}{Schema}\PYG{o}{.}\PYG{n}{Types}\PYG{o}{.}\PYG{n}{ObjectId}\PYG{p}{,} \PYG{n}{ref}\PYG{p}{:} \PYG{l+s+s2}{\PYGZdq{}}\PYG{l+s+s2}{Patient}\PYG{l+s+s2}{\PYGZdq{}}\PYG{p}{\PYGZcb{}}
\PYG{p}{\PYGZcb{}}\PYG{p}{)}
\end{sphinxVerbatim}

Patient\sphinxhyphen{}proms

\begin{sphinxVerbatim}[commandchars=\\\{\}]
\PYG{n}{const} \PYG{n}{PatientPromSchema} \PYG{o}{=} \PYG{n}{Schema}\PYG{p}{(}\PYG{p}{\PYGZob{}}
	\PYG{n}{data}\PYG{p}{:} \PYG{n}{Schema}\PYG{o}{.}\PYG{n}{Types}\PYG{o}{.}\PYG{n}{Mixed}\PYG{p}{,}
	\PYG{n}{date}\PYG{p}{:} \PYG{p}{\PYGZob{}}\PYG{n+nb}{type}\PYG{p}{:} \PYG{n}{Date}\PYG{p}{,} \PYG{n}{default}\PYG{p}{:} \PYG{n}{Date}\PYG{o}{.}\PYG{n}{now}\PYG{p}{\PYGZcb{}}\PYG{p}{,}
	\PYG{n}{definitionPromId}\PYG{p}{:} \PYG{p}{\PYGZob{}} \PYG{n+nb}{type}\PYG{p}{:} \PYG{n}{Schema}\PYG{o}{.}\PYG{n}{Types}\PYG{o}{.}\PYG{n}{ObjectId}\PYG{p}{,} \PYG{n}{ref}\PYG{p}{:} \PYG{l+s+s2}{\PYGZdq{}}\PYG{l+s+s2}{PromSchema}\PYG{l+s+s2}{\PYGZdq{}}\PYG{p}{\PYGZcb{}}\PYG{p}{,}
	\PYG{n}{createdBy}\PYG{p}{:} \PYG{p}{\PYGZob{}} \PYG{n+nb}{type}\PYG{p}{:} \PYG{n}{Schema}\PYG{o}{.}\PYG{n}{Types}\PYG{o}{.}\PYG{n}{ObjectId}\PYG{p}{,} \PYG{n}{ref}\PYG{p}{:} \PYG{l+s+s2}{\PYGZdq{}}\PYG{l+s+s2}{Patient}\PYG{l+s+s2}{\PYGZdq{}}\PYG{p}{\PYGZcb{}}
\PYG{p}{\PYGZcb{}}\PYG{p}{)}
\end{sphinxVerbatim}

PhenotypeHistory

\begin{sphinxVerbatim}[commandchars=\\\{\}]
\PYG{n}{const} \PYG{n}{PhenotypeHistorySchema} \PYG{o}{=} \PYG{n}{Schema}\PYG{p}{(}\PYG{p}{\PYGZob{}}
	\PYG{n}{validator\PYGZus{}id}\PYG{p}{:} \PYG{p}{\PYGZob{}}\PYG{n+nb}{type}\PYG{p}{:} \PYG{n}{String}\PYG{p}{,} \PYG{n}{default}\PYG{p}{:} \PYG{n}{null}\PYG{p}{\PYGZcb{}}\PYG{p}{,}
	\PYG{n}{validated}\PYG{p}{:} \PYG{p}{\PYGZob{}}\PYG{n+nb}{type}\PYG{p}{:} \PYG{n}{Boolean}\PYG{p}{,} \PYG{n}{default}\PYG{p}{:} \PYG{n}{false}\PYG{p}{\PYGZcb{}}\PYG{p}{,}
	\PYG{n}{inputType}\PYG{p}{:} \PYG{p}{\PYGZob{}}\PYG{n+nb}{type}\PYG{p}{:} \PYG{n}{String}\PYG{p}{,} \PYG{n}{default}\PYG{p}{:} \PYG{n}{null}\PYG{p}{\PYGZcb{}}\PYG{p}{,}
	\PYG{n}{date}\PYG{p}{:} \PYG{p}{\PYGZob{}}\PYG{n+nb}{type}\PYG{p}{:} \PYG{n}{Date}\PYG{p}{,} \PYG{n}{default}\PYG{p}{:} \PYG{n}{Date}\PYG{o}{.}\PYG{n}{now}\PYG{p}{\PYGZcb{}}\PYG{p}{,}
	\PYG{n}{data}\PYG{p}{:} \PYG{n}{Object}\PYG{p}{,}
	\PYG{n}{createdBy}\PYG{p}{:} \PYG{p}{\PYGZob{}} \PYG{n+nb}{type}\PYG{p}{:} \PYG{n}{Schema}\PYG{o}{.}\PYG{n}{Types}\PYG{o}{.}\PYG{n}{ObjectId}\PYG{p}{,} \PYG{n}{ref}\PYG{p}{:} \PYG{l+s+s2}{\PYGZdq{}}\PYG{l+s+s2}{Patient}\PYG{l+s+s2}{\PYGZdq{}}\PYG{p}{\PYGZcb{}}
\PYG{p}{\PYGZcb{}}\PYG{p}{)}
\end{sphinxVerbatim}

Phenotype

\begin{sphinxVerbatim}[commandchars=\\\{\}]
\PYG{n}{const} \PYG{n}{PhenotypeSchema} \PYG{o}{=} \PYG{n}{Schema}\PYG{p}{(}\PYG{p}{\PYGZob{}}
	\PYG{n}{validator\PYGZus{}id}\PYG{p}{:} \PYG{p}{\PYGZob{}}\PYG{n+nb}{type}\PYG{p}{:} \PYG{n}{String}\PYG{p}{,} \PYG{n}{default}\PYG{p}{:} \PYG{n}{null}\PYG{p}{\PYGZcb{}}\PYG{p}{,}
	\PYG{n}{validated}\PYG{p}{:} \PYG{p}{\PYGZob{}}\PYG{n+nb}{type}\PYG{p}{:} \PYG{n}{Boolean}\PYG{p}{,} \PYG{n}{default}\PYG{p}{:} \PYG{n}{false}\PYG{p}{\PYGZcb{}}\PYG{p}{,}
	\PYG{n}{inputType}\PYG{p}{:} \PYG{p}{\PYGZob{}}\PYG{n+nb}{type}\PYG{p}{:} \PYG{n}{String}\PYG{p}{,} \PYG{n}{default}\PYG{p}{:} \PYG{n}{null}\PYG{p}{\PYGZcb{}}\PYG{p}{,}
	\PYG{n}{date}\PYG{p}{:} \PYG{p}{\PYGZob{}}\PYG{n+nb}{type}\PYG{p}{:} \PYG{n}{Date}\PYG{p}{,} \PYG{n}{default}\PYG{p}{:} \PYG{n}{Date}\PYG{o}{.}\PYG{n}{now}\PYG{p}{\PYGZcb{}}\PYG{p}{,}
	\PYG{n}{data}\PYG{p}{:} \PYG{n}{Object}\PYG{p}{,}
	\PYG{n}{createdBy}\PYG{p}{:} \PYG{p}{\PYGZob{}} \PYG{n+nb}{type}\PYG{p}{:} \PYG{n}{Schema}\PYG{o}{.}\PYG{n}{Types}\PYG{o}{.}\PYG{n}{ObjectId}\PYG{p}{,} \PYG{n}{ref}\PYG{p}{:} \PYG{l+s+s2}{\PYGZdq{}}\PYG{l+s+s2}{Patient}\PYG{l+s+s2}{\PYGZdq{}}\PYG{p}{\PYGZcb{}}
\PYG{p}{\PYGZcb{}}\PYG{p}{)}
\end{sphinxVerbatim}

Seizures

\begin{sphinxVerbatim}[commandchars=\\\{\}]
\PYG{n}{const} \PYG{n}{SeizuresSchema} \PYG{o}{=} \PYG{n}{Schema}\PYG{p}{(}\PYG{p}{\PYGZob{}}
	\PYG{n}{disparadores}\PYG{p}{:} \PYG{p}{\PYGZob{}}\PYG{n+nb}{type}\PYG{p}{:} \PYG{n}{Object}\PYG{p}{,} \PYG{n}{default}\PYG{p}{:} \PYG{p}{[}\PYG{p}{]}\PYG{p}{\PYGZcb{}}\PYG{p}{,}
	\PYG{n}{disparadorEnfermo}\PYG{p}{:} \PYG{p}{\PYGZob{}}\PYG{n+nb}{type}\PYG{p}{:} \PYG{n}{String}\PYG{p}{,} \PYG{n}{default}\PYG{p}{:} \PYG{l+s+s1}{\PYGZsq{}}\PYG{l+s+s1}{\PYGZsq{}}\PYG{p}{\PYGZcb{}}\PYG{p}{,}
	\PYG{n}{disparadorOtro}\PYG{p}{:} \PYG{p}{\PYGZob{}}\PYG{n+nb}{type}\PYG{p}{:} \PYG{n}{String}\PYG{p}{,} \PYG{n}{default}\PYG{p}{:} \PYG{l+s+s1}{\PYGZsq{}}\PYG{l+s+s1}{\PYGZsq{}}\PYG{p}{\PYGZcb{}}\PYG{p}{,}
	\PYG{n}{disparadorNotas}\PYG{p}{:} \PYG{p}{\PYGZob{}}\PYG{n+nb}{type}\PYG{p}{:} \PYG{n}{String}\PYG{p}{,} \PYG{n}{default}\PYG{p}{:} \PYG{l+s+s1}{\PYGZsq{}}\PYG{l+s+s1}{\PYGZsq{}}\PYG{p}{\PYGZcb{}}\PYG{p}{,}
	\PYG{n}{descripcion}\PYG{p}{:} \PYG{p}{\PYGZob{}}\PYG{n+nb}{type}\PYG{p}{:} \PYG{n}{Object}\PYG{p}{,} \PYG{n}{default}\PYG{p}{:} \PYG{p}{[}\PYG{p}{]}\PYG{p}{\PYGZcb{}}\PYG{p}{,}
	\PYG{n}{descripcionRigidez}\PYG{p}{:} \PYG{p}{\PYGZob{}}\PYG{n+nb}{type}\PYG{p}{:} \PYG{n}{String}\PYG{p}{,} \PYG{n}{default}\PYG{p}{:} \PYG{l+s+s1}{\PYGZsq{}}\PYG{l+s+s1}{\PYGZsq{}}\PYG{p}{\PYGZcb{}}\PYG{p}{,}
	\PYG{n}{descripcionContraccion}\PYG{p}{:} \PYG{p}{\PYGZob{}}\PYG{n+nb}{type}\PYG{p}{:} \PYG{n}{String}\PYG{p}{,} \PYG{n}{default}\PYG{p}{:} \PYG{l+s+s1}{\PYGZsq{}}\PYG{l+s+s1}{\PYGZsq{}}\PYG{p}{\PYGZcb{}}\PYG{p}{,}
	\PYG{n}{descripcionOtro}\PYG{p}{:} \PYG{p}{\PYGZob{}}\PYG{n+nb}{type}\PYG{p}{:} \PYG{n}{String}\PYG{p}{,} \PYG{n}{default}\PYG{p}{:} \PYG{l+s+s1}{\PYGZsq{}}\PYG{l+s+s1}{\PYGZsq{}}\PYG{p}{\PYGZcb{}}\PYG{p}{,}
	\PYG{n}{descipcionNotas}\PYG{p}{:} \PYG{p}{\PYGZob{}}\PYG{n+nb}{type}\PYG{p}{:} \PYG{n}{String}\PYG{p}{,} \PYG{n}{default}\PYG{p}{:} \PYG{l+s+s1}{\PYGZsq{}}\PYG{l+s+s1}{\PYGZsq{}}\PYG{p}{\PYGZcb{}}\PYG{p}{,}
	\PYG{n}{postCrisis}\PYG{p}{:} \PYG{p}{\PYGZob{}}\PYG{n+nb}{type}\PYG{p}{:} \PYG{n}{Object}\PYG{p}{,} \PYG{n}{default}\PYG{p}{:} \PYG{p}{[}\PYG{p}{]}\PYG{p}{\PYGZcb{}}\PYG{p}{,}
	\PYG{n}{postCrisisOtro}\PYG{p}{:} \PYG{p}{\PYGZob{}}\PYG{n+nb}{type}\PYG{p}{:} \PYG{n}{String}\PYG{p}{,} \PYG{n}{default}\PYG{p}{:} \PYG{l+s+s1}{\PYGZsq{}}\PYG{l+s+s1}{\PYGZsq{}}\PYG{p}{\PYGZcb{}}\PYG{p}{,}
	\PYG{n}{postCrisisNotas}\PYG{p}{:} \PYG{p}{\PYGZob{}}\PYG{n+nb}{type}\PYG{p}{:} \PYG{n}{String}\PYG{p}{,} \PYG{n}{default}\PYG{p}{:} \PYG{l+s+s1}{\PYGZsq{}}\PYG{l+s+s1}{\PYGZsq{}}\PYG{p}{\PYGZcb{}}\PYG{p}{,}
	\PYG{n}{estadoAnimo}\PYG{p}{:} \PYG{p}{\PYGZob{}}\PYG{n+nb}{type}\PYG{p}{:} \PYG{n}{String}\PYG{p}{,} \PYG{n}{default}\PYG{p}{:} \PYG{l+s+s1}{\PYGZsq{}}\PYG{l+s+s1}{\PYGZsq{}}\PYG{p}{\PYGZcb{}}\PYG{p}{,}
	\PYG{n}{estadoConsciencia}\PYG{p}{:} \PYG{p}{\PYGZob{}}\PYG{n+nb}{type}\PYG{p}{:} \PYG{n}{String}\PYG{p}{,} \PYG{n}{default}\PYG{p}{:} \PYG{l+s+s1}{\PYGZsq{}}\PYG{l+s+s1}{\PYGZsq{}}\PYG{p}{\PYGZcb{}}\PYG{p}{,}
	\PYG{n}{duracion}\PYG{p}{:} \PYG{p}{\PYGZob{}}\PYG{n+nb}{type}\PYG{p}{:} \PYG{n}{Object}\PYG{p}{,} \PYG{n}{default}\PYG{p}{:} \PYG{p}{\PYGZob{}}\PYG{n}{hours}\PYG{p}{:} \PYG{l+m+mi}{0}\PYG{p}{,} \PYG{n}{minutes}\PYG{p}{:} \PYG{l+m+mi}{0}\PYG{p}{,} \PYG{n}{seconds}\PYG{p}{:}\PYG{l+m+mi}{0}\PYG{p}{\PYGZcb{}}\PYG{p}{\PYGZcb{}}\PYG{p}{,}
	\PYG{n+nb}{type}\PYG{p}{:} \PYG{p}{\PYGZob{}}\PYG{n+nb}{type}\PYG{p}{:} \PYG{n}{String}\PYG{p}{,} \PYG{n}{default}\PYG{p}{:} \PYG{n}{null}\PYG{p}{\PYGZcb{}}\PYG{p}{,}
	\PYG{n}{start}\PYG{p}{:} \PYG{p}{\PYGZob{}}\PYG{n+nb}{type}\PYG{p}{:} \PYG{n}{Date}\PYG{p}{,} \PYG{n}{default}\PYG{p}{:} \PYG{n}{null}\PYG{p}{\PYGZcb{}}\PYG{p}{,}
	\PYG{n}{end}\PYG{p}{:} \PYG{p}{\PYGZob{}}\PYG{n+nb}{type}\PYG{p}{:} \PYG{n}{Date}\PYG{p}{,} \PYG{n}{default}\PYG{p}{:} \PYG{n}{null}\PYG{p}{\PYGZcb{}}\PYG{p}{,}
	\PYG{n}{GUID}\PYG{p}{:} \PYG{p}{\PYGZob{}}\PYG{n+nb}{type}\PYG{p}{:} \PYG{n}{String}\PYG{p}{,} \PYG{n}{default}\PYG{p}{:} \PYG{l+s+s1}{\PYGZsq{}}\PYG{l+s+s1}{\PYGZsq{}}\PYG{p}{\PYGZcb{}}\PYG{p}{,}
	\PYG{n}{title}\PYG{p}{:} \PYG{p}{\PYGZob{}}\PYG{n+nb}{type}\PYG{p}{:} \PYG{n}{String}\PYG{p}{,} \PYG{n}{default}\PYG{p}{:} \PYG{l+s+s1}{\PYGZsq{}}\PYG{l+s+s1}{\PYGZsq{}}\PYG{p}{\PYGZcb{}}\PYG{p}{,}
	\PYG{n}{color}\PYG{p}{:} \PYG{p}{\PYGZob{}}\PYG{n+nb}{type}\PYG{p}{:} \PYG{n}{Object}\PYG{p}{,} \PYG{n}{default}\PYG{p}{:} \PYG{p}{\PYGZob{}}\PYG{p}{\PYGZcb{}}\PYG{p}{\PYGZcb{}}\PYG{p}{,}
	\PYG{n}{actions}\PYG{p}{:} \PYG{p}{\PYGZob{}}\PYG{n+nb}{type}\PYG{p}{:} \PYG{n}{Object}\PYG{p}{,} \PYG{n}{default}\PYG{p}{:} \PYG{p}{[}\PYG{p}{]}\PYG{p}{\PYGZcb{}}\PYG{p}{,}
	\PYG{n}{createdBy}\PYG{p}{:} \PYG{p}{\PYGZob{}} \PYG{n+nb}{type}\PYG{p}{:} \PYG{n}{Schema}\PYG{o}{.}\PYG{n}{Types}\PYG{o}{.}\PYG{n}{ObjectId}\PYG{p}{,} \PYG{n}{ref}\PYG{p}{:} \PYG{l+s+s2}{\PYGZdq{}}\PYG{l+s+s2}{Patient}\PYG{l+s+s2}{\PYGZdq{}}\PYG{p}{\PYGZcb{}}
\PYG{p}{\PYGZcb{}}\PYG{p}{)}
\end{sphinxVerbatim}

Social info

\begin{sphinxVerbatim}[commandchars=\\\{\}]
\PYG{n}{const} \PYG{n}{SocialInfoSchema} \PYG{o}{=} \PYG{n}{Schema}\PYG{p}{(}\PYG{p}{\PYGZob{}}
	\PYG{n}{education}\PYG{p}{:} \PYG{n}{String}\PYG{p}{,}
	\PYG{n}{completedEducation}\PYG{p}{:} \PYG{n}{String}\PYG{p}{,}
	\PYG{n}{currentEducation}\PYG{p}{:} \PYG{n}{String}\PYG{p}{,}
	\PYG{n}{work}\PYG{p}{:} \PYG{n}{String}\PYG{p}{,}
	\PYG{n}{hoursWork}\PYG{p}{:} \PYG{n}{String}\PYG{p}{,}
	\PYG{n}{profession}\PYG{p}{:} \PYG{n}{String}\PYG{p}{,}
	\PYG{n}{livingSituation}\PYG{p}{:} \PYG{n}{Array}\PYG{p}{,}
	\PYG{n}{support}\PYG{p}{:} \PYG{n}{Array}\PYG{p}{,}
	\PYG{n}{sports}\PYG{p}{:} \PYG{n}{Array}\PYG{p}{,}
	\PYG{n}{interests}\PYG{p}{:} \PYG{n}{Array}\PYG{p}{,}
	\PYG{n}{moreInterests}\PYG{p}{:} \PYG{n}{String}\PYG{p}{,}
	\PYG{n}{createdBy}\PYG{p}{:} \PYG{p}{\PYGZob{}} \PYG{n+nb}{type}\PYG{p}{:} \PYG{n}{Schema}\PYG{o}{.}\PYG{n}{Types}\PYG{o}{.}\PYG{n}{ObjectId}\PYG{p}{,} \PYG{n}{ref}\PYG{p}{:} \PYG{l+s+s2}{\PYGZdq{}}\PYG{l+s+s2}{Patient}\PYG{l+s+s2}{\PYGZdq{}}\PYG{p}{\PYGZcb{}}
\PYG{p}{\PYGZcb{}}\PYG{p}{)}
\end{sphinxVerbatim}

Vaccinations

\begin{sphinxVerbatim}[commandchars=\\\{\}]
\PYG{n}{const} \PYG{n}{VaccinationSchema} \PYG{o}{=} \PYG{n}{Schema}\PYG{p}{(}\PYG{p}{\PYGZob{}}
	\PYG{n}{name}\PYG{p}{:} \PYG{n}{String}\PYG{p}{,}
	\PYG{n}{date}\PYG{p}{:} \PYG{n}{Date}\PYG{p}{,}
	\PYG{n}{createdBy}\PYG{p}{:} \PYG{p}{\PYGZob{}} \PYG{n+nb}{type}\PYG{p}{:} \PYG{n}{Schema}\PYG{o}{.}\PYG{n}{Types}\PYG{o}{.}\PYG{n}{ObjectId}\PYG{p}{,} \PYG{n}{ref}\PYG{p}{:} \PYG{l+s+s2}{\PYGZdq{}}\PYG{l+s+s2}{Patient}\PYG{l+s+s2}{\PYGZdq{}}\PYG{p}{\PYGZcb{}}
\PYG{p}{\PYGZcb{}}\PYG{p}{)}
\end{sphinxVerbatim}

WeightHistory

\begin{sphinxVerbatim}[commandchars=\\\{\}]
\PYG{n}{const} \PYG{n}{WeightHistorySchema} \PYG{o}{=} \PYG{n}{Schema}\PYG{p}{(}\PYG{p}{\PYGZob{}}
	\PYG{n}{dateTime}\PYG{p}{:} \PYG{p}{\PYGZob{}}\PYG{n+nb}{type}\PYG{p}{:} \PYG{n}{Date}\PYG{p}{,} \PYG{n}{default}\PYG{p}{:} \PYG{n}{Date}\PYG{o}{.}\PYG{n}{now}\PYG{p}{\PYGZcb{}}\PYG{p}{,}
	\PYG{n}{value}\PYG{p}{:} \PYG{n}{String}\PYG{p}{,}
	\PYG{n}{createdBy}\PYG{p}{:} \PYG{p}{\PYGZob{}} \PYG{n+nb}{type}\PYG{p}{:} \PYG{n}{Schema}\PYG{o}{.}\PYG{n}{Types}\PYG{o}{.}\PYG{n}{ObjectId}\PYG{p}{,} \PYG{n}{ref}\PYG{p}{:} \PYG{l+s+s2}{\PYGZdq{}}\PYG{l+s+s2}{Patient}\PYG{l+s+s2}{\PYGZdq{}}\PYG{p}{\PYGZcb{}}
\PYG{p}{\PYGZcb{}}\PYG{p}{,} \PYG{p}{\PYGZob{}}
    \PYG{n}{versionKey}\PYG{p}{:} \PYG{n}{false} \PYG{o}{/}\PYG{o}{/} \PYG{n}{You} \PYG{n}{should} \PYG{n}{be} \PYG{n}{aware} \PYG{n}{of} \PYG{n}{the} \PYG{n}{outcome} \PYG{n}{after} \PYG{n+nb}{set} \PYG{n}{to} \PYG{n}{false}
\PYG{p}{\PYGZcb{}}\PYG{p}{)}
\end{sphinxVerbatim}

Weight

\begin{sphinxVerbatim}[commandchars=\\\{\}]
\PYG{n}{const} \PYG{n}{WeightSchema} \PYG{o}{=} \PYG{n}{Schema}\PYG{p}{(}\PYG{p}{\PYGZob{}}
	\PYG{n}{dateTime}\PYG{p}{:} \PYG{p}{\PYGZob{}}\PYG{n+nb}{type}\PYG{p}{:} \PYG{n}{Date}\PYG{p}{,} \PYG{n}{default}\PYG{p}{:} \PYG{n}{Date}\PYG{o}{.}\PYG{n}{now}\PYG{p}{\PYGZcb{}}\PYG{p}{,}
	\PYG{n}{value}\PYG{p}{:} \PYG{n}{String}\PYG{p}{,}
	\PYG{n}{createdBy}\PYG{p}{:} \PYG{p}{\PYGZob{}} \PYG{n+nb}{type}\PYG{p}{:} \PYG{n}{Schema}\PYG{o}{.}\PYG{n}{Types}\PYG{o}{.}\PYG{n}{ObjectId}\PYG{p}{,} \PYG{n}{ref}\PYG{p}{:} \PYG{l+s+s2}{\PYGZdq{}}\PYG{l+s+s2}{Patient}\PYG{l+s+s2}{\PYGZdq{}}\PYG{p}{\PYGZcb{}}
\PYG{p}{\PYGZcb{}}\PYG{p}{,} \PYG{p}{\PYGZob{}}
    \PYG{n}{versionKey}\PYG{p}{:} \PYG{n}{false} \PYG{o}{/}\PYG{o}{/} \PYG{n}{You} \PYG{n}{should} \PYG{n}{be} \PYG{n}{aware} \PYG{n}{of} \PYG{n}{the} \PYG{n}{outcome} \PYG{n}{after} \PYG{n+nb}{set} \PYG{n}{to} \PYG{n}{false}
\PYG{p}{\PYGZcb{}}\PYG{p}{)}
\end{sphinxVerbatim}


\subsection{2.4.6.4. Collections relationship}
\label{\detokenize{pages/SW/Code:collections-relationship}}
The following diagram shows a summary of the relationships between the different database collections for each environment:




\section{2.4.7. Multilanguage}
\label{\detokenize{pages/SW/Code:multilanguage}}
The Health29 platform has the option of being displayed in several languages for the user. In particular, it is implemented for English, Spanish and Dutch.

To achieve this different implementations will be made in the client and the server of the webapp.


\subsection{2.4.7.1. Multilanguage in client}
\label{\detokenize{pages/SW/Code:multilanguage-in-client}}
On the one hand, the client uses the \sphinxstylestrong{\sphinxhref{https://www.npmjs.com/package/@ngx-translate/core}{ngx\sphinxhyphen{}translate} library}.
For that, version 9.0.2 of the core has been installed in the project:

\begin{sphinxVerbatim}[commandchars=\\\{\}]
\PYG{n}{npm} \PYG{n}{install} \PYG{n+nd}{@ngx}\PYG{o}{\PYGZhy{}}\PYG{n}{translate}\PYG{o}{/}\PYG{n}{core}\PYG{o}{@}\PYG{o}{\PYGZca{}}\PYG{l+m+mf}{9.0}\PYG{o}{.}\PYG{l+m+mi}{2} \PYG{o}{\PYGZhy{}}\PYG{o}{\PYGZhy{}}\PYG{n}{save}
\end{sphinxVerbatim}

And, in addition, version 2.0.1 of the http\sphinxhyphen{}loader is used:

\begin{sphinxVerbatim}[commandchars=\\\{\}]
\PYG{n}{npm} \PYG{n}{install} \PYG{n+nd}{@ngx}\PYG{o}{\PYGZhy{}}\PYG{n}{translate}\PYG{o}{/}\PYG{n}{http}\PYG{o}{\PYGZhy{}}\PYG{n}{loader} \PYG{o}{\PYGZhy{}}\PYG{o}{\PYGZhy{}}\PYG{n}{save} 
\end{sphinxVerbatim}

To use it, you should configure app.module.ts:

\begin{sphinxVerbatim}[commandchars=\\\{\}]
\PYG{k+kn}{import} \PYG{p}{\PYGZob{}} \PYG{n}{TranslateModule}\PYG{p}{,} \PYG{n}{TranslateLoader} \PYG{p}{\PYGZcb{}} \PYG{k+kn}{from} \PYG{l+s+s1}{\PYGZsq{}}\PYG{l+s+s1}{@ngx\PYGZhy{}translate/core}\PYG{l+s+s1}{\PYGZsq{}}\PYG{p}{;}
\PYG{k+kn}{import} \PYG{p}{\PYGZob{}} \PYG{n}{TranslateHttpLoader} \PYG{p}{\PYGZcb{}} \PYG{k+kn}{from} \PYG{l+s+s1}{\PYGZsq{}}\PYG{l+s+s1}{@ngx\PYGZhy{}translate/http\PYGZhy{}loader}\PYG{l+s+s1}{\PYGZsq{}}\PYG{p}{;}
\end{sphinxVerbatim}

\begin{sphinxVerbatim}[commandchars=\\\{\}]
\PYG{n}{export} \PYG{n}{function} \PYG{n}{createTranslateLoader}\PYG{p}{(}\PYG{n}{http}\PYG{p}{:} \PYG{n}{HttpClient}\PYG{p}{)} \PYG{p}{\PYGZob{}}
    \PYG{k}{return} \PYG{n}{new} \PYG{n}{TranslateHttpLoader}\PYG{p}{(}\PYG{n}{http}\PYG{p}{,} \PYG{n}{environment}\PYG{o}{.}\PYG{n}{api}\PYG{o}{+}\PYG{l+s+s1}{\PYGZsq{}}\PYG{l+s+s1}{/assets/i18n/}\PYG{l+s+s1}{\PYGZsq{}}\PYG{p}{,} \PYG{l+s+s1}{\PYGZsq{}}\PYG{l+s+s1}{.json}\PYG{l+s+s1}{\PYGZsq{}}\PYG{p}{)}\PYG{p}{;}
\PYG{p}{\PYGZcb{}}
\end{sphinxVerbatim}

\begin{sphinxVerbatim}[commandchars=\\\{\}]
\PYG{n+nd}{@NgModule}\PYG{p}{(}\PYG{p}{\PYGZob{}}
   \PYG{p}{[}\PYG{o}{.}\PYG{o}{.}\PYG{o}{.}\PYG{p}{]}
   \PYG{n}{imports}\PYG{p}{:} \PYG{p}{[}
         \PYG{o}{.}\PYG{o}{.}\PYG{o}{.}
       \PYG{n}{TranslateModule}\PYG{o}{.}\PYG{n}{forRoot}\PYG{p}{(}\PYG{p}{\PYGZob{}}
               \PYG{n}{loader}\PYG{p}{:} \PYG{p}{\PYGZob{}}
                   \PYG{n}{provide}\PYG{p}{:} \PYG{n}{TranslateLoader}\PYG{p}{,}
                   \PYG{n}{useFactory}\PYG{p}{:} \PYG{p}{(}\PYG{n}{createTranslateLoader}\PYG{p}{)}\PYG{p}{,}
                   \PYG{n}{deps}\PYG{p}{:} \PYG{p}{[}\PYG{n}{HttpClient}\PYG{p}{]}
                 \PYG{p}{\PYGZcb{}}
           \PYG{p}{\PYGZcb{}}\PYG{p}{)}\PYG{p}{,}
         \PYG{o}{.}\PYG{o}{.}\PYG{o}{.}
   \PYG{p}{]}
   \PYG{p}{[}\PYG{o}{.}\PYG{o}{.}\PYG{o}{.}\PYG{p}{]}
\PYG{p}{\PYGZcb{}}\PYG{p}{)}
\end{sphinxVerbatim}

Furthermore, it is included in shared.module.ts:

\begin{sphinxVerbatim}[commandchars=\\\{\}]
\PYG{k+kn}{import} \PYG{p}{\PYGZob{}} \PYG{n}{TranslateModule} \PYG{p}{\PYGZcb{}} \PYG{k+kn}{from} \PYG{l+s+s1}{\PYGZsq{}}\PYG{l+s+s1}{@ngx\PYGZhy{}translate/core}\PYG{l+s+s1}{\PYGZsq{}}\PYG{p}{;}
\PYG{p}{[}\PYG{o}{.}\PYG{o}{.}\PYG{o}{.}\PYG{p}{]}
\PYG{n+nd}{@NgModule}\PYG{p}{(}\PYG{p}{\PYGZob{}}
    \PYG{n}{exports}\PYG{p}{:} \PYG{p}{[}
        \PYG{o}{.}\PYG{o}{.}\PYG{o}{.}
        \PYG{n}{TranslateModule}\PYG{p}{,}
        \PYG{o}{.}\PYG{o}{.}\PYG{o}{.}
    \PYG{p}{]}\PYG{p}{,}
    \PYG{n}{imports}\PYG{p}{:} \PYG{p}{[}
        \PYG{o}{.}\PYG{o}{.}\PYG{o}{.}
        \PYG{n}{TranslateModule}

    \PYG{p}{]}\PYG{p}{,}
    \PYG{p}{[}\PYG{o}{.}\PYG{o}{.}\PYG{o}{.}\PYG{p}{]}
\PYG{p}{\PYGZcb{}}\PYG{p}{)}
\PYG{p}{[}\PYG{o}{.}\PYG{o}{.}\PYG{o}{.}\PYG{p}{]}
\end{sphinxVerbatim}

And it will also be included in the module file of each component that is going to use this service in the same way, and in the .ts file of the component the translation service will be included:

\begin{sphinxVerbatim}[commandchars=\\\{\}]
\PYG{p}{[}\PYG{o}{.}\PYG{o}{.}\PYG{o}{.}\PYG{p}{]}
\PYG{k+kn}{import} \PYG{p}{\PYGZob{}} \PYG{n}{TranslateService} \PYG{p}{\PYGZcb{}} \PYG{k+kn}{from} \PYG{l+s+s1}{\PYGZsq{}}\PYG{l+s+s1}{@ngx\PYGZhy{}translate/core}\PYG{l+s+s1}{\PYGZsq{}}\PYG{p}{;}
\PYG{p}{[}\PYG{o}{.}\PYG{o}{.}\PYG{o}{.}\PYG{p}{]}
\PYG{n}{constructor}\PYG{p}{(}\PYG{o}{.}\PYG{o}{.}\PYG{o}{.}\PYG{p}{,}\PYG{n}{public} \PYG{n}{translate}\PYG{p}{:} \PYG{n}{TranslateService}\PYG{p}{,}\PYG{o}{.}\PYG{o}{.}\PYG{o}{.}\PYG{p}{)}\PYG{p}{\PYGZob{}}\PYG{o}{.}\PYG{o}{.}\PYG{o}{.}\PYG{p}{\PYGZcb{}}
\end{sphinxVerbatim}

\begin{sphinxVerbatim}[commandchars=\\\{\}]
\PYG{n}{this}\PYG{o}{.}\PYG{n}{translate}\PYG{o}{.}\PYG{n}{instant}\PYG{p}{(}\PYG{l+s+s2}{\PYGZdq{}}\PYG{l+s+s2}{JSON\PYGZus{}key1.JSON\PYGZus{}key2}\PYG{l+s+s2}{\PYGZdq{}}\PYG{p}{)}
\end{sphinxVerbatim}

Finally, the service lang.service.ts has been created to implement the load function of the different languages available in the platform: “loadDataJson(lang:string)”

\begin{sphinxVerbatim}[commandchars=\\\{\}]
\PYG{n}{this}\PYG{o}{.}\PYG{n}{http}\PYG{o}{.}\PYG{n}{get}\PYG{p}{(}\PYG{n}{environment}\PYG{o}{.}\PYG{n}{api}\PYG{o}{+}\PYG{l+s+s1}{\PYGZsq{}}\PYG{l+s+s1}{/assets/i18n/}\PYG{l+s+s1}{\PYGZsq{}}\PYG{o}{+}\PYG{n}{lang}\PYG{o}{+}\PYG{l+s+s1}{\PYGZsq{}}\PYG{l+s+s1}{.json}\PYG{l+s+s1}{\PYGZsq{}}\PYG{p}{)}
\end{sphinxVerbatim}

This function will be in charge of obtaining the information of the different JSON files for each language, located in scr/assets/i18n:



This function will be called when the user changes the language of the platform.

On the other hand, it has already been commented throughout this document that an Azure translation service is used for certain functions: \sphinxstylestrong{the cognitive service Translator text}. This has been explained in section 2.4.3.3. of this document at code level and in section 2.3.3.3. at component level.


\subsection{2.4.7.2. Multilanguage in server}
\label{\detokenize{pages/SW/Code:multilanguage-in-server}}
On the server, in the superadmin language controller (controllers/superadmin/langs.js) you can manage new languages.

On the other hand, \sphinxstylestrong{different views} have been implemented (one for each language in which we want to use the health29 platform) which will be selected according to the language information provided by the customer in their requests.


\section{2.4.8. Second factor autenthication}
\label{\detokenize{pages/SW/Code:second-factor-autenthication}}
The 2FA was incorporated in Health29 for the login on the platform of certain groups of patients.

Initially, an analysis was made of the different options that could be used for implementation, concluding in Authy.
Authy has the best combination of compatibility, usability, security and reliability. platforms including iOS, Android, Windows, Mac and Chrome, and has PIN and biometric protection for the application. Authy includes a secure cloud backup option, which makes it easy to use on multiple devices and makes your tokens easy to restore if you lose or replace your phone.

If you lose your phone, you lose access to your authentication application. To solve this problem, most authentication applications offer cloud\sphinxhyphen{}based backups (although security experts tend to recommend against using this feature), and some authentication application manufacturers are better at explaining how (or if) they encrypt these backups. Authy is the only application we tested that offers two security features that assist in account recovery: an encrypted cloud backup and support for a secondary device.

You can also install Authy on a secondary device, such as a computer or tablet, and use that device along with backups to recover your account in case you lose your phone. Authy calls this feature “multi\sphinxhyphen{}device”. Once you add the second device, Authy recommends that you disable the feature so that another person cannot add another device to take control of your account (Authy will continue to work on both devices). With backups and multi\sphinxhyphen{}device enabled, your tokens are synchronized across all devices where Authy is installed.

You can block the Authy application behind a PIN or biometric identification, such as a fingerprint or facial scan. If your phone is already blocked in this way (and it should be), this additional step is not necessary, but it’s a good touch if you want to use a different PIN for added security.

In particular, the Authy service we are using is accessed from \sphinxhref{https://www.twilio.com/login}{the Twilio console}. The project was created: 2FA Project and the application Health29\sphinxhyphen{}Duchenne. From this portal it is possible to configure the screen to be shown to the users, the options to be worked with, and also to manage the current users.

For the implementation the version 1.4.0 of \sphinxhref{https://www.npmjs.com/package/authy}{authy} has been used and is added to the code:

\begin{sphinxVerbatim}[commandchars=\\\{\}]
\PYG{n}{const} \PYG{n}{authy} \PYG{o}{=} \PYG{n}{require}\PYG{p}{(}\PYG{l+s+s1}{\PYGZsq{}}\PYG{l+s+s1}{authy}\PYG{l+s+s1}{\PYGZsq{}}\PYG{p}{)}\PYG{p}{(}\PYG{n}{APIKEY}\PYG{p}{)}\PYG{p}{;}
\end{sphinxVerbatim}

As mentioned above, the user’s model and controller have been modified:
\begin{itemize}
\item {} 
In the user model, the fields were added:

\end{itemize}

\begin{sphinxVerbatim}[commandchars=\\\{\}]
  \PYG{n}{authyId}\PYG{p}{:}\PYG{p}{\PYGZob{}}\PYG{n+nb}{type}\PYG{p}{:}\PYG{n}{String}\PYG{p}{,}\PYG{n}{default}\PYG{p}{:}\PYG{n}{null}\PYG{p}{\PYGZcb{}}\PYG{p}{,}
	\PYG{n}{authyDeviceId}\PYG{p}{:}\PYG{p}{\PYGZob{}}\PYG{n+nb}{type}\PYG{p}{:}\PYG{n}{Object}\PYG{p}{,}\PYG{n}{default}\PYG{p}{:}\PYG{p}{[}\PYG{p}{]}\PYG{p}{\PYGZcb{}}\PYG{p}{,}
	\PYG{n}{phone}\PYG{p}{:} \PYG{n}{String}\PYG{p}{,}
\end{sphinxVerbatim}

To store the information on each user required for interaction with Authy.
\begin{itemize}
\item {} 
In the user’s controller, the registration and user login functions were updated, and a new one was created to update those that existed before the addition of the functionality.
In all the functions the group of patients to which the user belongs is checked in order to apply or not the authentication conditions.

\end{itemize}

For these tasks the Authy functions will be used: register\_user and signin2FA.

In addition to this we offer a method for migrating users prior to the introduction of the second factor for a patient group, i.e. any user already registered and created on the platform who belongs to a patient group that did not have 2FA and then added will be prompted at login to enter their phone number and will be registered in Authy using registerUserInAuthy.




\chapter{3. Superadmin profile}
\label{\detokenize{pages/Superadmin profile:superadmin-profile}}\label{\detokenize{pages/Superadmin profile::doc}}
A super\sphinxhyphen{}administrator profile for the platform has been implemented. With this profile, some aspects of the platform can be managed visually or through a graphic environment so as not to depend on programmers directly.

With this profile it is possible to control and manage the needs that arise in the platform to incorporate new elements or to modify an existing one, without requiring programming. Therefore, when implementing new functions it is important to evaluate if what is being added will be configurable or modifiable, and in that case it is important to define the functionality in the most generic way possible and to add the control of this in the super\sphinxhyphen{}administrator role.

This profile will have control over all patient groups that have been incorporated into Healht29.

In order to better understand what has been explained here, the current list of actions that can be performed from this profile on the platform will be presented in this section. With this, we can already get an idea of the usefulness of this profile and the possibilities within the tool.

Next, this section will be subdivided into different sections to develop in depth each of the functionalities available for this profile in Health29.

These functionalities will be present in the navigation menu of the left panel of the platform for the superadmin role.

The implementation of the code for the functionalities of this profile is done in the same way as explained in section 2.4 of this document.


\section{3.1. Manage groups}
\label{\detokenize{pages/Superadmin profile:manage-groups}}
The first menu option of this profile is: “Groups”.



This page shows a table with the information:
\begin{itemize}
\item {} 
Name of the patient group included in Health29

\item {} 
Patient Group Manager

\item {} 
Type of subscription.

\end{itemize}

Here, you can change any of the above fields or delete groups.


\section{3.2. Manage symptoms}
\label{\detokenize{pages/Superadmin profile:manage-symptoms}}
This page manages the symptoms of each patient group.



By selecting a particular patient group, the associated symptoms are shown:



You can add new or remove existing ones.


\section{3.3. Manage languages}
\label{\detokenize{pages/Superadmin profile:manage-languages}}
The list of languages in which the platform is available is shown and allows you to add new




\section{3.4. Manage Frequently asked questions}
\label{\detokenize{pages/Superadmin profile:manage-frequently-asked-questions}}
On this page you can manage the platform’s frequently asked questions, visible to users from the FAQs section or those that the healthbot works with.
This page allows you to interact directly with the knowledge bases created in Qna maker, so that you can add, delete or update frequently asked questions.



As already explained in the architecture section of this document, each group of patients will have a knowledge base for each available language. So, first you have to select the patient group you want to modify or add this to:



There are several actions that can be performed on each available FAQ collection:
\begin{itemize}
\item {} 
Publish changes. For each update or modification made on the FAQs of a collection it will be necessary to publish the changes so that this is visible to the users of the patient group.

\item {} 
Delete. To delete a complete collection.

\end{itemize}

In addition, each question\sphinxhyphen{}answer pair is encapsulated within a category.Thus, multiple question\sphinxhyphen{}answer  pairs  can  belong  to  one  category.In  addition,  each  question\sphinxhyphen{}answer pair can belong to several categories.Thus, the general structure of a FAQ would consist of:
\begin{itemize}
\item {} 
A set of questions

\item {} 
Only oneanswer

\item {} 
One or more categories

\end{itemize}

Each collection of FAQs will have a list of available categories that can be expanded. These categories can beviewed by clicking on the “Categories” button.
Then, a popup will then be shown with the list of available categories and from here you can delete these categories by pressing the “X” button that appears next to each one of them.

Within a collection you can perform the following actions for the FAQs it contains:create a new FAQ, edit an existing FAQand delete a FAQ.But in any case, all these changes will not be visible to users until the changes to the collection are published.We will now explain how to perform each of these actions.


\subsection{3.4.1. Add new FAQ}
\label{\detokenize{pages/Superadmin profile:add-new-faq}}
Click on “New frequently asked question” button and a new page will appear:



Write the question and the answerand associate this FAQ to a categoryand click “Save”.If you want to add more questions for the same answer (recommended), click on“Add alternative question” button and write it in the text box.

To  associate  the  FAQ  with  a  category,  type  the name  of  the  category  in  the corresponding text field. As long as you enter this name, the existing categories in the FAQ collection of the particular language will appear as suggestions.
You can select the suggested category by clicking on it to associate it with the new FAQand remove the association between this category and this FAQusing the “X” next to it.
If the category you want to add is not yet included in the FAQ collection you are working on, it will be indicated as “Click to add the new category: category\_name”. By clicking on this button, this new category will be added to the collection you are working on and will be associated with the FAQ you are creating. As before, you can remove this association with the new FAQ with the “X”next to it.


\subsection{3.4.2. Edit a FAQ}
\label{\detokenize{pages/Superadmin profile:edit-a-faq}}
Press the edit button of one of the existing FAQs in a collection and a page like the one to create a new FAQ will be shown, but with the content of the FAQ selected to be edited.



The data can be modified in the same way as when creating a new FAQ.


\subsection{3.4.3. Delete a FAQ}
\label{\detokenize{pages/Superadmin profile:delete-a-faq}}
Press the deletebutton of one of the existing FAQs in a collection and a required confirmation popup will appear.
\begin{itemize}
\item {} 
Selecting “Delete” will remove the FAQ from the collection

\item {} 
Selecting “No,cancel” will cancel the operation and will not remove the FAQ from the collection.

\end{itemize}


\section{3.5. Manage datapoints/proms}
\label{\detokenize{pages/Superadmin profile:manage-datapoints-proms}}


In this section you can manage the datapoints of each group of patients of the platform, that is, the content of the Course of the disease (CoD) section of the user profile.

As in the previous sections, the first thing to do is to choose the group of patients you want to work with



Initially, the table on the right will appear, indicating the sections of the CoD page that will be presented to the user. This is the internal navigation menu of this section or the contents that make it up.

From here we can manage these sections: add new, edit or delete existing ones.
\begin{itemize}
\item {} 
The action of deleting requires confirmation, so when pressing the X button of a section a popup will appear to be able to cancel the operation or continue with it.

\item {} 
The actions of adding or editing will open a popup showing the configurable fields: the name, a description, the order in which it will appear for the users regarding the rest of the sections and “enabled section”, that is, if it will be visible or not.

\end{itemize}

Besides this, for each section there is an icon in the “Proms” column. What this does is to show the table on the right, that is, the datapoints that will be included in each section.

Here again you can add, edit or remove datapoints from a section. And these will also be assigned an order of appearance within the section they belong to.

For the definition of a new datapoint you will have to indicate at least:
\begin{itemize}
\item {} 
The name or an identifier

\item {} 
The question field that will be the label or the content that will accompany the datapoint

\item {} 
The order

\item {} 
The size: small, normal, medium or large, with which it will be represented on the screen.

\end{itemize}

There is a requirement in the form of implementation used in Health29: the first datapoint will correspond to a question that will be asked initially to the user in each section. This is important for the calculation of statistics: it is not the same as a patient not answering or indicating that he or she does not have something.

According to what you have answered in this question, several things will happen in the presentation in the user profile:
\begin{itemize}
\item {} 
If you have not answered or the answer is negative, the rest of the section content will not be shown (the rest of the datapoints will be hidden)

\item {} 
If you have answered yes, then the rest of the datapoints will be visible.

\end{itemize}

This behavior is achieved as follows:
\begin{enumerate}
\sphinxsetlistlabels{\arabic}{enumi}{enumii}{}{.}%
\item {} 
The first datapoint must always be assigned the name: “Question” or “Options” and must always correspond to the question explained in the previous paragraph.

\item {} 
The rest of the datapoints must have the field “Related to” assigned to the first datapoint.

\end{enumerate}

In the following image there is a list of the different types of datapoints that have been implemented so far in Health29, or what is the same, the different types of fillable fields can be used in the users’ CoD section. On the one hand we have specific types such as “Title” or “Date”, and on the other hand we have designed combinations between them.



Finally, you can define the translations of these fields by pressing the “Translations” button where, by choosing the group of patients and one of the languages available on the platform, you can modify the text corresponding to each datapoint or section.




\section{3.6. Manage translations}
\label{\detokenize{pages/Superadmin profile:manage-translations}}
In this section you can modify the translations of the texts present in the Health29 platform.
You can choose the language in which the original text will be shown, and the language in which you want to modify the translation.




\section{3.7. Manage information about clinical history}
\label{\detokenize{pages/Superadmin profile:manage-information-about-clinical-history}}
In this section you can manage the relevant aspects of the clinical information of the patient groups.

At the moment it consists of a single section: Medication.



It includes for each patient group the medicines, the side effects and the adverse effects associated with their diagnosis.
What is included in this section is the content that will appear on the Drugs page of the users of a patient group.







Here you can add, edit or delete individual medications and side/adverse effects.

The deletion of any element requires confirmation, that is, when you press the “X” of a drug, a side effect or an adverse effect a popup is displayed indicating that the element is going to be deleted and whether you want to confirm or cancel the operation.
In addition, a button has been included in drugs and side effects to remove all items that also require confirmation.

The first time you access it, a field appears on the right to add new elements: drugs, side effects or adverse effects.

Thus, when you want to add a drug, the section on the right shows the fields to be filled in:
\begin{itemize}
\item {} 
First name or identifier and other text fields for translations in the different languages available on the platform.

\item {} 
A selectable field to choose the side effects related to this drug.

\end{itemize}

For the side effects and adverse effects you only have to fill in the text fields related to the identifier and the languages.

When you select the edition of a drug, a side effect or an adverse effect, these fields appear in an analogous way for each case in the section on the right but with the information of the selected element. The fields can be modified and by pressing the “Update” button the values will be updated. Once the changes have been confirmed, you will return to the screen to add new element.


\section{3.8. Manage notifications}
\label{\detokenize{pages/Superadmin profile:manage-notifications}}


From this page the superadministrator will be able to create notifications to be sent to all users of a selected patient group.



Here you can add, edit or delete notifications.

Deleting will require confirmation, and creating or editing one will open a new page where the specified fields will have to be filled in:



The general architecture of a notification will be:
\begin{itemize}
\item {} 
Type: Notifications of the administrator type or patient group or periodical (6 or 12 months).

\item {} 
Launch date: The date from which the notification will be launched

\item {} 
Identifier: a number, word or short text to identify the notification.

\item {} 
Title and text: By selecting add translations and the languages in which you want to display the notification, these fields appear for each language. The title is a brief description to identify the notification in the subscriptions view, and the text is what will appear in the user notification window.

\item {} 
URL. This field is not mandatory, it allows to add an external URL to appear in the notification to the users or an internar URL with “\#” and name of the section. It will be composed by the URL and a title that will appear in the button that the users will see next to the notification.

\end{itemize}


\section{3.9. Support}
\label{\detokenize{pages/Superadmin profile:support}}
When users or administrators of a patient group send a message to support, it will be displayed on this page.



The information is displayed in table format:
\begin{itemize}
\item {} 
The email address of the user or administrator who sent the message

\item {} 
A message subject

\item {} 
The content or text

\item {} 
The files in the case of any attachments

\item {} 
The date of the message

\item {} 
The state: unread, read and solved.

\item {} 
The status update date

\item {} 
The type of issue, if applicable.

\item {} 
The language

\end{itemize}




\chapter{4. Technical debt}
\label{\detokenize{pages/Technical debt:technical-debt}}\label{\detokenize{pages/Technical debt::doc}}
There are some aspects that have remained pending during the implementation of the platform. These are both new functionalities and the modification of some implementations that, due to lack of time, we do not consider optimal.

Thus, this section will list the proposals of the technical team or the user feedback gathered to date.
It should be noted that the content of this section is of a high technical level, so it is necessary to understand the architecture of the application at the level of the code in order to follow it.


\section{4.1. Modifications requiered}
\label{\detokenize{pages/Technical debt:modifications-requiered}}\begin{itemize}
\item {} 
User has to reload page to be able to be able to login (all times but the first time)

\item {} 
Create registration such us in this \sphinxhref{https://docs.microsoft.com/en-us/rest/api/notificationhubs/create-registration}{link} for native android/iOS notifications.

\item {} 
Alerts are not shown until the site is refreshed

\item {} 
Think about providing the user with the possibility of snooze notifications “freely”. Fields “number” of “hours?/days/months”

\item {} 
Deploy and test pipelines.

\item {} 
Translation management.

\end{itemize}


\section{4.2. New items}
\label{\detokenize{pages/Technical debt:new-items}}\begin{itemize}
\item {} 
Incorporate \sphinxhref{https://docs.microsoft.com/en-US/azure/healthcare-apis/}{FHIR} for data storage.

\end{itemize}



\renewcommand{\indexname}{Index}
\printindex
\end{document}